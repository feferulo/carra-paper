\subsection{Regular Categories}
We will denote with $\cc{C}at$ the category of small categories, 
and $\cc{E}ns$ the category of sets. 

Let $\cc{C}$ be a small category.  In what follows all the diagrams are in 
$\cc{C}$.

\vspace{1ex}

Recall the notion of \emph{strict epimorphism} introduced in 
SGA4 \cite{sga4}[10.3, p. 180].

%\cite{[sga4, 10.3, p. 180]} \

\vspace{1ex}

\begin{definition}
Given a morphism ${X \mr{f} }Y$,  a morphism ${X \mr{g} Z }$  is \emph{compatible} if for every pair ${Z \mrpair{x}{y} X}$ such that 
${fx=fy}$ it follows that ${gx=gy}$.

\noindent A morphism ${X \mr{f} Y}$ is a \emph{strict epimorphism} if for every compatible 
${X \mr{g} Z}$,  there exists a unique ${Y \mr{h} Z }$ such that 
$h \circ f = g$.
The situation is described in a diagram:

\vspace{-1ex}

% where the 
%family $g_{\alpha}$ is compatible with the family $f_{\alpha}$:
%$$        
%x\ymatrix@1
%        {
%         X _{\alpha} \;\; \ar @<+2pt> `u[r] `[rr]^{g_{\alpha}} [rr]
%                                             \ar[r]^{f_{\alpha}} 
%         & \;\;X\;\;  \ar@{-->}[r]^{\exists ! g} 
%         & \;\;Y      
%        }
%$$

$$        
\xymatrix@1
        {
         X \;\; \ar @<+2pt> `u[r] `[rr]^{\forall \,  g \; compatible} [rr]
                 \ar[r]^{f} 
         & \;\;Y\;\;  \ar@{-->}[r]^{\exists \, ! \,  h} 
         & \;\;Z      
        }
$$
\end{definition}

%\noindent We will use the symbol $\xymatrix{ \ar @{->>} [r] & }$ to label strict epimorphisms.


A morphism ${X \mr{f} Y }$ is a monomorphism precisely when  $id_X$ is compatible. Strict epimorphisms are epimorphisms,   
if ${X \mr{f} Y}$ is a monomorphism  and a strict epimorphism, then it is an isomorphism. Strict epimorphisms do not compose in general. 

\begin{remark}\label{kernelpair}
For any arrow $X \mr{f} Y$, let $R_f \mrpair{}{} X$ be a kernel pair of $f$.
Then $f$ is a strict epimorphism if and only if the diagram 
$R_f \mrpair{}{} X \mr{f} Y$ 
is a coequalizer.
It follows that when pullbacks exists \emph{a functor that preserves pull-backs and coequalizers preserves strict epimorphisms.}
\end{remark}
%\end{document}

\vspace{1ex}

Recall the definition of regular category:
\begin{definition} \label{regular}
A category is \emph{regular} if it has finite limits, any arrow can be factorized into a monomorphism
composed with an  strict epimorphism, and strict
epimorphisms  are universal.
\end{definition}


\begin{remark} \label{regular2}
Given $X \mr{f} Y \mr{g} Z$ in a regular category, the following holds:

1) If $f$ and $g$ are strict epimorphisms, so it is the composite 
$g \circ f$.

2) If the composite $g \circ f$ is a strict epimorphisms, so it is $g$.
\end{remark}

%\end{document}
%%%%%%%%%%%%%%%%%%%%%%%%%% COMMENT %%%%%%%%%%%%%%%%%%%%%%%%%%%%%%%%%%% 
\begin{comment}      
We will denote $\cc{C}at_{fl}$ the subcategory of $\cc{C}at$ whose objects are finitely complete categories and whose morphisms are limit-preserving functors. Recall that limit preserving functors preserve monomorphisms

\vspace{1ex}

We will denote $\cc{R}eg$ the subcategory of $\cc{C}at$ whose objects are regular categories and whose morphisms are limit-preserving functors that preserve strict factorizations. We call the morphisms in $\cc{R}eg$ \textit{regular} functors.   $\cc{R}eg$ is in fact a subcategory of  
$\cc{C}at_{fl}$.
\end{comment}
%%%%%%%%%%%%%%%%%%%%%%%%%%%%%%%%%%%%%%%%%%%%%%%%%%%%%%%%%%%%%%%%%%%%%%

\begin{definition}
A functor between regular categories is a \emph{regular functor} if it preserves finite limits (hence it preserves monics) and either one of the two following equivalent conditions hold:

1) $F$ preserves strict factorizations.

2) $F$ preserves strict epimorphisms.
\end{definition}

\begin{definition} \label{globally supported}
An object $X$ is \emph{of global support} if for all terminal objects the morphism $X \mr{} 1$ is a strict epimorphism, or, equivalently, if there exists a terminal object such that the morphism $X \mr{} 1$ is a strict epimorphism.
\end{definition}

\begin{proposition}
A finite product is an object of global support if and only if each factor is of global support. Furthermore, the projections are strict epimorphisms.
\end{proposition}
\begin{proof}
For a binary product the statement follows immediately from 
\ref{regular} and \ref{regular2}. Then proceed by induction.
\end{proof}






