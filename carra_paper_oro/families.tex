\subsection{Regular Categories}
In this section we will introduce the basic concepts we need to define  what a regular category is and prove some basic properties these categories satisfy. We will denote with $\cc{C}at$ the category of small categories whose morphisms are functors and $\cc{E}ns$ the category of small sets. 

\subsubsection{Definition}

Let $\cc{C}$ be a small category.  In what follows all the diagrams are in $\cc{C}$.


\begin{definition}
$\cc{C}$ is \textbf{finitely complete} if finite limits exist.
\end{definition}

\begin{remark}
$\cc{C}$ is finitely complete if and only if pullbacks exist and a terminal object exists.
\end{remark}

\noindent We will denote $\cc{C}at_{fl}$ the subcategory of $\cc{C}at$ whose objects are finitely complete categories and whose morphisms are limit-preserving functors.

\begin{observation}
A morphism $X \mr{f} Y$ is monic if and only if the following diagram is a pullback.

\begin{align*}
\xymatrix @+3.5ex {X  \ar [r] ^{id_X}  \ar [d] _{id_X}  \ar @{} [dr] |{p.b.}  &  X  \ar [d] ^{f}  \\
		   X  \ar [r] _{f}  &  Y }
\end{align*}
\end{observation}

\begin{proposition}  \label{limit-preserving functors preserve monics}
If ${\cc{C} \mr{F} \cc{D} \in \cc{C}at_{fl}}$, then $F$ preserves monics.
\end{proposition}

\begin{proof}
It is an immediate consequence from the previous observation.
\end{proof}

For what follows it is useful to have the following result present.

\begin{lemma} \label{mono + retraccion = isomorphism}
If a monic $\xymatrix {X  \ar  [r] ^{f}  &  Y }$ admits a section, that is there exists a morphism ${Y \mr{h} X}$ such that $fh=id_Y$, then $f$ is an isomorphisms.
\end{lemma}


\begin{proof}
We have that $f \cdot hf=fh \cdot f=id_Y \cdot f=f \cdot id_X$. Since  $f$ is monic it follows that $hf=id_X$.

\end{proof}





\begin{definition}
For morphisms ${X \mr{f} }Y$ and ${X \mr{g} Z }$ we say that $g$ is \textbf{compatible} with $f$ if for every pair ${W \mrpair{x}{y} X \in \cc{C}}$ such that ${fx=fy}$ it follows that ${gx=gy}$.
\end{definition}

\noindent We will also express this by saying that $g$ is $f$-compatible.


\begin{definition}
A morphism ${X \mr{f} Y}$ is a \textbf{strict epimorphism} if for every $f$-compatible morphism ${X \mr{g} Z}$  there exists a unique morphism ${Y \mr{h} Z }$ such that $hf=g$.



\hspace{2.cm} $
\xymatrix @+5ex {X \ar [r] ^f \ar [d] _{\forall\ f-compatible\ g} & Y \ar @{-->} [dl] ^{\exists !} \\
		  Z & \ar @{} [ul] |(.7) \equiv}
$
\end{definition}

\noindent We will use the symbol $\xymatrix{ \ar @{->>} [r] & }$ to label strict epimorphisms.

\begin{observation}
Strict epimorphism are epimorphism.
\end{observation}



\begin{observation} \label{equivalencia monomorfismo con identidad compatible}
A morphism ${X \mr{f} Y }$ is monic if and only if $id_X$ is $f$-compatible. 
\end{observation}



\begin{proposition} \label{monic + strict epic = isomorphism}
If ${X \mr{f} Y}$ is monic and a strict epimorphism , then $f$ is an isomorphism.
\end{proposition}

\begin{proof}
We will prove that $f$ admits a section. From the previous observation we know there is a unique morphism ${Y \mr{h} X}$ such that the following diagram commutes.

\begin{align*}
\xymatrix @+5ex {X \ar [r] ^f \ar [d] _{id_X} & Y \ar [dl] ^{h} \\
		  X & \ar @{} [ul] |(.7) \equiv}
\end{align*}

\noindent Thus we have that ${fh \cdot f=f \cdot hf=f \cdot id_X=id_Y \cdot f}$. Since $f$ is epic we conclude that $fh=id_Y$. The result follows.

\end{proof}


We will use the symbol $\xymatrix {\ar @{>->} [r] & }$ to label monomorphisms. A monomorphism $\xymatrix {A\ \ar @{>->} [r] & Y }$ will be called a \textit{subobject} of Y. Consider the following commutative diagram. 

\begin{align*}
\newdir{(>}{{}*!/-6pt/\dir{>}}
\xymatrix @+2.5ex {X \ar [rr] ^{f} \ar [dr] _{e} && Y \\ 
		  & A \ar @{} [u] | (.6) \equiv \ar @{(>->} [ur] _{m} }
\end{align*}

\noindent We will say the pair $\xymatrix {X \ar [r] ^{e} & A\ \ar @{>->} [r] ^{m} & Y }$ is a \textit{factorization of $f$ through a subobject of $Y$}. If $e$ is a strict epimorphism we call the pair $\xymatrix {X \ar @{->>} [r] ^{e} & A\ \ar @{>->} [r] ^{m} & Y }$ a \textit{strict} factorization of $f$.

\begin{remark}
A strict factorization $\xymatrix {X \ar @{->>} [r] ^{e} & A\ \ar @{>->} [r] ^{m} & Y }$ of $f$ is universal with respect to every factorization of $f$ throught subobjects of $Y$. That is if $\xymatrix {X \ar [r] ^{e'} & A'\ \ar @{>->} [r] ^{m'} & Y }$ is a factorization of $f$ through a subobject of $Y$, then there exists a unique $A \mr{s} A'$ such that $se=e'$ and $m's=m$.
\end{remark}

Consider the following pullback diagram.
\begin{align*}
 \xymatrix @+3.5ex {A \ar [r] \ar [d] _{f'} & B \ar [d] ^{f} \\ 
		   D \ar [r] _{g} \ar @{} [ur] |{p.b.} & C } 
\end{align*}

\noindent We will call $f'$ a \textit{pullback of $f$} \textit{along} $g$ or simply a \textit{pullback of $f$}.

\begin{definition}
Strict epimorphisms are \textbf{stable} if a pullback of a strict epimorphism is a strict epimorphism.
\end{definition}

\begin{definition}
$\cc{C}$ is \textbf{regular} if finite limits exist, strict factorizations exist for every morphism and strict epimorphisms are stable.
\end{definition}


\noindent We will denote $\cc{R}eg$ the subcategory of $\cc{C}at$ whose objects are regular categories and whose morphisms are limit-preserving functors that preserve strict factorizations. We call the morphisms in $\cc{R}eg$ \textit{regular} functors.   $\cc{R}eg$ is in fact a subcategory of  $\cc{C}at_{fl}$.

 
\subsubsection{A few facts} 
 In what follows $\cc{C}$ represents a regular category.



\begin{proposition}\label{strict epics are closed under composition}
Strict epimorphisms are closed under composition.
\end{proposition}

\begin{proof}
Take composable strict epimorphisms $\xymatrix {X \ar @{->>} [r] ^{f} & Y \ar @{->>} [r] ^{g} & Z }$ and take $\xymatrix {X \ar @{->>} [r] ^{e} & A\ \ar @{>->} [r] ^{m} & Z }$ a  strict factorization of $gf$. We will prove $m$ is an isomorphism. It suffices to prove that $m$ admits a section. We have that  $e$ is $f$-compatible because $m$ is monic. Thus there exists a unique $h$ such that $hf=e$.

  
\begin{align*}
\newdir{(>}{{}*!/-6pt/\dir{>}}
\xymatrix @+5ex {X  \ar @{->>} [r] ^{f}  \ar @{->>} [dr] _{e} &  Y  \ar @{->>} [r] ^{g}  \ar @{-->} [d] _{\exists ! h}  \ar @{} [dl] |(.3){\equiv}  \ar @{} [dr] |(.3){\equiv}  &  Z  \\
		   & A  \ar @{(>->} [ur] _{m}   &  }
\end{align*}


$$
\xymatrix
     {
      X \ar[r]^{P} \ar[rd]^{t} 
    & Y \ar[d]^{} \ar[r]^{u} 
    & Z
   \\
    & A \ar[ru]^{a}
     }
$$
   


\noindent  Since $f$ is epic we also have that $mh=g$. Because $m$ is monic $h$ is $g$-compatible. Thus there exists a unique $Z \mr{h'} A$ such that $h'g=h$. Composing with $m$ we obtain that $mh' \cdot g= id_Y \cdot g $ and since $g$ is epic we have that $mh'=id_Y$. The result follows.

\end{proof}


\begin{definition} \label{globally supported}
An object $X$ is \textbf{globally supported} if for every terminal object $1$ the unique morphism $X \mr{} 1$ is a strict epimorphism.
\end{definition}

\begin{observation}
If for any terminal object $1$ the unique morphism $X \mr{} 1$ is a strict epimorphism, then $X$ is globally supported.
\end{observation}


 Let $[n]=\{0,1,2,...,n-1\}$ denote the finite ordinal. Take a product $\{P \mr{\pi_i} B_i\}_{i \in [n]}$ of the family $\Bi$. 
 
\begin{observation} 
For $B_{n} \in \cc{C}$ we can calculate a product of the family $\{B_i\}_{i \in [n+1]}$ taking the following pullback.

\begin{align*}
\xymatrix{P'  \ar [r]  \ar [d] _{\hat{\pi}_n}  &  B_{n}  \ar[d]  \\
		  P  \ar [r]  \ar @{} [ur] |{p.b.}  &  1}
\end{align*} 

The morphism $\hat{\pi}_n$ can be interpreted as \textit{truncating the $n^{th}$ coordinate}. 
\end{observation} 
 
\begin{lemma}
If $B_{n}$ is globally supported, then truncating the $n^{th}$ coordinate $P' \mr{\hat{\pi}_n} P$ is a strict epimorphism.
\end{lemma}

\begin{proof}
It is immediate from the previous diagram.
\end{proof}

\begin{corollary} \label{projectar es epi estricto}
If the objects of the family $\Bi$ are globally supported, then every projection $P \mr{\hat{\pi}_i} B_i$ is a strict epimorphism. Consequently $P$ is globally supported.
\end{corollary}

\begin{proof}
The result follows from that fact that strict epimorphisms are closed under composition and that truncating one coordinate at a time we reach to the desired projection. 
\end{proof}

The following result will be useful throughout the thesis.
\begin{proposition}
If  $\cc{D}$ is a regular category and $\cc{C} \mr{F} \cc{D}$ is  a limit-preserving functor, the the following are equivalent:

\begin{itemize}
\item  $F$ is a regular functor.
\item  $F$ preserves strict epimorphisms.
\end{itemize}

\end{proposition}

\begin{proof}
It follows from Proposition \ref{limit-preserving functors preserve monics}.
\end{proof}

