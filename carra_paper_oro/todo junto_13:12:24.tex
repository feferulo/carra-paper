\documentclass[12pt]{article}
%\documentclass[a4paper, 12pt]{amsart}
%\documentclass{book}

%[a4paper, 11pt]


\usepackage{geometry}\geometry{top=5cm,bottom=2cm,left=3cm,right=3cm}
\usepackage{amssymb}
\usepackage{amsmath}
\usepackage{graphicx}


\sloppy
\input xy   
\xyoption{all}
%\swapnumbers
\usepackage{amsmath, amsthm, amsfonts, amssymb}
\usepackage{enumerate}
\usepackage{stackrel}
\usepackage{colordvi}
\usepackage{hyperref}
%\usepackage[latin1]{inputenc}
\usepackage[english]{babel}
\usepackage{color}
\usepackage{verbatim}
\usepackage[all]{xy}
\usepackage{graphicx}
\usepackage{stmaryrd}

\numberwithin{equation}{section}
\theoremstyle{plain}
  \begingroup
        \newtheorem{theorem}[equation]{Theorem}
        \newtheorem{lemma}[equation]{Lemma}
        \newtheorem{proposition}[equation]{Proposition}
        \newtheorem{corollary}[equation]{Corollary}
        \newtheorem{fact}[equation]{Fact}
        \newtheorem{facts}[equation]{Facts}
        \newtheorem{assumption}[equation]{Assumption}
        \newtheorem{remarkitalica}[equation]{Remark}
	    \newtheorem{definition}[equation]{Definition}
        \newtheorem{notation}[equation]{Notation}
        \newtheorem{convention}[equation]{Convention}
        \newtheorem{sinnadaitalica}[equation]{}
        \newtheorem{construction}[equation]{Construction}

\endgroup

\theoremstyle{definition}
  \begingroup
        \newtheorem{remark}[equation]{Remark}
        \newtheorem{example}[equation]{Example}
        \newtheorem{sinnadastandard}[equation]{}
        \newtheorem*{observation}{Observation}
        \newtheorem{nobservation}[equation]{Observation}
        \newtheorem{definitionst}[equation]{Definition}
\endgroup


\newcommand{\cqd}{\hfill$\Box$}  %fin de demostracion%

\newcommand{\cqdhere}{$\Box$}    %fin de demostracion en el lugar%

\newcommand{\car}{\circ}    

\newcommand{\rojo}{\textcolor{red}}  %comentario%
\newcommand{\azul}{\textcolor{blue}}
\newcommand{\blanco}{\textcolor{white}}
\newcommand{\verde}{\textcolor{green}}
\newcommand{\rosa}{\textcolor{magenta}}

\newcommand{\erojo}{\color{red}}  %comentario%
\newcommand{\eazul}{\color{blue}}
\newcommand{\eblanco}{color{white}}
\newcommand{\everde}{color{green}}
\newcommand{\erosa}{\color{magenta}}


              % comandos para letras en math mode %

\newcommand{\cc}{\mathcal}
\newcommand{\bb}{\mathbb}
\newcommand{\ff}{\mathsf}
\newcommand{\nn}{\mathbf}
\newcommand{\rr}{\mathrm}

\newcommand{\lm}{\ell}
\newcommand{\im}{\imath}
\newcommand{\jm}{\jmath}

                   % comandos para flechas %
                   

%\newcommand{\Mr}[1]{\stackrel{#1}{\Rightarrow}}
%\newcommand{\Mrl}[1]{\stackrel{#1}{\Longrightarrow}}
%\newcommand{\Ml}[1]{\stackrel{#1}{\Leftarrow}}
%\newcommand{\Mll}[1]{\stackrel{#1}{\Longleftarrow}}
\newcommand{\xr}[1]{\xrightarrow{#1}}
\newcommand{\xl}[1]{\xleftarrow{#1}}
\newcommand{\Xr}[1]{\xRightarrow{#1}}
\newcommand{\Xl}[1]{\xLeftarrow{#1}}
\newcommand{\mmmr}[1]{\buildrel {#1} \over \hookrightarrow}
\newcommand{\mvr}[1]{\xymatrix{{} \ar@{~>}[r]^{#1} & {}}}


\newcommand{\mr}[1]{\overset {#1} {\longrightarrow}}
\newcommand{\ml}[1]{\overset {#1} {\longleftarrow}}

\newcommand{\tocar}{\xymatrix{{} \ar[r]|\car & {}}}

\newcommand{\mmr}[1]{\overset {#1} {\hookrightarrow}}

\newcommand{\Mr}[1]{\overset {#1} {\Longrightarrow}}
\newcommand{\Ml}[1]{\overset {#1} {\Longleftarrow}}

\newcommand{\mrpairviejo}[2]
   {
    \xymatrix@C=5ex@R=2.4ex
            {
             {} \ar@<1.6ex>[r]^{#1} 
	            \ar@<-1.1ex>[r]^{#2} 
	         & {}
            }
   }


\newcommand{\mrpair}[2]
   {
    \xymatrix@C=5ex@R=2.4ex
            {
             {} \ar@<.5ex>[r]^{#1} 
	            \ar@<-.5ex>[r]_{#2} 
	         & {}
            }
   }


\newcommand{\mlpair}[2]
   {
    \xymatrix@C=5ex@R=2.4ex
            {
             {} 
              & {} \ar@<1.0ex>[l]_{#2} 
	          \ar@<-1.7ex>[l]_{#1}
            }
    }

                    % comandos para 2-celdas %

\newcommand{\cellrd}[3] % flechas para adelante, cell para abajo %
 {
  \xymatrix@C=7ex@R=2.4ex
         {
          {} \ar@<1.6ex>[r]^{#1} 
             \ar@{}@<-1.3ex>[r]^{\!\! {#2} \, \!\Downarrow}
             \ar@<-1.1ex>[r]_{#3} 
          & {}
         }
 }
 \newcommand{\modif}[3] % flechas para adelante, cell para abajo %
 {
  \xymatrix@C=7ex@R=2.4ex
         {
          {} \ar@<1.6ex>@{=>}[r]^{#1} 
             \ar@{}@<-1.3ex>@{=>}[r]^{\!\! {#2} \, \!\downarrow}
             \ar@<-1.1ex>[r]_{#3} 
          & {}
         }
 }

\newcommand{\cellld}[3] % flechas para atras, cell para abajo %
 {
  \xymatrix@C=6ex@R=2.4ex
         {
            {} 
          & {} \ar@<1.0ex>[l]^{#3} 
          \ar@{}@<-1.7ex>[l]^{\!\! {#2} \, \!\Downarrow}
	                                 \ar@<-1.7ex>[l]_{#1}
         }
 }

\newcommand{\cellpairrd}[4] % como \cellrd pero con dos cells %
 {
  \xymatrix@C=10ex@R=2.4ex
         {
          {} \ar@<1.6ex>[r]^{#1} 
             \ar@{}@<-1.3ex>[r]^{\!\! {#2} \, \!\Downarrow 
                                 \;\; {#3} \, \!\Downarrow }
             \ar@<-1.1ex>[r]_{#4} 
          & {}
         }
 }

                 % comandos para bilimites %



\newcommand{\coLim}[2]
   {
    \underset{#1}{\underrightarrow{\ff{Lim}}}
    \; {#2}
   }

\newcommand{\Lim}[2]
   {
    \underset{#1}{\underleftarrow{\ff{Lim}}}
    \; {#2}
   }

   
\newcommand{\biLim}[2]
   {
    \underset{#1}{\underleftarrow{\ff{biLim}}}
    \; {#2}
   }
   % comandos para limites % 

\newcommand{\bicoLim}[2]
   {
    \underset{#1}{\underrightarrow{\ff{biLim}}}
    \; {#2}
   }
\newcommand{\colimite}[2]
   {
    \underset{#1}{\underrightarrow{lim}}
    \; {#2}
   }

\newcommand{\limite}[2]
   {
    \underset{#1}{\underleftarrow{lim}}
    \; {#2}
   }


 % COMANDOS DE ASCENSORES PARA USAR DENTRO DE UN \xymatrix %
 %                Fijarse las redefiniciones:              %
 %          s = source, t = target, d = down, u = up       %
 %     la regla es de cuantas s a cuantas t va la celda    %
 
\newcommand{\dst}{\dcell}
\newcommand{\dcell}[1]  % celda down %
          {
					 \ar@<8pt>@{-}[d]+<-4pt,8pt> 
           \ar@<-8pt>@{-}[d]+<4pt,8pt>
           \ar@{}[d]|{#1}
          }

\newcommand{\dstb}{\dcellb}
\newcommand{\dcellb}[1]   % celda ancha down %
          {
           \ar@<10pt>@{-}[d]+<-5pt,8pt> 
           \ar@<-10pt>@{-}[d]+<5pt,8pt>
           \ar@{}[d]|{#1}
          }

\newcommand{\deq}        % identidad down %
         {
          \ar@{=}[d]
         }
        
\newcommand{\dreq}       % identidad down right %
         {
          \ar@{=}[dr]
         }

\newcommand{\dleq}       % identidad down left %
         {
          \ar@{=}[dl]
         }

\newcommand{\dstt}{\dccell}
\newcommand{\dccell}[1]    % celda de uno a dos lugares down % 
          {
           \ar@{-}[ld] 
           \ar@{-}[rd] 
           \ar@{}[d]|{#1}  
          }

\newcommand{\dcellbb}[1]   % celda mas ancha down %
          {
           \ar@<20pt>@{-}[d]+<-10pt,12pt> 
           \ar@<-20pt>@{-}[d]+<10pt,12pt>
           \ar@{}[d]|{#1}
          } 
%\newcommand{\dsstt}{\dcccell}
%\newcommand{\dcccell}[1]    % celda de dos a dos lugares down % 
%          {                        % HAY QUE MEJORARLA %
%           \ar@<6.5pt>@{-}'+<0pt,-2pt>[d] 
%           \ar@<-6.5pt>@{-}'+<0pt,-2pt>[d] 
%           \ar@{}[d]|{#1}  
%          }



%%%%%%%%PARA HACER CELDAS DEL TAMANO QUE UNO QUIERA%%%%%%%%%%%%%%
\newcommand{\dl}    % una raya para poner a la izquierda (inclinada) % 
          {                        
           \ar@<-2pt>@{-}[d]+<4pt,8pt>
          }

\newcommand{\dr}    % una raya para poner a la derecha (inclinada) % 
          {                        
           \ar@<2pt>@{-}[d]+<-4pt,8pt> 
          }
\newcommand{\dc}[1]    % para poner el label en el centro % 
          {                        
           \ar@{}[d]|{#1}  
          }
\newcommand{\dcr}[1]    % para poner el label en el centro a la derecha para tamanos impares% 
          {                        
           \ar@{}[dr]|{#1}  
          }



\newcommand{\ustt}{\uccell}
\newcommand{\uccell}[1]      % celda de uno a dos lugares up %
          { 
           \ar@{-}[ur] 
           \ar@{}[u]|{#1} 
           \ar@{-}[ul] 
          }

\newcommand{\usttb}{\uccellb}          
\newcommand{\uccellb}[1]     % celda ancha de uno a dos lugares up %
          { 
           \ar@<-1ex>@{-}[ur] 
           \ar@{}[u]|{#1} 
           \ar@<1ex>@{-}[ul] 
          }


\newcommand{\dcellop}[1]  % celda down abriendo %
          {
					 \ar@<6pt>@{-}[d]+<6pt,8pt> 
           \ar@<-6pt>@{-}[d]+<-6pt,8pt>
           \ar@{}[d]|{#1}
          }


\newcommand{\dcellopb}[1]  % celda down abriendo mas ancha%
          {
					 \ar@<7pt>@{-}[d]+<7pt,8pt> 
           \ar@<-7pt>@{-}[d]+<-7pt,8pt>
           \ar@{}[d]|{#1}
          }

\newcommand{\did}{\ar@2{-}[d]}

\newcommand{\op}[1]
          {
           \ar@{-}[ld] 
           \ar@{-}[rd] 
           \ar@{}[d]|{#1}  
          }

\newcommand{\cl}[1]
          { 
           \ar@{-}[ur] 
           \ar@{}[u]|{#1} 
           \ar@{-}[ul] 
          }


\newcommand{\Pro}[1]{2\hbox{-}\cc{P}ro(\cc{#1})}

\newcommand{\tc}{\textcolor}

	\newcommand{\Bi}
		{
		\{B_i\}_{ i \in [n]}
		}
		
\newcommand{\Cj}
		{
		\{C_j\}_{ j \in [m]}
		}

\newcommand{\mnr}[1]{\xymatrix{{} \ar@{>->}[r]^{#1} & {}}}%monos comunes

\newcommand{\intnat}[1]{[ {#1} ]} %intervalo natural hasta n

%\newcommand{\subC}[2]{\cc{#1}_{\cc{C}^{#2}}}
\newcommand{\subC}[2]{\cc{#1}_{\cc{A}^{#2}}}


\begin{document} 

 %    puta 

%\end{document}

\title{Demostraci\'on constructiva del teorema de Barr}

\author{Federico Carr\'a, Eduardo J. Dubuc}

\date{\vspace{-5ex}}

\maketitle

\begin{abstract}
xx%change abstract just to check editor works well ////
\end{abstract}



\input{introduction_v2.tex}

%\pagebreak

\tableofcontents


%\pagebreak
%%% Body of the thesis

\section{CATEGORICAL PRELIMINARIES}\label{preliminares}

\subsection{Regular Categories}
We will denote with $\cc{C}at$ the category of small categories, 
and $\cc{E}ns$ the category of sets. 

Let $\cc{C}$ be a small category.  In what follows all the diagrams are in 
$\cc{C}$.

\vspace{1ex}

Recall the notion of \emph{strict epimorphism} introduced in 
SGA4 \cite{sga4}[10.3, p. 180].

%\cite{[sga4, 10.3, p. 180]} \

\vspace{1ex}

\begin{definition}
Given a morphism ${X \mr{f} }Y$,  a morphism ${X \mr{g} Z }$  is \emph{compatible} if for every pair ${Z \mrpair{x}{y} X}$ such that 
${fx=fy}$ it follows that ${gx=gy}$.

\noindent A morphism ${X \mr{f} Y}$ is a \emph{strict epimorphism} if for every compatible 
${X \mr{g} Z}$,  there exists a unique ${Y \mr{h} Z }$ such that 
$h \circ f = g$.
The situation is described in a diagram:

\vspace{-1ex}

% where the 
%family $g_{\alpha}$ is compatible with the family $f_{\alpha}$:
%$$        
%x\ymatrix@1
%        {
%         X _{\alpha} \;\; \ar @<+2pt> `u[r] `[rr]^{g_{\alpha}} [rr]
%                                             \ar[r]^{f_{\alpha}} 
%         & \;\;X\;\;  \ar@{-->}[r]^{\exists ! g} 
%         & \;\;Y      
%        }
%$$

$$        
\xymatrix@1
        {
         X \;\; \ar @<+2pt> `u[r] `[rr]^{\forall \,  g \; compatible} [rr]
                 \ar[r]^{f} 
         & \;\;Y\;\;  \ar@{-->}[r]^{\exists \, ! \,  h} 
         & \;\;Z      
        }
$$
\end{definition}

%\noindent We will use the symbol $\xymatrix{ \ar @{->>} [r] & }$ to label strict epimorphisms.


A morphism ${X \mr{f} Y }$ is a monomorphism precisely when  $id_X$ is compatible. Strict epimorphisms are epimorphisms,   
if ${X \mr{f} Y}$ is a monomorphism  and a strict epimorphism, then it is an isomorphism. Strict epimorphisms do not compose in general. 

\begin{remark}\label{kernelpair}
For any arrow $X \mr{f} Y$, let $R_f \mrpair{}{} X$ be a kernel pair of $f$.
Then $f$ is a strict epimorphism if and only if the diagram 
$R_f \mrpair{}{} X \mr{f} Y$ 
is a coequalizer.
It follows that when pullbacks exists \emph{a functor that preserves pull-backs and coequalizers preserves strict epimorphisms.}
\end{remark}
%\end{document}

\vspace{1ex}

Recall the definition of regular category:
\begin{definition} \label{regular}
A category is \emph{regular} if it has finite limits, any arrow can be factorized into a monomorphism
composed with an  strict epimorphism, and strict
epimorphisms  are universal.
\end{definition}


\begin{remark} \label{regular2}
Given $X \mr{f} Y \mr{g} Z$ in a regular category, the following holds:

1) If $f$ and $g$ are strict epimorphisms, so it is the composite 
$g \circ f$.

2) If the composite $g \circ f$ is a strict epimorphisms, so it is $g$.
\end{remark}

%\end{document}
%%%%%%%%%%%%%%%%%%%%%%%%%% COMMENT %%%%%%%%%%%%%%%%%%%%%%%%%%%%%%%%%%% 
\begin{comment}      
We will denote $\cc{C}at_{fl}$ the subcategory of $\cc{C}at$ whose objects are finitely complete categories and whose morphisms are limit-preserving functors. Recall that limit preserving functors preserve monomorphisms

\vspace{1ex}

We will denote $\cc{R}eg$ the subcategory of $\cc{C}at$ whose objects are regular categories and whose morphisms are limit-preserving functors that preserve strict factorizations. We call the morphisms in $\cc{R}eg$ \textit{regular} functors.   $\cc{R}eg$ is in fact a subcategory of  
$\cc{C}at_{fl}$.
\end{comment}
%%%%%%%%%%%%%%%%%%%%%%%%%%%%%%%%%%%%%%%%%%%%%%%%%%%%%%%%%%%%%%%%%%%%%%

\begin{definition}
A functor between regular categories is a \emph{regular functor} if it preserves finite limits (hence it preserves monics) and either one of the two following equivalent conditions hold:

1) $F$ preserves strict factorizations.

2) $F$ preserves strict epimorphisms.
\end{definition}

\begin{definition} \label{globally supported}
An object $X$ is \emph{of global support} if for all terminal objects the morphism $X \mr{} 1$ is a strict epimorphism, or, equivalently, if there exists a terminal object such that the morphism $X \mr{} 1$ is a strict epimorphism.
\end{definition}

\begin{proposition}
A finite product is an object of global support if and only if each factor is of global support. Furthermore, the projections are strict epimorphisms.
\end{proposition}
\begin{proof}
For a binary product the statement follows immediately from 
\ref{regular} and \ref{regular2}. Then proceed by induction.
\end{proof}








\subsection{Multislice Categories of Regular Categories} \label{slices}
In this section we will define what a multislice category is and prove that  multislice categories of finitely complete categories and  of regular categories are finitely complete and regular, respectively.

\begin{definition}
For a category $\cc{C}$ and a family $\Bi$ of objects of $\cc{C}$ we define the \emph{multislice category} $\cc{C}_{/\Bi}$ whose objects are families 
${\{X \mr{x_i} B_i\}_{ i \in [n]}}$ of arrows of $\cc{C}$ and morphisms ${\{X \mr{x_i} B_i\}_{ i \in [n]} \mr{f} \{Y \mr{y_i} B_i\}_{ i \in [n]}}$ are arrows $X \mr{f} Y$ in $\cc{C}$ such that for every ${ i \in [n]}$

\[
\xymatrix{X \ar[rr]^f \ar[dr]_{x_i} && Y \ar[dl]^{y_i}
\\
		    & B_i \ar@{}[u]|(.6)\equiv & }
\]
\end{definition}

\begin{remark} \label{Sigma}
Clearly there is a forgetful functor,   
$\cc{C}_{/{\{B_i\}_{i \in I}}} \mr{\Sigma} \cc{C}$\,, 
\mbox{$\Sigma({\{X \mr{x_i} B_i\}_{ i \in [n]}} \,=\, X$, $\Sigma(f) = f$,}
which is faithful and reflects isomorphisms. We will refer to this functor  as \emph{the functor $\Sigma$}. \cqd
\end{remark}

%\end{document}
The following holds by definition of products:
 \begin{remark} \label{propSigma}
Let ${\{B \mr{\pi_i} B_i\}_{ i \in [n]}}$ be a product diagram in $\cc{C}$, then the functor:
$$
\cc{C}_{/B}  \mr{\Phi}  \cc{C}_{/\Bi}: \hspace{6ex} 
\Phi(X \mr{x} B)  \;=\;  {\{X \xr{\pi_i  x} B_i\}_{ i \in [n]}}
$$
and $\Phi(f) = f$ on arrows, establishes an isomorphism of categories. Furthermore the family of its projections is a terminal object in the multislice category.
\end{remark}
The reason we consider multislice categories is not that we want to work with categories lacking products, but lacking a distinguished choice of products.

\emph{For categories with finite products we can use this Remark to transport to multislice categories the properties of slice categories.}

Though properties needed here may be proven directly for multislice categories, to simplify the notation we will do the proofs for slice categories and then use this remark.



%\end{document}
%\begin{remark}\label{props de F}
%The domain functor $\cc{C}_{/{\{B_i\}_{i \in I}}} \mr{\Sigma} \cc{C}$ is faithful, conservative and preserves pullbacks. It follows that $\Sigma$ reflects monomorphisms and pullbacks (\ref{con pullback reflejo monos}).
%\end{remark}

\begin{remark} \label{adjuntoderecha}
Let $X \in \cc{C}$, let  
$B \ml{\pi_2} E \mr{\pi_1}  X$ be a product diagram in $\cc{C}$, and let 
$Y \mr{y} B \, \in \, \cc{C}_{/B}$.  Define $\Pi(X) = (E \xr{\pi_2} B)$, then we have:

$$
\xymatrix@C=4ex@R=2ex
     {
      Y \ar@/^0.8pc/[drr]^{g}
        \ar@/_0.8pc/[ddr]^{y} 
        \ar[dr]^f
    \\
    & E \ar[d]^{\pi_2}
        \ar[r]^{\pi_1}
    & X
   \\
    & B
     }
\hspace{8ex} 
\xymatrix@C=5ex@R=0ex
     {
      & (Y \xr{y} B) \ar[r]^f  &  \Pi(X) %(E \xr{\pi_2} B)
     \\
     {} \ar@{-}[rrr] &&& {}
     \\
      & \Sigma(Y \xr{y} B)  \ar[r]^g & X
     }
$$    
By definition of product the diagram on the left establishes a bijective correspondence between the arrows $f$ and $g$, which is the same that the  correspondence  indicated in the right diagram. This shows that the object $\Pi(X) \in \cc{C}_{/B}$ is the value at $X$ of a right adjoint to the functor $\Sigma$, the defining universal property shows it is defined in arrows in a way that preserves composition, compare with Definition \ref{pbk1}.
 
It follows that \emph{when the product $E$ exists the functor $\Sigma$ will preserve any colimit that may exists in $\cc{C}_{/B}$}.


 \cqd
\end{remark}

The following is immediate and very easy to prove:
{
%\eazul
\begin{proposition}\label{p-pulback}
The category  $\cc{C}_{/\Bi}$ inherits any pull-back that exists in $\cc{C}$, moreover the functor $\Sigma$ preserves and reflects pull-backs. It follows it always preserves and reflects monomorphisms \cqd
\end{proposition} 
}

Since as observed in \ref{propSigma} for a category with finite products any multislice category has terminal objects, it follows:
\begin{proposition} \label{finitelycomplete}
If $\cc{C}$ is finitely complete, then $\cc{C}_{/\Bi}$ is finitely complete.
\cqd
\end{proposition}

%%%%%%%%%%%%%%%%%%%%%%%%%%%%%%%%%%%%%%%%%%%%%%%%%%%%%%%%%%%%%%%%%%%%
%%%%%%%%%%%%%%%%%%%%%%%% COMENTARIO %%%%%%%%%%%%%%%%%%%%%%%%%%%%%%%%
\begin{comment}
\begin{proposition} \label{reflexepi}
The functor $\Sigma$ reflex and preserves strict epimorphism.
\end{proposition}
\begin{proof}
Let $(X \mr{x} B) \mr{f} (Y \mr{y} B)$ be an arrow in 
$\cc{C}_{/B}$. We will argue below referring to the following diagram:
$$
\xymatrix
    {
     C \ar@<.5ex>[r]^r 
       \ar@<-.5ex>[r]_s 
       \ar@/_1.3pc/[ddrr]^{c} 
   & X \ar[rr]^f 
       \ar[dr]^g 
       \ar@/_/[ddr]^{x} 
  && Y \ar@/^/[ddl]^{y} 
       \ar[dl]_{h}
\\ 
  && Z \ar[d]^{z} 
  & 
\\
  && B 	
  &		  
     }
$$
\emph{reflex}: Assume $f$ is a strict epimorphism in $\cc{C}$. Let 
$C \mrpair{r}{s} X$ any two arrows in $\cc{C}$, and
$(X \mr{x} B) \mr{g} (Z \mr{z} B)$ a $f$-compatible arrow in $\cc{C}_{/B}$. Composing with $x$ we have $C \mrpair{xr}{xs} B$, so that $r,\;s$ determine arrows in  
$\cc{C}_{/B}$. This shows that $g$ is also \mbox{$f$-compatible} in 
$\cc{C}$, so we have a unique arrow $h$ such that $hf = g$ in $\cc{C}$. Since $f$ is also an epimorphism, it follows that $zh = y$, thus $h$ is an arrow in $\cc{C}_{/B}$.

\vspace{1ex}

\noindent \emph{preserves}: Assume $f$ is a strict epimorphism in 
$\cc{C}_{/B}$.
% Let $(C \mr{c} B) \mrpair{r}{s}  (X \mr{x} B)$ be any two arrows in $\cc{C}_{/B}$, and 
Let $C \mrpair{r}{s} X$ be any two arrows in $\cc{C}$, as before they determine two arrows in $\cc{C}_{/B}$, and let 
$(X \mr{x} B) \mr{g} (Z \mr{z} B)$ be a $f$-compatible arrow  in 
$\cc{C}_{/B}$. It follows there is a unique $h$, $hf = g$ in $\cc{C}_{/B}$, and since the functor  
$\Sigma$ is faithful, $h$ is also uniqe in $\cc{C}$. 
\end{proof}
\end{comment}
%%%%%%%%%%%%%%%%%%%%%%%%%%%%% FIN COMENTARIO %%%%%%%%%%%%%%%%%%%%%%%%%%
%%%%%%%%%%%%%%%%%%%%%%%%%%%%%%%%%%%%%%%%%%%%%%%%%%%%%%%%%%%%%%%%%%%%%%%

 From Remaks \ref{kernelpair}, \ref{adjuntoderecha}, and Proposition 
 \ref{p-pulback} if follows:
 \begin{proposition}\label{preserves}
 If $\cc{C}$ has pullbacks and binary products, then the functor $\Sigma$ preserves strict epimorphisms. 
 \end{proposition}
 
 \begin{proposition} \label{reflects}
The functor $\Sigma$ always reflects strict epimorphisms.
\end{proposition}
\begin{proof}
Let $X \mr{f} Y$ be a strict epimorphism in $\cc{C}$. We want to prove that any 
$$
(X \mr{x} B) \mr{f} (Y \mr{y} B)
\hspace{10ex} 
\xymatrix 
     {
      X \ar[rr]^{f} 
        \ar@/_5pt/[rd]^{x}
   && Y \ar@/^5pt/[ld]_{y}
  \\ 
    & B
     }
$$    
is a strict epimorphism in $\cc{C}_B$. Let $g$ be a $f$-compatible arrow in 
$\cc{C}_B$
$$
(X \mr{x} B) \mr{g} (Z \mr{y} B)
\hspace{10ex}
\newdir{(>}{{}*!/-6pt/\dir{>}}
\xymatrix
    {
     X \ar[rr]^f 
       \ar@{->>}[dr]_g 
       \ar@/_/[ddr]_{x} 
  && Y \ar@/^/[ddl]^{y} 
   \\  
   & Z %\ar@{}[u]|(.6){\equiv} 
       \ar[d]^(.43){z} 
   &
    \\
   & B    
   &	          }
$$ 
 We want to see that $g$ is $f$-compatible in $\cc{C}$. Let $C \mrpair{r}{s} X$ in $\cc{C}$ be such that $fr = fs$. Since $xs = yfs = yfr = xr, \; say \; = c$,  composing with 
$X \mr{x} B$ yields 
$$\xymatrix 
     {
      C \ar@<.5ex>[rr]^{r}
        \ar@<-.5ex>[rr]_{s}
        \ar@/_5pt/[rd]^{c}
   && X \ar@/^5pt/[ld]_{x}
  \\ 
    & B
     }
$$ 
Thus $r$ and $s$ determine arrows in $\cc{C}_{/B}$ such that $fr = fs$, and since $g$ is $f$-compatible in $\cc{C}_{/B}$ it holds $gr = gs$. This shows that $g$ is $f$-compatible in $\cc{C}$. It follows then that there exists a unique $Y \mr{h} Z$ such that $hf = g$ in $\cc{C}$. We have
$$
\newdir{(>}{{}*!/-6pt/\dir{>}}
\xymatrix
    {
     X \ar[rr]^f 
       \ar@{->>}[dr]^g 
       \ar@/_/[ddr]_{x} 
  && Y \ar@/^/[ddl]^{y} 
       \ar[dl]_{h}
   \\  
   & Z %\ar@{}[u]|(.6){\equiv} 
       \ar[d]^(.43){z} 
   &
    \\
   & B    
   &
   }
$$
It remains to see that $h$ is an arrow in $\cc{C}_{/B}$, that is, 
$zh = y$. But $zhf = zg =x =yf$, and since $f$ is in particular an epimorphism, this shows what we want.
\end{proof}



\begin{proposition}\label{slice have factorization}
For any arrow  
\mbox{$(X \mr{x} B) \mr{f} (Y \mr{y} B)$} in $\cc{C}_{/B}$, a strict factorization of $f$ in $\cc{C}$ determines a strict factorization of $f$ in 
$\cc{C}_{/B}$. Thus if every arrow in $\cc{C}$ admits a strict factorization, so does every arrow in $\cc{C}_{/\Bi}$.
\end{proposition}
\begin{proof} 
Let $(X \mr{x} B) \mr{f} (Y \mr{y} B)$ be an arrow in 
$\cc{C}_{/B}$, and 
take a strict factorization of $f$ in $\cc{C}$:
%
$$
\newdir{(>}{{}*!/-6pt/\dir{>}}
\xymatrix
    {
     X \ar[rr]^f \ar@{->>}[dr]_e 
       \ar@/_/[ddr]_{x} && Y \ar@/^/[ddl]^{y} 
   \\  
   & Z \ar@{(>->}[ur]_m 
       \ar@{}[u]|(.6){\equiv} 
       \ar@{-->}[d]^(.43){z} 
   &
    \\
   & B    
   &	          }
$$ 
% 
Since $x$ is $e$-compatible, there is a unique arrow 
$Z \mr{z} B$ such that $ze = x$. Since $e$ is epic it follows that $ym = z$. By \ref{reflects} $e$ is an strict epimorphism in  $\cc{C}_{/B}$, and by \ref{p-pulback} $m$ is a monomorphism in  $\cc{C}_{/B}$. 
\end{proof}

%\begin{observation}
%If every morphism in $\cc{C}$ has a distinguished strict factorization, then so does every morphism in $\cc{C}_{/\Bi}$ and $\Sigma$ preserves them.
%\end{observation}

%{\eazul
\begin{proposition} \label{universal}
Strict epimorphisms are universal in $\cc{C}_{/\Bi}$.
\end{proposition}
\begin{proof}
It follows from the fact that the functor $\Sigma$ preserves and reflects pullbacks and strict epimorphism, \ref{p-pulback},   
\ref{preserves}, \ref{reflects}.       
\end{proof}

\vspace{1ex}

Propositions \ref{finitelycomplete},  \ref{slice have factorization} and
 \ref{universal} put together show:

\begin{proposition} \label{sliceregular}
If $\cc{C}$ is regular, then $\cc{C}_{/\Bi}$ is regular. \cqd
\end{proposition}
%}
%%%%%%%%%%%%%%%%%%%%%%%%%%%%%%%%%%%%%%%%%%%%%%%%%%%%%%%%%%%%%%%%%%%%
%%%%%%%%%%%%%%%%%%%%%%%% COMENTARIO %%%%%%%%%%%%%%%%%%%%%%%%%%%%%%%%
\begin{comment}
In this section we will establish some generalities on the collective behaviour that a family of functors may have. We will prove that for a regular category $\cc{C}$,  a \textit{monic-conservative} family of regular functors is faithful and conservative.

 Let $\mathcal{F}$ be a family of functors with common domain $\cc{C}$.

\begin{definition}
We will say that pullbacks (pushouts,...) are \textbf{preserved} by $\cc{F}$ if for every $F \in \cc{F}$ the functor $F$ preserves pullbacks (pushouts,...).
\end{definition}

\begin{definition}
We will say that monomorphisms (epimorphisms,...) are \textbf{reflected} by $\cc{F}$ if for every arrow ${X\mr{u}}Y \in \mathcal{C}$ such that for every $F \in \cc{F}$  its image $Fu$ is a monomorphism (epimorphism,...), it follows that $u$ is a monomorphism (epimorphism,...).
\end{definition}


\begin{definition}
$\cc{F}$ is \textbf{conservative} if $\cc{F}$ reflects isomorphisms.
\end{definition}

\begin{definition}
$\cc{F}$ is \textbf{monic} (\textbf{epic})\textbf{-conservative} if for every monic (epic) ${{X \mr{u} Y}  \in \mathcal{C}}$ such that for every $F \in \cc{F}$ its image $Fu$ is an isomorphism, it follows that $u$ is an isomorphism.  
 \end{definition}


\begin{observation}
A conservative family is monic (epic)-conservative. 
\end{observation}

\begin{definition}
$\cc{F}$ is \textbf{faithful} if for every pair ${X\mrpair{u}{v}Y} \in \cc{C}$ such that for every $ F\in \cc{F}$ their images are  equal ($Fu=Fv$), it follows that $u=v$.
 \end{definition}


\begin{lemma}\label{si fiel, refleja monos}
If $\cc{F}$ is faithful, then $\cc{F}$ reflects monics(epics). 
\end{lemma}

\begin{proof}
Let $X \mr{u} Y$ in $\cc{C}$ be such that for every $F\in\cc{F}$, $Fu$ is monic. Suppose we have $A \mrpair{x}{y} X$ such that $ux=uy $. Then for every $F\in\cc{F}$, $Fu \cdot Fx=Fu \cdot Fy$ and since $Fu$ is monic, it follows that for every $F\in\cc{F}$, $Fx=Fy$. Thus $x=y$. The dual proposition follows.
\end{proof}

\begin{lemma}\label{con pullback reflejo monos}
If $\cc{C}$ has pullbacks (pushouts), $\cc{F}$ preserves them and $\cc{F}$ is monic (epic)-conservative, then $\cc{F}$ reflects monics (epics). 
\end{lemma}

\begin{proof}
Let $X \mr{u} Y$ be such that for every $F \in \cc{F}$, $Fu$ is monic. We will prove that the following diagram is a pullback
\[
\xymatrix{X \ar[r]^{id_X} \ar[d]_{id_X} & X \ar[d]^u \\ X \ar[r]_u & Y}
\]

Take a pullback in $\cc{C}$ and $\delta$ as follows:

%
%$$  
%\xymatrix @+3.5ex {X \ar@/_1.5pc/[ddr]_{id_X} \ar@/^/[drr]^{id_X} \ar@{-->}[dr]_{\exists! \delta} & \\ & P \ar@{}[dr]|{p.b.} \ar@<.1ex>@{}[u]|(.36){\equiv} \ar@<.1ex>@{}[l]|(.5){\equiv} \ar[r]^{p_1} \ar[d]_{p_2} & X \ar[d]^u \\ & X \ar[r]_u & Y} 
%$$
% 
$$ 
\xymatrix @+3.5ex
     { 
      X \ar@/_1.5pc/[ddr]_{id_X} 
        \ar@/^/[drr]^{id_X} 
        \ar@{-->}[dr]_{\exists! \delta} 
     & {}
   \\ 
     & P \ar@{}[dr]|{p.b.} 
         \ar@<.1ex>@{}[u]|(.36){\equiv} 
         \ar@<.1ex>@{}[l]|(.5){\equiv} 
         \ar[r]^{p_1} 
         \ar[d]_{p_2} 
      & X \ar[d]^u 
     \\ 
     & X \ar[r]_u 
     & Y
     } 
$$
 
From this diagram we see $\delta$ is monic and for every $F \in \cc{F}$, $F\delta$ is the isomorphism between the corresponding pullback diagrams. Therefore $\delta$ is an isomorphism. The dual proposition follows.
\end{proof}

\begin{remark}
The proof of Lemma \ref{con pullback reflejo monos} gives us a technique to prove that under the additional hypothesis of preserving limits (colimits) of a certain type, conservative families reflect limits (colimits) of that type.
\end{remark}

\begin{proposition}\label{pb + mc implica c}
If $\cc{C}$ has pullbacks (pushouts), $\cc{F}$ preserves them and $\cc{F}$ is monic (epic)-conservative, then $\cc{F}$ is conservative. 
\end{proposition}

\begin{proof}
Observe that monic (epic)-conservative families that reflect monics (epics) are conservative.
\end{proof}

\begin{proposition}\label{cmc implica cf}
If $\cc{C}$ has equalizers (coequalizers), $\cc{F}$ preserves them and $\cc{F}$ is monic/strict-monic (epic/strict-epic)-conservative, then $\cc{F}$ is faithful.
\end{proposition}

\begin{proof}
Take $X\mrpair{u}{v}Y$ in $\cc{C}$ such that for every $F \in \cc{F}$ their images are equal ($Fu=Fv$). An equalizer $E\mr{e}X$  of $u$ and $v$  is monic and its image $Fe$ is an equalizer of $Fu$ and $Fv$ which are equal, so $Fe$ is an isomorphism. Thus $e$ is an isomorphism and consequently $u=v$. For the strict case we need only note that equalizers are strict monics. The dual propositions follow.
\end{proof}

\begin{proposition}
If $\cc{C}$ has equalizers (coequalizers), $\cc{F}$ preserves them and $\cc{F}$ is monic (epic)-conservative, then $\cc{F}$ is conservative.
\end{proposition}

\begin{proof}
It follows from Lemma \ref{si fiel, refleja monos} and the observation made in Proposition \ref{pb + mc implica c}.
\end{proof}


\begin{corollary}
If $\cc{C}$ has equalizers, $\cc{F}$ preserves them and $\cc{F}$ is monic (epic)-conservative, then $\cc{F}$ is conservative and faithfull.
\end{corollary}

\begin{remark}\label{monic conservative implica conservative}
If $\cc{C}$ is regular and $\cc{F}$  the set of all regular functors $\cc{C} \mr{} \cc{E}ns$ is monic conservative, it follows that it is conservative and faithful.
\end{remark}


\begin{remark}
Even under the strictest limit-preserving conditions we will not be able to guarantee that a faithful family is conservative in any sense. Take the following counterexample: Let $\cc{C}=\{0 \mr{u} 1\}$ and take the family whose only member is the functor $\cc{C} \mr{F} \{*\}$. $\cc{C}$ is a regular category that in fact  has all limits and colimits, $F$ is regular and preserves all limits and colimits , $F$ is faithful but nevertheless does not reflect the isomorphism $Fu$.
\end{remark}
 
\begin{proposition}
If in $\cc{C}$ every bimorphism (a morphism that is both epic and monic) is an isomorphism and $\cc{F}$ is faithful, then $\cc{F}$ is conservative.
\end{proposition}
\end{comment}
%%%%%%%%%%%%%%%%%%%%%%%%%%%%% FIN COMENTARIO %%%%%%%%%%%%%%%%%%%%%%%%%%
%%%%%%%%%%%%%%%%%%%%%%%%%%%%%%%%%%%%%%%%%%%%%%%%%%%%%%%%%%%%%%%%%%%%%%%

%\end{document}

\subsection{Families of Functors With Common Domain} \label{families}


In this section we will establish a result, Proposition 
\ref{pb + mc implica c} below, which is essential (although elementary) for the completeness theorem in this paper.

\vspace{1ex}

%Consider a family $\cc{F}$ of functors with comun domain a category 
%$\cc{X}$.
%\begin{proposition} \label{monic-cons ==> cons}
%For a category $\cc{X}$ with pull-backs, any family $\cc{F}$ of pull-back preserving functors with comun domain $\cc{X}$  which is 
%\emph{monic-conservative} is \emph{conservative}. 
%\end{proposition}

Everything in this section is based and contained in 
\cite{sga4}[Ex I, \S 6], for the convenience of the reader we recall the terminology and extract only the parts that we need.

%We will prove that for a regular category $\cc{C}$,  a \textit{monic-conservative} family of regular functors is faithful and conservative.

 Let $\mathcal{F}$ be a family of functors with common domain $\cc{C}$.

\begin{definitionst} ${}$
\begin{enumerate} 
\item $\cc{F}$ \emph{preserves} pullbacks, equalizers, if for every $F \in \cc{F}$ the functor $F$ preserves pullbacks, equalizers, respectively.
\item $\cc{F}$ \emph{reflects} monomorphisms if for every arrow 
${X\mr{u}}Y$ such that for every $F \in \cc{F}$  its image $Fu$ is a monomorphism, it follows that $u$ is a monomorphism.
\item $\cc{F}$ is \emph{conservative} if $\cc{F}$ reflects isomorphisms.
\item $\cc{F}$ is \emph{monic-conservative} if for every monomorphism ${X \mr{u} Y}$ such that for every $F \in \cc{F}$ its image $Fu$ is an isomorphism, it follows that $u$ is an isomorphism.
\item 
$\cc{F}$ is \emph{faithful} if for every pair ${X\mrpair{u}{v}Y}$ such that for every $ F\in \cc{F}$ their images are  equal, $Fu=Fv$, it follows that $u=v$.
\end{enumerate}
\end{definitionst}

%\begin{observation}
%A conservative family is monic-conservative. 
%\end{observation}

\begin{proposition} \label{reflects + m-cons ==> cons}
If $\cc{F}$ reflects monomorphisms and is monic-conservative, then it is conservative.
\end{proposition}
\begin{proof}
Let $u$ be such that for every $F\in\cc{F}$, $Fu$ is an isomorphism. Then in particular $u$ is a monomorphism, and so by the first assumption it is a monomorphism, and in turn by the second assumption it is a isomorphism.  
\end{proof}

We now put a condition that assures us that F reflects monomorphisms.

\begin{proposition}\label{con pullback reflejo monos}
If $\cc{C}$ has pullbacks, $\cc{F}$ preserves them and $\cc{F}$ is monic-conservative, then $\cc{F}$ reflects monomorphism. 
\end{proposition}

\begin{proof}
Let $X \mr{u} Y$ be such that for every $F \in \cc{F}$, $Fu$ is monic. We will prove that the following diagram is a pullback
\[
\xymatrix{X \ar[r]^{id_X} \ar[d]_{id_X} & X \ar[d]^u \\ X \ar[r]_u & Y}
\]

Take a pullback in $\cc{C}$ and $\delta$ as follows:

%
%$$  
%\xymatrix @+3.5ex {X \ar@/_1.5pc/[ddr]_{id_X} \ar@/^/[drr]^{id_X} \ar@{-->}[dr]_{\exists! \delta} & \\ & P \ar@{}[dr]|{p.b.} \ar@<.1ex>@{}[u]|(.36){\equiv} \ar@<.1ex>@{}[l]|(.5){\equiv} \ar[r]^{p_1} \ar[d]_{p_2} & X \ar[d]^u \\ & X \ar[r]_u & Y} 
%$$
% 
$$ 
\xymatrix @+3.5ex
     { 
      X \ar@/_1.5pc/[ddr]_{id_X} 
        \ar@/^/[drr]^{id_X} 
        \ar@{-->}[dr]_{\exists! \delta} 
     & {}
   \\ 
     & P \ar@{}[dr]|{p.b.} 
         \ar@<.1ex>@{}[u]|(.36){\equiv} 
         \ar@<.1ex>@{}[l]|(.5){\equiv} 
         \ar[r]^{p_1} 
         \ar[d]_{p_2} 
      & X \ar[d]^u 
     \\ 
     & X \ar[r]_u 
     & Y
     } 
$$
 
From this diagram we see $\delta$ is a monomorphism (since either projection is a section), and for every $F \in \cc{F}$, $F\delta$ is the isomorphism between the corresponding pullback diagrams. Therefore $\delta$ is an isomorphism. \end{proof}

From \ref{reflects + m-cons ==> cons} and \ref{con pullback reflejo monos} it follows:
\begin{corollary}\label{pb + mc implica c}
If $\cc{C}$ has pullbacks, $\cc{F}$ preserves them and $\cc{F}$ is 
monic-conservative, then $\cc{F}$ is conservative. 
\end{corollary}



%\begin{remark}\label{monic conservative implica conservative}
%If $\cc{C}$ is regular and $\cc{F}$  the set of all regular functors $%\cc{C} \mr{} \cc{E}ns$ is monic conservative, it follows that it is conservative and faithful.
%\end{remark}


%%%%%%%%%%%%%%%%%%%%%%%%%%%%%%%%%%%%%%%%%%%%%%%%%%%%%%%%%%%%%%%%%%%%%%
%%%%%%%%%%%%%%%%%%%%%%% COMENTARIO %%%%%%%%%%%%%%%%%%%%%%%%%%%%%%%%%%%

\begin{comment}
\begin{lemma}\label{si fiel, refleja monos}
If $\cc{F}$ is faithful, then $\cc{F}$ reflects monics(epics). 
\end{lemma}

\begin{proof}
Let $X \mr{u} Y$ in $\cc{C}$ be such that for every $F\in\cc{F}$, $Fu$ is monic. Suppose we have $A \mrpair{x}{y} X$ such that $ux=uy $. Then for every $F\in\cc{F}$, $Fu \cdot Fx=Fu \cdot Fy$ and since $Fu$ is monic, it follows that for every $F\in\cc{F}$, $Fx=Fy$. Thus $x=y$. The dual proposition follows.
\end{proof}


\begin{remark}
The proof of Lemma \ref{con pullback reflejo monos} gives us a technique to prove that under the additional hypothesis of preserving limits (colimits) of a certain type, conservative families reflect limits (colimits) of that type.
\end{remark}

\begin{proposition}\label{cmc implica cf}
If $\cc{C}$ has equalizers (coequalizers), $\cc{F}$ preserves them and $\cc{F}$ is monic/strict-monic (epic/strict-epic)-conservative, then $\cc{F}$ is faithful.
\end{proposition}

\begin{proof}
Take $X\mrpair{u}{v}Y$ in $\cc{C}$ such that for every $F \in \cc{F}$ their images are equal ($Fu=Fv$). An equalizer $E\mr{e}X$  of $u$ and $v$  is monic and its image $Fe$ is an equalizer of $Fu$ and $Fv$ which are equal, so $Fe$ is an isomorphism. Thus $e$ is an isomorphism and consequently $u=v$. For the strict case we need only note that equalizers are strict monics. The dual propositions follow.
\end{proof}

\begin{proposition}
If $\cc{C}$ has equalizers (coequalizers), $\cc{F}$ preserves them and $\cc{F}$ is monic (epic)-conservative, then $\cc{F}$ is conservative.
\end{proposition}

\begin{proof}
It follows from Lemma \ref{si fiel, refleja monos} and the observation made in Proposition \ref{pb + mc implica c}.
\end{proof}


\begin{corollary}
If $\cc{C}$ has equalizers, $\cc{F}$ preserves them and $\cc{F}$ is monic (epic)-conservative, then $\cc{F}$ is conservative and faithfull.
\end{corollary}



\begin{remark}
Even under the strictest limit-preserving conditions we will not be able to guarantee that a faithful family is conservative in any sense. Take the following counterexample: Let $\cc{C}=\{0 \mr{u} 1\}$ and take the family whose only member is the functor $\cc{C} \mr{F} \{*\}$. $\cc{C}$ is a regular category that in fact  has all limits and colimits, $F$ is regular and preserves all limits and colimits , $F$ is faithful but nevertheless does not reflect the isomorphism $Fu$.
\end{remark}
 
\begin{proposition}
If in $\cc{C}$ every bimorphism (a morphism that is both epic and monic) is an isomorphism and $\cc{F}$ is faithful, then $\cc{F}$ is conservative.
\end{proposition}

\end{comment}

%%%%%%%%%%%%%%%%%%%%%%%% FIN COMENTARIO %%%%%%%%%%%%%%%%%%%%%%%%%%%%%%%%%%
%%%%%%%%%%%%%%%%%%%%%%%%%%%%%%%%%%%%%%%%%%%%%%%%%%%%%%%%%%%%%%%%%%%%%%%%%%


\subsection{Weakly Projective Objects}
Here we give a characterization of which hom-functors of a regular category $\cc{A}$ are regular functors.

\begin{definition}
An object $A$ in a category $\cc{A}$ is \emph{weakly projective} if the functor $\cc{A} \xr{hom_{\cc{A}}(A,-)} \cc{E}ns$ 
preserves strict epimorphisms. That is, for any strict epimorphism 
$\xymatrix{X \ar@{->>}[r]^{u} &  Y}$, any $A \mr{t} Y$ has a 
\emph{lift} $A \mr{v} X$, as indicated in the diagram 
$$
\xymatrix
     {
      X \ar@{->>}[rr]^{u} 
   && Y 
   \\ 
   && A \ar[u]_{\forall t}
        \ar@/^/@{.>}[ull]^{\exists v} 
        \ar@{}[ul]|(.6){\equiv}}
$$
\end{definition}


\begin{remark}
In a regular category a object $A$ is weakly projective if and only if every strict epimorphism $\xymatrix{X \ar@{->>}[r]^{u} & A}$ admits a section.
\end{remark}
\begin{proof}
If $A$ is weakly projective, a lift of $id_A$  yields a section. For the converse let $\xymatrix{X \ar@{->>}[r]^{u} & Y}$ be a strict epimorphism. For a given $A \mr{t} Y$ take a pullback of $u$ along $t$.
$$
\xymatrix{X \ar@{->>}[r]^{u} \ar@{}[rd]|{p.b.} & Y \\ P \ar@{.>}[u]^{p_1} \ar@{.>}[r]_{p_2} & A \ar[u]_t}
$$
Since $p_2$ is a strict epimorphism it admits a section $v$. The composite 
$p_1 v$ \mbox{is a lift of $t$.}
\end{proof} 





\section{PREFIBRED CATEGORIES} \label{fibrations}
\subsection{Basic Notions}

%In this subsection we recall  some notions of fibered categories and we set notation and terminology. 
%as we will use in this paper.

In this section we recall the context of fibered categories introduced in \cite{sga1}.

Consider a functor $\cc{E} \mr{F} \cc{G}$: We say that an object $X$ in $\cc{E}$ \emph{sits} over an object $\alpha$ in $\cc{G}$ if $F(X) = \alpha$. An arrow $X \mr{f} Y$   \emph{sits} over an arrow    
$\alpha \mr{\varphi} \beta$ if 
$F(X \mr{f} Y) = \alpha \mr{\varphi} \beta$. Given $\alpha$ in $\cc{G}$, the \emph{fiber} $\cc{E}_\alpha$ is the subcategory of $\cc{E}$ of objects $X$ over $\alpha$ and arrows $f$ over the identity $id_\alpha$ (we will say over 
$\alpha$). We denote  
$hom_{\varphi}(X,Y) \subset hom(X,Y)$ the set of arrows sitting over $\varphi$.
We will refer to arrows in a fibre as \emph{vertical arrows}.

We display this situation in a double diagram:

$$
\xymatrix
      {
       Z \ar[d]_h 
       & 
       \\
	   X \ar[r]_f 
	   & Y  
	   \\
		\alpha \ar[r]_\varphi & \beta
	  }
$$     
\begin{sinnadastandard}[{\bf Basic Facts and Definitions}] \label{basicdefinitions}

\begin{enumerate} $ $

\item Recall that an arrow $X \mr{f} Y$ is:

\emph{cartesian}, if for every 
arrow $Z \mr{h} Y$ over $\alpha \mr{\varphi} \beta$ there exists a unique $Z \mr{g} X$ over $\alpha$ such that $fg = h$.
That is, postcomposing with $f$,  
${hom_{\alpha}(Z, X) \mr{f_*} hom_{\varphi}(Z,Y)}$, is a bijection.

\vspace{1ex}

\emph{strongly cartesian}, if for every 
$\gamma \mr{\psi} \alpha$, $Z$ over $\gamma$, and 
$Z \mr{h} Y$ over the composite $\varphi \psi$, there exists a unique $g$ over $\psi$ such that $fg = h$. 

   That is, postcomposing with $f$,  
${hom_{\psi}(Z, X) \mr{f_*} hom_{\varphi \psi}(Z,Y)}$ is a bijection.

%\vspace{1ex}

We see these definitions in  the following diagrams:   
$$
\xymatrix
    {
     Z \ar@/^/[dr]^{\forall} 
       \ar@{-->}[d]_{\exists!} 
   & {}
   \\
     X \ar@{}[ur]|(.4){\equiv} 
       \ar[r]_f  
   & Y  
   \\
     \alpha \ar[r]_\varphi 
   & \beta}
\hspace{8ex}
\xymatrix
    { 
     Z \ar@/^/[rrd]^\forall 
       \ar@{-->}[rd]_{\exists !}
   && {}
   \\
    & X \ar@{}[u]|(.33){\equiv} 
        \ar[r]_f  
    & Y  
    \\
	  \gamma \ar[r]_\psi 
	& \alpha \ar[r]_\varphi 
	& \beta}
$$

\item \label{cartesianiffiso}
Identity arrows are cartesian, and a vertical arrow is cartesian if and only if it is an isomorphism

\item  \label{cartesiancomposing} 
Given a composite $X \mr{f} Y \mr{g} Z$, $h = g\, f$, then:

If $f$ and $g$ are strongly cartesian, so it is $h$.

If $h$ and $g$ are strongly cartesian, so it is $f$. 

\item In general cartesian arrows do not compose, but they do if and only if they are strongly cartesian.

\item Recall that a functor $\cc{E} \mr{F} \cc{G}$ is:

\emph{a prefibration} 
 if for every ${\alpha \mr{\varphi} \beta \in \cc{G}}$ and for every 
 ${Y \in \cc{E}_\beta}$ there exists a cartesian morphism over $\varphi$ with target $Y$.
 
\emph{a fibration} if it is a prefibration and the set of cartesian morphisms is closed under composition, equivalently, if cartesian morphisms are strongly cartesian.

 \item In a fibration the concepts of 
cartesian and strongly cartesian coincide.

\item  We write $Y^* \mr{} Y$ to indicate that we are labeling a cartesian morphism over $\varphi$. We omit the label $\varphi$ in the usual notation  $\varphi^* Y$ to remind us we are making a choice of a single cartesian arrow and that we do not assume to have a clivage.

\item \emph{We have the dual definitions of cocartesian morphism, precofibration and cofibration. We will freely use these notions and the dual theorems.}

\end{enumerate}
\end{sinnadastandard}

\begin{proposition} \label{fib<=>cof}
For $F$ prefibred and precofibred, $F$ is a fibration if and only if it is a cofibration.
\end{proposition}
\begin{proof}
We prove one side of the duality: \emph{if $F$ is a cofibration, then it is a fibration.} In fact, cartesian morphisms are strong cartesian. We see this as follows:
Let $Y^* \mr{f} Y$ be a cartesian morphisms over ${\alpha \mr{\varphi} \beta}$. Given ${\gamma \mr{\psi} \alpha \mr{\varphi} \beta \in \cc{G}}$ and $Z \in \cc{E}_\gamma$, take $Z \mr{g} Z_*$ cocartesian over $\psi$, and consider the following diagram: 
$$
\xymatrix{{hom_{\psi}(Z,Y^*)} \ar[r]^{f_*} \ar@{}[dr]|\equiv & {hom_{\varphi \psi}(Z,Y)} 
\\
		  {hom_\alpha(Z_*,Y^*)} \ar[r]_{f_*} \ar[u]^{g^*} & {hom_{\varphi}(Z_*,Y)} \ar[u]_{g^*}	}	  
$$
The bottom arrow is bijective because $f$ is cartesian and the vertical arrows are bijections because $g$ is strong cocartesian. It follows the top arrow is a bijection.
\end{proof}
\begin{proposition}\label{adjoints}
For  $\cc{E} \mr{F} \cc{G}$ prefibered and precofibred, given $\beta \mr{\varphi} \alpha$ in $\cc{G}$,  
$X \in \cc{E}_\beta$ and  $Y \in \cc{E}_\alpha$, there is a natural bijection  $hom_\alpha(X,Y^*) \, \approx \, hom_\beta(X_*, Y)$.
\end{proposition}
\begin{proof}
Consider the following diagram:
$$
\xymatrix
    {
     X \ar[r]^{b_X} \ar[d]^{f} & X_* \ar[d]^{g} 
   \\
     Y^* \ar[r]^{a_Y} & Y
    }
$$ 
with $b_X$ cocartesian and $a_Y$ cartesian over $\varphi$. This establishes a natural bijection such that 
$a_Y \circ f = g \circ b_X$, see \cite{sga1}.VI.10.   
\end{proof} 

\vspace{2ex}

%%%%5555%%%%%%%%%%%%%%%%%%%%%%%%% COMMENT %%%%%%%%%%%%%%%%%%%%%%%%%%%%%
\begin{comment}
{
\erojo

9. A \emph{clivage} is 
 a set $K$ of cartesian morphisms that verifies that for  each ${\alpha \mr{\varphi} \beta\ \in \cc{G}}$ and $X \in \cc{E}_\beta$, there is a unique morphism $s \in K$ over $\varphi$ with \mbox{target $X$.}

\vspace{1ex}

10. Every cleaved functor is a prefibration. Using choice it follows that every prefibration admits a clivage. 
%We will not assume such a choice has been made.  

\vspace{1ex}

11. A cleaved prefibration $\cc{E} \mr{F} \cc{G}$ with clivage $K$ is \emph{split} if the morphisms in  $K$ are closed under composition.

\vspace{1ex}

12. Every functor that admits a split clivage is a fibration. 

\vspace{2ex}
}
\end{comment}
%%%%%%%%%%%%%%%%%%%%%%%%%%%%%% END %%%%%%%%%%%%%%%%%%%%%%%%%%%%%%%%%%%%%

The category of prefibrations over $\cc{G}$ is the subcategory of $\cc{C}at/\cc{G}$ whose objects are prefibrations and whose morphisms are $\cc{G}$-functors that transform cartesian morfisms into cartesian morphisms. We call these morphisms \textit{cartesian} $\cc{G}$-functors and we will denote this category $Prefib(\cc{G})$. The category of fibrations over $\cc{G}$ is the full subcategory of $Prefib(\cc{G})$ whose objects are fibrations. We denote this category $Fib(\cc{G})$.

%%%%%%%%%%%%%%%%%%%%%%%%%%%%%%%% COMMENT %%%%%%%%%%%%%%%%%%%%%%%%%%%%%%%
\begin{comment}
\rojo{ We define the category of cleaved (split) prefibrations as the category whose objects are pairs ${(\cc{E} \mr{F} \cc{G},K)}$ where $K$ is a clivage for $F$ and whose morphisms are cartesian $\cc{G}$-functors that  preserve the clivages.  We will denote this category $Cprefib(\cc{G})$ ($Sprefib(\cc{G})$). Similar notations will be used for the categories of cleaved and split fibrations.}

{\erojo
\begin{remark}
In a cleaved prefibration $(\cc{E} \mr{F} \cc{G},K)$ there is associated to ${\alpha \mr{\varphi} \beta \in \cc{G}}$ a functor ${\cc{E}_\beta \mr{\varphi^*} \cc{E}_\alpha}$ called the \textit{pullback functor} along $\varphi$ of the prefibration  determined by the diagram below where ${X \mr{m} Y \in \cc{E}_\beta}$ and ${s,t \in K }$.  


\begin{align*}
\xymatrix{\varphi^*X \ar[r]^{s} \ar@{-->}[d]_{\varphi^*(m)} & X \ar[d]^m \\
		  \varphi^*Y \ar[r]_{t} & Y \ar@{}[ul]|\equiv \\
		  \alpha \ar[r]^\varphi & \beta}
\end{align*}

\noindent   For a general prefibration we will use a simpler version of this diagram notation to indicate we are labelling a cartesian morphism over $\varphi$ with target $X$. 

\begin{align*}
\xymatrix{X^* \ar[r] & X \\
		  \alpha \ar[r]^\varphi & \beta }
\end{align*}


\noindent This variation in notation is done to remind us we are making a momentary choice of a single cartesian arrow and that we do not assume to have a clivage. We may also indicate this by saying  we have a cartesian morphism $X^* \mr{} X$  over $\varphi$.

\end{remark}
}
\rojo{FIX-001.01 START Eliminar Proposition ?}

{\erojo
\begin{proposition}\label{cartes son cartes fuertes en una }
If $F$ is a prefibred and cofibred, then $F$ is a fibration.
\end{proposition}
\begin{proof} {SGA1 proposition 10.1}
We will prove that cartesian morphisms are strong cartesian:
%(see Remark \ref{cartesianas cerradas por composicion %implican fibracion}).
%
 Take $Y^* \mr{f} Y$ a cartesian morphisms over ${\alpha \mr{\varphi} \beta}$. Take ${\gamma \mr{\psi} \alpha \mr{\varphi} \beta \in \cc{G}}$ and $X \in \cc{E}_\gamma$. We will prove that ${hom_{\psi}(X,Y^*) \mr{f_*} hom_{\varphi \psi}(X,Y)}$ is a bijection. Take $X \mr{r} X_*$ a strong cocartesian morphism over $\psi$. The situation can be described as follows.
\[
\xymatrix @+6ex {X  \ar [r] ^r  \ar @/^6pc/ [rrd] ^{\forall}  \ar @{-->} [dr] _(.5){\exists ! c}  &  X_*  \ar @{} [r] |(.36)\equiv  \ar @{} [dl] |(.3)\equiv  \ar @{-->} [dr] ^(.45){\exists ! a}  \ar @{-->} [d] ^(.5){\exists ! b}  &  \\
		             &  Y^*  \ar [r] _{f}  \ar @{} [ru] |(.3) {\equiv}  &  Y  \\
		  \gamma  \ar [r] ^{\psi}  &  \alpha  \ar [r] ^{\varphi}  &  \beta}
\]
\noindent The arrow $c$ shows that $f$ is strong cartesian.  
\vspace{2ex}
For an alternative proof take the following commutative diagram.
\[
\xymatrix{{hom_{\psi}(X,Y^*)} \ar[r]^{f_*} \ar@{}[dr]|\equiv & {hom_{\varphi \psi}(X,Y)} 
\\
		  {hom_\alpha(X_*,Y^*)} \ar[r]_{f_*} \ar[u]^{r^*} & {hom_{\varphi}(X_*,Y)} \ar[u]_{r^*}	}	  
\]
\noindent The bottom arrow is bijective because $f$ is cartesian and the vertical arrows are bijections because $r$ is strong cocartesian. The result follows. 
\end{proof}
}
\rojo{FIX-001.01 END }

{
\erojo 
\begin{definition}
 A subset $\cc{A} \subset Ob(\cc{E})$ is \textbf{stable} if for every $X \in \cc{A}$ and any cartesian morphism $X^* \mr{} X$ it follows that $X^* \in \cc{A}$.
\end{definition}

If $X^* \mr{} X$ is cartesian over $\alpha \mr{\varphi} \beta$ we think of $X$  being \textit{pulled back} to $X^*$ along $\varphi$. We will call $X^*$ a \textit{pullback} of $X$ along $\varphi$. This means that $\cc{A}$ is stable when its objects are pulled back to objects of $\cc{A}$ exclusively.  

\begin{remark}
Stable subsets are closed under isomorphisms.
\end{remark}

\begin{definition}
We will say that \textbf{terminal objects are stable}  if the set of terminal objects of the fibres is a stable set.
\end{definition}

\noindent That is to say  if $1_\beta$ is a terminal object of $\cc{E}_\beta$ and ${(1_\beta)^* \mr{} 1_\beta }$ is cartesian over $\alpha \mr{\varphi} \beta \in \cc{G}$, it follows that $(1_\beta)^*$ is terminal in $\cc{E}_\alpha$.

\rojo{considerar eliminar la nocion general de stable set en caso que solo se utilice para los objetos terminal}

\begin{nobservation}
Vertical arrows can  be  pulled back  as follows. For ${X \mr{m} Y \in \cc{E}_\beta}$ choose two cartesian morphisms ${X^* \mr{} X}$ and ${Y^* \mr{s} Y}$ over $\alpha \mr{\varphi} \beta \in \cc{G}$. These determine an arrow in ${X^* \mr{m^*} Y^* \in\cc{E}_\alpha}$ given by $m^*$ in the following diagram.


\[
\xymatrix{X^* \ar[r] \ar@{-->}[d]_{\exists ! m^*} & X \ar[d]^{m} \\
		  Y^* \ar[r]_s  & Y \ar@{}[ul]|\equiv  \\
		  \alpha \ar[r]^{\varphi} & \beta}
\]

\noindent We call $m^*$ a \textit{pullback} of $m$ along $\varphi$. In a fibration the diagram on top is in fact a pullback in $\cc{E}$ and thus $m^*$ is a pullback of $m$ along $s$.  

Arrows in $\cc{E}_\alpha$ that are isomorphisms (epimorphisms,...) \textit{in} $\cc{E}_\alpha$  will be refered to as \textit{vertical} isomorphisms (epimorphisms,...).
\end{nobservation}

\begin{definition}
We will say that \textbf{isomorphisms (epimorphisms,...) are stable}  if for every vertical isomorphism (epimorphism,...) $m$ it follows that every pullback $m^*$ of it is a vertical isomorphism (epimorphism,...). 
\end{definition}

\begin{nobservation}
Isomorphisms are stable in any prefibration or precofibration.    
\end{nobservation}
}
\end{comment}
%%%%%%%%%%%%%%%%%%%%%%%% END %%%%%%%%%%%%%%%%%%%%%%%%%%%%%%%%%%%%%%%%%%%

\subsection{Stability in a Prefibration}

The following concepts are inspired in defining properties of the pullback functors of a cleaved prefibration without using clivages.

\begin{sinnadastandard}
[{\bf Pulling back along an arrow of $\cc{G}$}]
\label{pbk1} $ $

Let $\cc{E} \mr{F} \cc{G}$ be a prefibration and 
$\alpha \mr{\varphi} \beta$ an arrow in $\cc{G}$.

\begin{enumerate} 

\item 
%
%
%\end{enumerate}
%\end{sinnadastandard}
%\end{document}
%
Given $X$ in $\cc{E}_\beta$ and ${X^* \tocar X }$ cartesian over 
$\alpha \mr{\varphi} \beta$, we say that $X^*$ in $\cc{E}_\alpha$ is a \emph{pull-back} of $X$ along $\varphi$.

 
\item \label{vpull} 
\emph{Vertical arrows are pulled back as follows:} For ${X \mr{f} Y \in \cc{E}_\beta}$, choose cartesian morphisms ${X^* \tocar X}$ and 
 ${Y^* \tocar Y}$ over $\alpha \mr{\varphi} \beta$. These determine a unique arrow ${X^* \mr{f^*} Y^* \in\cc{E}_\alpha}$ as in the following diagram.
$$
\xymatrix
   {
    X^* \ar[r]|\car
        \ar@{-->}[d]_{f^*} 
  & X \ar[d]^{f} 
  \\
    Y^* \ar[r]|\car 
  & Y \ar@{}[ul]|\equiv  
  \\
    \alpha \ar[r]^{\varphi} 
  & \beta
   }
$$
We say that $f^*$ is a \emph{pull-back} of $f$ along $\varphi$.

\vspace{1ex}

%%%%%%%%%%%%%%%%%%%%%%%%%%%%% COMENT %%%%%%%%%%%%%%%%%%%%%%%%%%%%%%%%
\begin{comment}
{\bf 3.}  By uniqueness it follows that the pull-back of a composition is the composition of the pull-backs.
 For $X \mr{f} Y \mr{g}$, $(g f)^* = g^* f^*$. In a diagram:
$$
\xymatrix
   {
  & {}
    X^* \ar[r] 
        \ar@{-->}[d]_{f^*}
        \ar@/_2pc/[drdl]_{(g f) ^*} 
  & X \ar[d]^{f}
      \ar@/^2pc/[drdl]^{g f} 
  \\
    {} \ar@{}[r]|{\hspace{2ex}\equiv}
  & Y^* \ar[r] 
        \ar@{-->}[d]_{g^*}  
  & Y \ar@{}[ul]|\equiv
      \ar[d]^{g}
      \ar@{}[r]|{\hspace{-1.75ex}\equiv} 
  & {}
  \\ 
  & {} 
    Z^* \ar[r]  
  & Z \ar@{}[ul]|\equiv  
  \\
  & {}
    \alpha \ar[r]^{\varphi} 
  & \beta
   }
$$
%In a fibration the squares are in fact pullback diagrams in $\cc{E}$, this follows because cartesian arrows are strong cartesian. \cqd
\end{comment}
%%%%%%%%%%%%%%%%%%%%%%%%%%%%% END %%%%%%%%%%%%%%%%%%%%%%%%%%%%%%%%%%%%

\item \label{vdpullback}
\emph{Finite vertical diagrams are pulled back as follows:} 

Let $\cc{D}$ be a finite category and  $\cc{D} \mr{X} \cc{E}_\beta$ a diagram:  

%\vspace{1ex}. 

Choose for each $i \in \cc{D}$ a cartesian arrow $X_i^* \tocar X_i$. 
For each $i \mr{f} j$ we have an arrow  $ X_i \xr{X_f} X_j $ in $\cc{E}$ which determines an arrow  $X_i^* \xr{X_f^*} X_j^*$ as in \ref{vpull}. For $i \mr{f} j \mr{g} k$ in $\cc{D}$ the equation 
$X_{gf}^* = X_g^* X_f^*$ follows by uniqueness. In a diagram:
$$
\xymatrix
   {
  & {}
    X^* \ar[r]|\car 
        \ar@{-->}[d]^{X_f^*}
        \ar@{-->}@/_2pc/[dd]_{X_{gf}^*} 
  & X \ar[d]^{X_f}
      \ar@/^2pc/[dd]^{X_{gf}} 
  \\
    %{} \ar@{}[r]|{\hspace{2ex}\equiv}
  & Y^* \ar[r]|\car 
        \ar@{-->}[d]^{X_g^*}  
  & Y %\ar@{}[ul]|\equiv
      \ar[d]^{X_g}
      %\ar@{}[r]|{\hspace{-1.75ex}\equiv} 
  & {}
  \\ 
  & {} 
    Z^* \ar[r]|\car 
  & Z %\ar@{}[ul]|\equiv  
  \\
  & {}
    \alpha \ar[r]^{\varphi} 
  & \beta
   }
$$
Thus we have a diagram $\cc{D} \mr{X^*} \cc{E}_\alpha$ in 
$\cc{E}_\alpha$,  and a cartesian natural transformation 
$j_\alpha X^* \tocar j_\beta X$ over $\varphi$, that is, for all $i$,
$X_i^* \tocar X_i$ is a cartesian \mbox{arrow over $\varphi$.} 

\item  
Recall that a \emph{cone} is the same thing that a natural transformation $\Delta C \mr{p} X$, where $\Delta C$ is the \emph{constant diagram}: 

$\forall i \in \cc{D}$, 
$(\Delta C)_i = C$,
$(\Delta C)(i \mr{m} j) = C \mr{id_C} C$. 

Given 
$A \mr{f} C$, $\Delta A \mr{\Delta_f} \Delta C$ is the constant natural transformation: $(\Delta_f)_i = f$.

\item 
\emph{Vertical cones of a finite vertical diagram are pulled back as follows:}

Let $C \mr{p_i} X_i$ be a cone in $\cc{E}_\beta$. 

Choose a cartesian arrow $C^* \tocar C$, pulling back $p_i$ we get arrows  
$C^* \mr{p_i^*} X_i^*$ in $\cc{E}_\alpha$. The cone equations follow by uniqueness as in \ref{vdpullback}.  
\end{enumerate}
\end{sinnadastandard}
%%%%%%%%%%%%%%%%%%%%%%%%%%%%%%%%%%%%%%%%%%%%%%%%%%%%%%%%%%%%%%%%%%%%%
\begin{comment}
 natural transformation 
 $j_\alpha \circ X^*  \Mr{\eta} j_b \circ X $ as in the following diagram: 
$$
\xymatrix
   {
  & \cc{D} \ar[dl]_{X^*} 
           \ar[dr]^{X} 
  & & &
  \\
    \cc{E}_\alpha \ar[dr]_{j_\alpha}  
                  \ar@{}[rr]|{\overset{\eta}{\implies}} 
 && \cc{E}_\beta \ar[dl]^{j_\beta} 
 &  X_i^*  \ar[r]^{\eta_i} 
 &  X_i
 &  {} 
  \\
 &\cc{E} 
 && \alpha \ar[r]^{\varphi} 
 & \beta
   }
$$
\end{comment}
%%%%%%%%%%%%%%%%%%%%%%%%%%%%%%%%%%%%%%%%%%%%%%%%%%%%%%%%%%%%%%%%%%%%%

\begin{sinnadastandard}
[{\bf Pulling back along a cone of $\cc{G}$}]
\label{pbk2} $ $


Here it is necessary to assume that the cartesian arrows compose, which implies that in \ref{pbk1}, 
\ref{vdpullback} we  have in addition the equations 
$X^{**} = X^*$, $f^{**} = f^*$. 
%\begin{definitionst} \label{nvpbk1}${}$

Let $\cc{E} \mr{F} \cc{G}$ be a fibration.

\begin{enumerate}

%%%%%%%%%%%%%%%%%%%%%%% COMMENT %%%%%%%%%%%%%%%%%%%%%%%%%%%%%%%%%%%
%\begin{comment}
\item \emph{Non vertical arrows are pulling back as follows:}
Let $X \mr{f} Y$ an arrow in 
$\cc{E}$ over $\alpha \mr{\varphi} \beta$. Let $\delta \mr{\phi} \alpha$, 
$\delta  \mr{\mu} \beta$ arrows in $\cc{G}$ such that  
$\varphi \phi = \mu$. Then choosing cartesian arrows, $f$ is pulled back along $\mu$ to $\cc{E}_\delta$, as shown in the following diagram:
%
%$\xymatrix{X \ar[r]|\circleddash & Y}$
%$\xymatrix{X \ar[r]|{\car} & Y}$
%
$$
\xymatrix 
    {
     X^* \ar[r]|\car
         \ar@{-->}[d]^{f^*}
   & X \ar@{-->}[d]^{f^*}
       \ar@/^0.5pc/[dr]^f
  \\
     Y^* \ar[r]|\car 
   & Y^* \ar[r]|\car
   & Y 
  \\
     \delta \ar[r]^\phi 
   & \alpha \ar[r]^\varphi
   & \beta
    }
$$
%\end{comment}
%%%%%%%%%%%%%%%%%%%%% END COMMENT %%%%%%%%%%%%%%%%%%%%%%%%%%%%%%%%%%%
\emph{Remark:} Notice that from \ref{basicdefinitions}, 
\ref{cartesiancomposing} and \ref{cartesianiffiso}, it follows that if $f$ is cartesian, then the arrows $f^*$ are isomorphisms.

\item \emph{Finite diagrams are pulled back as follows:}
\label{hdpullback}
Let $\cc{D}$ be a finite category and  
$\cc{D} \mr{X} \cc{E}$ a diagram in $\cc{E}$. Denote the diagram $FX$ in 
$\cc{G}$ by $FX_i = \alpha_i$ and $FX_f = \varphi_f$. Let 
$\delta \mr{\phi_i} \alpha_i$ be a cone of FX in $\cc{G}$.

%\vspace{1ex}. 

Choose for each $i \in \cc{D}$ a cartesian arrow $X_i^* \tocar X_i$. 
For each $i \mr{f} j$ we have an arrow  $ X_i \mr{X_f} X_j $ in $\cc{E}$. Pulling back $X_f$ along $\phi_j$ we get arrows  $X_i^* \mr{X_f^*} X_j^*$ in $\cc{E}_\delta$ as in {\bf 1.} For 
$i \mr{f} j \mr{g} k$ in $\cc{D}$ the equation 
$X_{gf}^* = X_g^* X_f^*$ follows by uniqueness. In a diagram:
%
%{\bf 2.}   By uniqueness it follows that the pull-back of a composition is %the composition of the pull-backs.
% For $X \mr{f} Y \mr{g}$, $(g f)^* = g^* f^*$. In a diagram:
%
$$
\xymatrix 
    {
     X_i^* \ar[r]|\car
           \ar@{-->}[d]^{X_f^*}
           \ar@{-->}@/_2pc/[dd]_{X_{gf}^*}
   & X_i   \ar@{-->}[d]^{X_f^*}
           \ar@/^0.4pc/[dr]^{\hspace{-1.2ex}X_f}
           \ar@{-->}@/^2.4pc/[drdr]^{\hspace{-1.2ex}X_{gf}}
  \\
     X_j^* \ar[r]|\car 
           \ar@{-->}[d]^{X_g^*}
   & X_j^* \ar[r]|\car
           \ar@{-->}[d]^{X_g^*}
   & X_j   \ar@{-->}[d]^{X_g^*}
           \ar@/^0.4pc/[dr]^{\hspace{-1.2ex}X_g}
  \\ 
     X_k^* \ar[r]|\car
   & X_k^* \ar[r]|\car
   & X_k^* \ar[r]|\car
   & X_k
  \\
     \delta \ar[r]^{\phi_i} 
   & \alpha_i \ar[r]^{\varphi_f}
   & \alpha_j  \ar[r]^{\varphi_g}
   & \alpha_k
    }
$$
Thus we have  a diagram 
$\cc{D} \mr{X^*} \cc{E}_\delta$ in $\cc{E}_\delta$,
and a cartesian natural transformation 
$j_\delta X^* \tocar X$ over the cone $\delta \mr{\phi_i} \alpha_i$, that is, 
$X_i^* \tocar X_i$ is a cartesian arrow over $\phi_i$ for all $i$.
%\end{definitionst}

% Recall that a cone is the same thing that a natural transformation $\Delta C \mr{p} X$.

%\vspace{1ex}.  %arrows $C^* \mr{} X_i^*$,

\item \emph{Cones of a finite diagram are pulled back as follows:} 

{\erojo  Considerar eliminar este item ya que no es necesario mas adelante}
 
Let $C \mr{p_i} X_i$ be a cone of $X$ in $\cc{E}$, which sits over a cone 
$\alpha \mr{\pi_i} \alpha_i$ of $FX$ in $\cc{G}$. 
Let $\delta \mr{\phi} \alpha$ in $\cc{G}$, and choose a cartesian arrow 
$C^* \tocar C$ over $\phi$. Pulling back $p_i$ along $\pi_i \phi$ we get arrows 
$C^* \mr{p_i^*} X_i^*$ in $\cc{E}_\alpha$.  The cone equations follow by uniqueness as in \ref{hdpullback}.
\end{enumerate}
\end{sinnadastandard}

\vspace{1ex}

To study the stability properties of the pulling back operations, for clarity of exposition and proofs we find convenient to introduce the following construction.
 
Let $\cc{D}$ be a finite category, 
$\cc{E} \mr{F} \cc{G}$ be a prefibration and $\alpha \mr{\varphi} \beta$ an arrow in $\cc{G}$. We consider the following prefibered subcategory of the functor category $\cc{E}^\cc{D}$, that we denote  $\cc{E}^{(\cc{D})}$.

\begin{definitionst}[{\bf The $\Delta$-exponential}]  \label{ExpoDelta} 
We omit the label $j_\alpha$. 

\vspace{1ex}
 
\emph{Objects}:  Pairs $(X, \, \alpha)$, 
$X \in \cc{E}_\alpha^\cc{D} \subset \cc{E}^\cc{D},\; \alpha \in \cc{G}.$  

\vspace{1ex}

\emph{Morphisms}:  Pairs $(\eta,\, \varphi)$ 
$$
\xymatrix@R=0ex@C=0ex
    {
   && (X, \, \alpha) \ar[r]^{(\eta, \, \varphi)}
    & (Y, \, \beta)
   \\
      {} \ar@{-}[rrrrrr]
   &&&&&& {}
   \\
   &&
   (X \mr{\eta} Y,\, \alpha \mr{\varphi} \beta), & 
   F(\eta_i) = \varphi \; \forall i \in \cc{D}
   }
$$
The composition is clear, $\cc{E}^{(\cc{D})} \subset \cc{E}^\cc{D} \times \cc{G}$. Since the fibers are disjoint the first projection determines  
$\cc{E}^{(\cc{D})}$ as a subcategory of the functor category
 $\cc{E}^\cc{D}$, $\cc{E}^{(\cc{D})} \mr{i} \cc{E}^\cc{D}$.  
 %we denote the inclusion $\cc{E}^{(\cc{D})} \mr{i} \cc{E}^\cc{D}$.  

\vspace{1ex}
%
\emph{The prefibered structure}.
The second projection defines a functor 
 $\cc{E}^{(\cc{D})} \mr{F^{(\cc{D})}} \cc{G}$ whose fibers are 
 $(\cc{E}^{(\cc{D})})_\alpha = \cc{E}_\alpha^\cc{D}$. 
%           
%$\xymatrix{G \ar@{|->}[r] & F(G)}$, 
%$\xymatrix{\eta \ar@{|->}[r] & \rho}$
%  
The functor $F^{(\cc{D})}$ is prefibred, we define a morphism to be cartesian if and only if for every $i \in \cc{D}$ the morphisms $\eta_i$ are cartesian. Then, given $(X,\, \beta)$ and 
$\alpha \mr{\varphi} \beta$, pulling back $X$ as in Definition 
\ref{pbk1}, 2., determines an object $(X^*, \, \alpha)$ and a cartesian arrow  
$(X^*, \, \alpha) \mr{} (X,\, \beta)$, which is the pull-back in 
$\cc{E}^{(\cc{D})}$.
It is clear that if $F$ is a fibration, so it is $\cc{E}^{(\cc{D})}$.
\cqd 
   
%$C^* \mr{p_i^*} X_i^*$,  
%$\Delta C^* \mr{p} X^*$, 
%$(\Delta C^*, \, \alpha) \mr{(p^*,\, id_\alpha)} (X^*, \alpha)$, which is the pull-back cone in $\cc{E}^{(\cc{D})}$.
%\cqd
%Consequently if $F$ is a fibration, then $F^{(\cc{D})}$ is a fibration. 
%Similarly if $F$ is cleaved (split), then so is $F^{(\cc{D})}$.
\end{definitionst}

\begin{definitionst}[{\bf The diagonal functor}]
There is a cartesian $\cc{G}$-functor 
$\cc{E} \xr{(\Delta, F)} \cc{E}^{(\cc{D})}$ which sends an object $C$ into the pair $(\Delta C, FC)$, where $\Delta C$ is the constant diagram, see definition  \ref{pbk1}, 3..
 
%
%We denote $\cc{E}_\beta \mr{\Delta_\beta} \cc{E}^{(\cc{D})}$ the restriction of $\Delta$ to the fiber, thus $\Delta_\beta = \Delta \circ j _\beta$. 
%
%We will omit the label $\Delta$ and just write $C$ also for the constant %diagram.
\end{definitionst}

Although in this paper we have no use of this remark, it is pertinent to mention the following:
\begin{remark}\label{F(D) es fibracion} 
Let  $\cc{E}^{\cc{D}} \mr{F^{\cc{D}}}   \cc{G}^{\cc{D}}$ be the functor defined as postcomposing with $F$. The functor $F^{(\cc{D})}$ is a pullback of $F^{\cc{D}}$ along $\cc{G} \mr{\Delta} \cc{G}^{\cc{D}}$ in $\cc{C}at$ as indicated in the following diagram,  where $\delta = (\Delta, F)$:
$$
\xymatrix{\cc{E} \ar@/^1.5pc/[rrd]^\Delta \ar@/_2.pc/[ddr]_F \ar[dr]^{\delta} & \\
		  &\cc{E}^{(\cc{D})} \ar@{}[l]|\equiv \ar@{}[u]|\equiv \ar[r]^i \ar[d]_{F^{(\cc{D})}} & \cc{E}^{\cc{D}}  \ar[d]^{F^{\cc{D}}} \\
		  &\cc{G} \ar[r]_\Delta & \cc{G}^{\cc{D}} \ar@{}[ul]|{p.b.}}
$$ 

\vspace{-4ex}

\cqd
\end{remark}

\begin{nobservation} \label{cone=arrow}
For a diagram $\cc{D} \mr{X} \cc{E}_\beta$, note that a cone 
$C \mr{p_i} X_i$ in the fiber $\cc{E}_\beta$    
is the same thing that a natural transformation $\Delta C \mr{p} X$ such that the pair   
 \mbox{$(\Delta C, \, \beta) \xr{(p,\, id_\beta)} (X, \beta)$} is an arrow in 
 $\cc{E}^{(\cc{D})}$, \;    
 $c\,\cc{E}_\beta(C, X) = 
 (\cc{E}^{(\cc{D})})_\beta[(\Delta C, \beta), (X, \beta)]$.
 \end{nobservation}
 
\vspace{1ex}

More generally, given $\alpha \mr{\varphi} \beta$ in $\cc{G}$, and 
a object $A$ over $\alpha$, a  cone  $A \mr{\pi_i} X_i$ over $\varphi$
is the same thing that a natural transformation $\Delta A \mr{p} X$ such that the pair $(\Delta A, \, \alpha) \xr{(p,\, \varphi)} (X, \beta)$ is an arrow in $\cc{E}^{(\cc{D})}$, \; 
$c\,\cc{E}_\varphi(A, X) = 
 (\cc{E}^{(\cc{D})})_\varphi[(\Delta A, \alpha), (X, \beta)]$.
%


%%%%%%%%%%%%%%%%%%%%%%%%%%%%%%%%%%%%%%%%%%%%%%%%%%%%%%%%%%%%%%%%%%%%%
\begin{comment}
 natural transformation 
 j_\alpha \circ X^*  \Mr{\eta} j_b \circ X $ as in the following diagram: 
$$
\xymatrix
   {
  & \cc{D} \ar[dl]_{X^*} 
           \ar[dr]^{X} 
  & & &
  \\
    \cc{E}_\alpha \ar[dr]_{j_\alpha}  
                  \ar@{}[rr]|{\overset{\eta}{\implies}} 
 && \cc{E}_\beta \ar[dl]^{j_\beta} 
 &  X_i^*  \ar[r]^{\eta_i} 
 &  X_i
 &  {} 
  \\
 &\cc{E} 
 && \alpha \ar[r]^{\varphi} 
 & \beta
   }
$$
\end{comment}
%%%%%%%%%%%%%%%%%%%%%%%%%%%%%%%%%%%%%%%%%%%%%%%%%%%%%%%%%%%%%%%%%%%%%

%$\{C^* \mr{p_i^*} X_i^*\}_{i \in \cc{D}}$ which define a cone 
%$C^* \Mr{p^*} X^*$ in  $\cc{E}_\alpha$. 
 
%We say that  $C^* \Mr{p^*} X^*$ is a pull-back cone along $\varphi$ of 
%$C \Mr{p} X$.
%\end{definitionst}
%
\begin{definition} 
For a category $\cc{D}$, we say that limits of type $\cc{D}$ are
\emph{stable} if for every $\alpha \mr{\varphi} \beta$ in $\cc{G}$,  and limit cone  
$C \mr{\pi_i} X$ in $\cc{E}_\beta$, the pull-back cone  
$C^* \mr{\pi^*_i} X^*_i$ is a limit cone in $\cc{E}_\alpha$.
\end{definition}
%
\begin{proposition}\label{prefi mas precofib preserva limites}
If $\cc{E} \mr{F} \cc{G}$ is prefibered and precofibred, then limits of type $\cc{D}$ are stable for any category $\cc{D}$ with a finite set of objects, including $\cc{D} = \emptyset$.
\end{proposition}
\begin{proof}
Using Proposition \ref{adjoints} the proof follows by the usual argument that functors with a left adjoint preserves limits.   
\end{proof} 
%}
\begin{nobservation} \label{lcone=larrow}
Recall that a cone  
\mbox{$C \mr{\pi_i} X_i$} in a fiber $\cc{E}_\beta$ is a 
\emph{limit} cone in $\cc{E}_\beta$ if for any object $A$ over $\beta$, the map 
$\cc{E}_\beta(A,C) \mr{\pi_*} c \,\cc{E}_\beta(A,X)$ is a bijection.

\vspace{1ex}

From Observation \ref{cone=arrow} we see that this is the same as to say that the map  
%$(\cc{E}^{(D)})_\beta[(\Delta A,\,\beta), (X,\,\beta)]$ 
$\xymatrix@C=9ex
   {
    (\cc{E}^{(D)})_\beta[(\Delta A,\,\beta), (\Delta C,\,\beta)]_{+} 
    %\cc{E}_\beta(A,C)  
    \ar[r]^{(\pi,\, id_\beta)_*}
    %\ar[r]^{\pi_*}
  & %\cc{E}_\varphi(A,C)
  (\cc{E}^{(D)})_\beta[(\Delta A,\,\beta), (X,\,\beta)]
    }
$
is a bijection, where by  $[\;\;\;]_{+}$ we denote the subset of constant natural transformations. \cqd
\end{nobservation}
\begin{definition} \label{prepreserves}
We say that a cone  $C \mr{\pi_i} X_i$ in a fiber $\cc{E}_\beta$ is a $F$-limit if for every $\alpha \mr{\varphi} \beta$ in 
$\cc{G}$, and a object $A$ over $\alpha$, the map 
$\cc{E}_\varphi(A,C) \mr{\pi_*} c \,\cc{E}_\varphi(A,X)$ is a bijection.

\vspace{1ex}

From Observation \ref{cone=arrow} we see that this is the same as to say that the map  
$\xymatrix@C=9ex
   {
    (\cc{E}^{(D)})_\varphi[(\Delta A,\,\alpha), (\Delta C,\,\beta)]_{+} 
    \ar[r]^{(\pi,\, id_\beta)_*}
   & (\cc{E}^{(D)})_\varphi[(\Delta A,\,\alpha), (X,\,\beta)]
    }
$
is a bijection.   
\end{definition}

Clearly by the respective definition it follows
\begin{observation}  
\emph 
   {
    limit in  $\cc{E}$ \; $\Rightarrow$ \;
    F-limit \; $\Rightarrow$ \;
    limit in  $\cc{E}_\beta$ 
   }
\end{observation} 
%
The following is a characterization of stable limits.
\begin{proposition} \label{preGray4.1}
A cone $C \mr{\pi_i} X_i$ is a $F$-limit cone if and only if, all, or any, of its pullback cones $C^* \mr{\pi^*_i} X^*_i$ are limit cones.
\end{proposition}
\begin{proof}
Let $\Delta C \mr{p} X$ be a cone in $\cc{E}_\beta$, and consider the following commutative diagram:
$$
\xymatrix
    {
     (\cc{E}^{(D)})_\alpha[(\Delta A,\,\alpha), (\Delta C^*,\,\alpha)]_{+}  
                  \ar[r]^\approx
                  \ar[d]^{(\pi^*, id_\alpha)_*} 
   & (\cc{E}^{(D)})_\varphi[(\Delta A,\,\alpha), (\Delta C,\,\beta)]_{+} 
                  \ar[d]^{(\pi,\, id_\beta)_*}
  \\
     (\cc{E}^{(D)})_\alpha[(\Delta A,\,\alpha), (X^*,\,\alpha)]
                 \ar[r]^\approx
   & (\cc{E}^{(D)})_\varphi[(\Delta A,\,\alpha), (X,\,\beta)]
    }	  
$$
The horizontal arrows are bijections by definition of cartesian arrows, which shows the equivalence between the bijectivity condition for the vertical arrows.   
\end{proof}
Since limits in $\cc{E}$ are in particular $F$-limits, we have:
\begin{corollary}
If the functors $\cc{E}_\alpha \mr{j_\alpha} \cc{E}$ preserve limits of type 
$\cc{D}$, then limits of \mbox{type $\cc{D}$} are stable. \cqd
\end{corollary}

The reverse implication does not hold in general. For $\cc{D} = \emptyset$, trivially the identity functor of any category with two non isomorphic objects is a fibration where the fibers have stable terminal objects which can not all be terminal in the whole category. Next we show an example with a non empty $\cc{D}$. 
\begin{example}
The following describes a fibration in which products are stable  but the inclusion functors of the fibres don't preserve products. 
Take \mbox{$\cc{G} = \{\alpha \mrpair{\varphi_1}{\varphi_2} \beta\}$,} and consider the prefibration $\cc{E} \mr{} \cc{G}$ described in the following diagram:
$$
\xymatrix
    {
	 W \ar[d]_{c_i}          
   \\
     P_i \ar[r]^{p_i} 
         \ar[d]_{a_i} 
         \ar@/_2pc/[dd]_{b_i} 
   & P \ar[d]^{a}   
       \ar@/^2pc/[dd]^{b}   
   \\
	 X_i \ar[r]^{x_i} 
   & X       
    \\
	 Y_i \ar[r]^{y_i} 
   & Y       
    \\
     \alpha \ar[r]^{\varphi_i} 
   & \beta
}
$$
Where $\cc{E}$  is the finite category generated by the arrows $p_i$, $a_i$, $b_i$, $c_i$, $x_i$ and $y_i$, $i = 0, \, 1$, satisfying  the relations $a.p_i = x_i.a_i$, $b.p_i = x_i.b_i$. We have a functor $F$ defined by the equations $F(a_i) = F(b_i) = F(c_i) = id_\alpha ,\,
F(a) =  F(b) = id_\beta ,\,$ and  \mbox{$F(p_i) = F(x_i) = F(y_i) = \varphi_i$.}

There are no arrows directly connecting the two diagrams $\{P_0 \mr{a_0} X_0,P_0 \mr{b_0} Y_0\}$ and $\{P_1 \mr{a_1} X_1,P_1 \mr{b_1} Y_1\}$. It can be verified that these diagrams are products in $\cc{E}_\alpha$ and that the diagram $\{P \mr{a} X,P \mr{b} Y\}$ is a product in $\cc{E}_\beta$.  $F$ is a fibration where the arrows $p_i$, $x_i$ and $y_i$ are cartesian. 
$\{P_0 \mr{a_0} X_0,\, P_0 \mr{b_0} Y_0\}$ and 
\mbox{$\{P_1 \mr{a_1} X_1, \, P_1 \mr{b_1} Y_1\}$} are pullbacks of the diagram $\{P \mr{a} X,P \mr{b} Y\}$ along the arrows 
$\varphi_0$ and $\varphi_1$ respectively. Products are stable in this fibration but the diagram 
$\{P \mr{a} X,P \mr{b} Y\}$ is not a product en $\cc{E}$. This can be verified by the fact that the diagram 
$\{W \xr{x_0 a_0 c_0} X, \,W \xr{y_1 b_1 c_1} Y\}$ has no factorization through $P$. \cqd
\end{example}

%\vspace{1ex}
%
%A second example, worth noting and trivial, is the case in which the %category $\cc{D}$ is the empty category. Taking  
%$\cc{G} = \{\alpha \mrpair{\varphi_1}{\varphi_2} \beta\}$, and   
%$\cc{E} = \cc{G}$, the identity functor is a fibration where the fibers %have stable terminal objects, which are not terminal in $\cc{E}$ since
%$\cc{E}$ does not have terminal objects.   
%\end{example}
%

Next we consider a sufficient condition under which the functors $j_\alpha$ preserve stable limits.
 
\begin{proposition} \label{ifconnected}
Let $\cc{D}$ be a non empty connected category and 
$\cc{D} \mr{X} \cc{E}_\beta$ a vertical diagram. Then:

1.  For any cone 
$A \mr{p_i} X_i$, 
\mbox{$(\Delta C, \, \beta) \xr{(p,\, id_\beta)} (X, \beta)$}, with A sitting over $\alpha$, there exists 
$\alpha \mr{\varphi} \beta$ such  that 
$F(p_i) = \varphi \; \forall\, i$.

%\vspace{1ex}

2. The diagonal functors  $\cc{E} \mr{\Delta} \cc{E}^\cc{D}$ and 
$\cc{E} \mr{(\Delta, F)} \cc{E}^{(\cc{D})}$ are full and faithful.

%\vspace{1ex}

3 The functor   
$\cc{E}^{(\cc{D})} \mr{i} \cc{E}^{\cc{D}}$ is full and faithful. 
\end{proposition}
%
\begin{proof}
By assumption for any two $i, j \in \cc{D}$, there is a zig-zag of arrows  $\xymatrix{{i} \ar@2{~>}@<0.25ex>[r]^{{m}} & {j}}$.

1. Let $i, j \in \cc{D}$, we have a zig-zag 
$\xymatrix{{X_i} \ar@2{~>}@<0.25ex>[r]^{{X_m}} & {X_j}}$ in $\cc{E}_\beta$, such that 
 $p_j = X_m \circ p_i$. Thus  $F(p_j) = id_\beta  \circ F(p_i)$,  
 $F(p_j) = F(p_i)$.
.
 \vspace{1ex}
 
 2. Let $X,Y \in \cc{E}$ over $\alpha$, $\beta$ respectively, and 
 $\Delta X  \mr{\eta} \Delta Y$ in 
 $\cc{E}^{\cc{D}}$. Let $i, j \in \cc{D}$, the naturality condition on $\eta$ means 
 $\Delta Y(m) \circ \eta_i = \eta_j \circ \Delta X(m)$. Since $\Delta X(m) = id_X$, $\Delta Y(m) = id_Y$, we have  $\eta_i = \eta_j$. 
%Take any $k \in \cc{D}$, and let $f = \eta_k$. Then, $\eta = \Delta f$
For the functor $(\Delta, F)$ we do in the same way. 

\vspace{1ex}

3. We have a factorization $\Delta = i \circ (\Delta, F)$, the statement follows.
\end{proof} 
   
\begin{proposition} \label{stable=preserve}
Let $\cc{D}$ be any non empty connected category. Then limits of type 
$\cc{D}$ are stable if and only if the  the functors 
$\cc{E}_\alpha \mr{j_\alpha} \cc{E}$ preserve limits of type $\cc{D}$, that is, they remain universal in the category $\cc{E}$.
\end{proposition}
\begin{proof}
Let 
$C \mr{\pi_i} X_i$ be a limit cone  
in a fiber $\cc{E}_\beta$, then by proposition \ref{preGray4.1} and  \mbox{Proposition \ref{ifconnected}, 1., 3.} 
(recall Observation \ref{lcone=larrow}), it follows that this limit is stable if and only if the map
$\xymatrix@C=5ex
   {
    \cc{E}^{\cc{D}}[\Delta A, \, \Delta C]_{+} 
    \ar[r]^{\pi_*}
   & \cc{E}^{\cc{D}}[\Delta A, \, X]
    }
$
is a bijection. 
In turn by Proposition \ref{ifconnected}, 2., it follows that this map is a bijection if and only if the map
$\cc{E}(A, C) \mr{\pi_*} \cc{E}^\cc{D}(\Delta A, X)$ is a bijection, that is, if and only if the cone $C \mr{\pi_i} X_i$ is a limit cone in the category $\cc{E}$.
\end{proof}
%
%\rojo{esto es proposicion 4.1 de Gray}
%
%%%%%%%%%%%%%%%%%%%%%%%%%%%%%%% COMMENT %%%%%%%%%%%%%%%%%%%%%%%%%%%%%%%
\begin{comment}
{\erojo   
\begin{definition} \label{estabilidad}
 A subset $\cc{A} \subset \cc{E}$ of vertical arrows is 
 \emph{stable} if for any $X \mr{m} Y \in \cc{A}$ and pullback $X^* \mr{m^*} Y^*$, it follows that
 $m^* \in \cc{A}$
 \end{definition}

%\begin{definition}
%We will say that \textbf{isomorphisms (epimorphisms,...) are stable}  if given any vertical isomorphism (epimorphism,...) $m$ and pullback  $m^*$, it follows that  $m^*$ is also a pullbackof it is a vertical isomorphism (epimorphism,...). 
%\end{definition}
%
%\begin{nobservation}
%Isomorphisms are stable in any prefibration or precofibration.    
%\end{nobservation}


\subsubsection{Pulling back finite diagrams}

\begin{definition}
We will say that \emph{terminal objects are stable}  if given $1_\beta$ a terminal object of $\cc{E}_\beta$, then 
$(1_\beta)^*$ is a terminal object of $\cc{E}_\alpha$. That is, if we are given also a terminal object 
$1_\alpha$ of $\cc{E}_\alpha$,  the unique map 
$(1_\beta)^* \to 1_\alpha$ is an isomorphism.
\end{definition}


{Objects in $\cc{E}$ and vertical arrows are examples of a more general type of object that we can \textit{pullback} in a prefibration. It is well known that a functor preserves finite limits if and only if the functor preserves terminal objects and pullbacks. We have already developed a notion of stability for terminal object  and classes of morphisms in the previous section. In this section we define stability of pullbacks taken in a fibre. 


\rojo{FIX-002.01 START Usar 1 en lugar de t para representar objetos terminales y dar def del funtor $F^{(\cc{D})}$ sin requerir objeto terminal, solo $D$ no vacio}

%\verde{To encompass the three types of objects we will take a fixed finite category $\cc{D}$ with a terminal object $1$  and consider the set of functors ${\cc{D} \mr{G} \cc{E}}$ that factor through a fibre.}

{To encompass the three types of objects we will take a fixed finite category $\cc{D}$ with a terminal object $1$ and consider the set of functors $\cc{D} \mr{G} \cc{E}$ that factor through a fibre.}

\rojo{FIX-002.01 END }

\[
\xymatrix{& \cc{D} \ar[dr]^{G} \ar[dl]_{G_\alpha} \ar@{}[d]|(.6){\equiv} & 
\\
	 \cc{E}_\alpha \ar[rr]_{j_\alpha}  && \cc{E}}
\]



\noindent We will  allow the abuse of notation ${\cc{D} \mr{G} \cc{E}_\alpha}$ to indicate through which fibre such a functor factors. 

\rojo{FIX-002.02 START }


{
These functors form the objects of a category $\cc{E}^{(\cc{D})}$ whose morphisms from ${\cc{D} \mr{G} \cc{E}_\alpha}$ to ${\cc{D} \mr{H} \cc{E}_\beta}$ are natural transformations $G \overset{\eta}{\implies} H$ of functors $\cc{D} \mr{} \cc{E}$ that are projected onto a single arrow 
$\rho$ in $\cc{G}$, for every ${ d \in \cc{D}}$, $F(\eta_d) = \rho$. 
$$
\xymatrix
   {
  & \cc{D} \ar[dl]_{G_\alpha} 
           \ar[dr]^{H_\beta} 
  & & &
  \\
    \cc{E}_\alpha \ar[dr]_{j_\alpha}  
                  \ar@{}[rr]|{\overset{\eta}{\implies}} 
 && \cc{E}_\beta \ar[dl]^{j_\beta} 
 &  Gd \ar[r]^{\eta_d} 
 &  Hd 
 &  {} 
  \\
 &\cc{E} 
 && \alpha \ar[r]^{\rho} 
 & \beta
   }
$$
}

%\rojo{definir t como la imagen de F(nu sub t) y experesar que todos son iguales a t}

\rojo{FIX-002.02 END}

 \noindent 
The category $\cc{E}^{(\cc{D})}$ is a subcategory of the functor category $\cc{E}^{\cc{D}}$ and we denote the inclusion $\cc{E}^{(\cc{D})} \mr{i} \cc{E}^{\cc{D}}$. 

\rojo{FIX-002.03 START }

%\verde{The assignment ${\xymatrix{\eta \ar@{|->}[r] & t}}$ yields a functor ${\cc{E}^{(\cc{D})} \mr{F^{(\cc{D})}} \cc{G}}$. }

{The assignments ${\xymatrix{G \ar@{|->}[r] & F(G)}}$ and ${\xymatrix{\eta \ar@{|->}[r] & \rho}}$ yield a functor ${\cc{E}^{(\cc{D})} \mr{F^{(\cc{D})}} \cc{G}}$. }

\rojo{FIX-002.03 END }

For this functor there  is a natural identification between the categories $(\cc{E}^{(\cc{D})})_\alpha$ and ${\cc{E}_\alpha}^{\cc{D}}$ and we will allow the abuse of language of sometimes using the latter as if it were the fibre itself. 
  

\begin{remark}
The functor $F^{(\cc{D})}$ is prefibred. A morphism ${G \mr{\eta} H \in \cc{E}^{(\cc{D})}}$ is cartesian if and only if for every $d \in \cc{D}$ the morphisms $\eta_d$ are cartesian. Consequently if $F$ is a fibration, then $F^{(\cc{D})}$ is a fibration. Similarly if $F$ is cleaved (split), then so is $F^{(\cc{D})}$.
\end{remark}


\begin{observation}
For any category $\cc{A}$ and $X \in \cc{A}$ we will use $\cc{D} \mr{\Delta X} \cc{A}$ to denote the functor defined for every ${d \mr{a} d' \in \cc{D}}$ as ${(\Delta X)(d \mr{a} d')=X \mr{1_X} X}$. We call this the \textit{constant} functor $X$. For ${X \mr{f} Y \in \cc{A}}$ we associate a natural transformation between the constant functors $\Delta X \mr{\Delta f} \Delta Y$ defined as the constant family $f$. This yields \azul{the \emph{diagonal}} functor ${\cc{A} \mr{\Delta} \cc{A}^{\cc{D}}}$.
\end{observation}


\begin{remark}\label{F(D) es fibracion} 
Take  $\cc{E}^{\cc{D}} \mr{F^{\cc{D}}}   \cc{G}^{\cc{D}}$  the functor defined as postcomposing with $F$. The functor $F^{(\cc{D})}$ is a pullback of $F^{\cc{D}}$ along $\cc{G} \mr{\Delta} \cc{G}^{\cc{D}}$ in $\cc{C}at$ and we have the following diagram.

\[
\xymatrix{\cc{E} \ar@/^1.5pc/[rrd]^\Delta \ar@/_2.pc/[ddr]_F \ar[dr]^{\delta} & \\
		  &\cc{E}^{(\cc{D})} \ar@{}[l]|\equiv \ar@{}[u]|\equiv \ar[r]^i \ar[d]_{F^{(\cc{D})}} & \cc{E}^{\cc{D}}  \ar[d]^{F^{\cc{D}}} \\
		  &\cc{G} \ar[r]_\Delta & \cc{G}^{\cc{D}} \ar@{}[ul]|{p.b.}}
\]

\noindent The functor  $\delta$ is cartesian. If $F$ is cleaved, thenr $\delta$ is a morphism between cleaved functors.

\rojo{ver si al categoria pull-back es $\cc{E}^{((\cc{D}))}$ o $\cc{E}^{(\cc{D})}$ ?}
\end{remark}
\rojo{la imposicion de que un funtor de $\cc{E}^{\cc{D}}$ se factorize por $\Delta$ impone que ese functor se factoriza por una fibra de F }

\rojo{primero analizo factorizacion en los objetos en la exponencial y da presisamente que se factoriza por una fibra. Segundo analizo en una flecha y da justo la condicin de la definicion de $\cc{E}^{(\cc{D})}$}

\begin{remark}\label{FD es fibracion}
If $F$ is a prefibration,  $F^{\cc{D}}$ will not necessarily be a prefibration. This does not happen with fibrations. We show now that when $F$ is a fibration, so is $F^{\cc{D}}$. 

If $G \mr{f} H \in \cc{E}^{\cc{D}}$ satisfies that  for every ${d \in \cc{D}}$ the morphisms $f_d$ are strong cartesian, it follows that $f$ is cartesian. If $F$ is a fibration, given ${A \mr{\eta} B \in \cc{G}^{\cc{D}}}$ and ${X \in (\cc{E}^{\cc{D}})_B}$, a choice of strong cartesian morphisms ${{{X_d}^*} \mr{f_d} X_d}$ over $\eta_d$ determines a functor $\cc{D} \mr{X^*} \cc{E}$ over $A$ and a cartesian morphism $X^* \mr{f} X$ over $\eta$ (only finite choices are being made).

\[
\xymatrix{{X_d}^* \ar[rr]^{f_d} \ar@{-->}[dr]_{\exists ! X^*(a)} & & X_d \ar[dr]^{X(a)} \\
		  & {X_{d'}}^* \ar[rr]^{f_{d'}} \ar@{}[ur]|\equiv && X_{d'} \\
		  A_d \ar[rr]^{\eta_d} \ar[dr]_{G(a)} && B_d \ar[dr]^{H(a)} \\
		  & A_{d'} \ar[rr]^{\eta_{d'}} \ar@{}[ur]|\equiv && B_{d'}}
\]

\noindent Thus  the existence of the necessary cartesian morphisms for $F^{\cc{D}}$ to be a fibration follows without using choice. In such a case we have that $i$ in Remark \ref{F(D) es fibracion} preserves  cartesian morphisms. If $F$ is a cleaved (split) fibration, then $F^{\cc{D}}$ is  cleaved (split) fibration and $i$ preserves transport morphisms. 
\end{remark}

\begin{remark}
The conclusions in Remarks \ref{FD es fibracion} and \ref{F(D) es fibracion} are true if we replace the words \textit{fibration} and \textit{cartesian} for \textit{cofibration} and \textit{cocartesian} respectively. 
\end{remark}


\begin{observation}
Let \textbf{2} denote the category $\{0 \mr{} 1\}$. Vertical arrows in a fibration $F$ are naturally identified with the objects of $\cc{E}^{(\textbf{2})}$ and pulling back vertical arrows in $F$ is the same as pulling back objects in $F^{(\textbf{2})}$. This allows us to generalize the notion of pulling back diagrams of type $\cc{D}$ 
in a prefibration.
\end{observation}


\subsubsection{Pulling back cones}

\begin{observation}
For any category $\cc{A}$ and a functor $\cc{D} \mr{G} \cc{A}$, a cone ${\{C \mr{c_d} G_d\}_{d \in \cc{D}}}$ of $G$ in $\cc{A}$ is nothing but an arrow $\Delta C \mr{c} G$ in $\cc{A}^{\cc{D}}$. 
\end{observation}

We will call an arrow $\delta C \mr{c} G$ in ${\cc{E}_\alpha}^{\cc{D}}$ a \textit{vertical cone} in $F$.

\begin{definition}
We will say a vertical cone $\delta C \mr{c} G  \in {\cc{E}_\alpha}^{\cc{D}}$ is \textbf{universal} if ${\{C \mr{c_d} G_d\}_{d \in \cc{D}}}$ is a limit cone of $G$ in $\cc{E}_\alpha$.
\end{definition}
 
\begin{observation} 

A vertical cone $\delta C \mr{c} G$ in ${\cc{E}_\alpha}^{\cc{D}}$ is universal if and only if for every $X \in \cc{E}_\alpha$ and for every vertical cone $\delta X \mr{x} G$ in ${\cc{E}_\alpha}^{\cc{D}}$ there exists a unique $X \mr{f} C \in \cc{E}_\alpha$ such that $x=c \cdot \delta (f)$. That is for every $X \in \cc{E}_\alpha$ we have the following universal property.

\rojo{FIX-003.01 START escribir c en forma horizontal. Recordar renombrar label del diagrama}

$$
\xymatrix
   {
    X \ar@{-->}[d]_{\exists ! f} 
  & \delta X \ar[d]_{\delta f} 
             \ar@/^1.5pc/[drdl]^{\forall} 
  & {}
  \\
    C 
  & \delta C \ar[d]_c 
  & {}
  \\
  & G 
  & {}
  \\
    \alpha 
  & \alpha 
   }
$$
{
\begin{align} \label{cono universal pedorro}
\xymatrix{X \ar@{-->}[d]_{\exists ! f}  & \delta X \ar[d]_{\delta f} \ar[rd]^(.4){\forall} &  \\
	C	   & \delta C \ar[r]_c  \ar@{}[ur]|(.3)\equiv & G \\
		  \alpha & \alpha \ar[r]^{id_\alpha} & \alpha}
\end{align}
}
\rojo{escribir "c" en forma horizontal siendo que es vertical no es una buena idea}

\rojo{FIX-003.01 END}

\end{observation}

\begin{observation}
 We can pullback a cone $\delta C \mr{c} G \in {\cc{E}_\beta}^{\cc{D}}$  along  ${\alpha \mr{\varphi} \beta \in \cc{G}}$ to a vertical cone in ${\cc{E}_\alpha}^{\cc{D}}$  choosing a cartesian morphism ${C^* \mr{s} C}$ over $\varphi$ in $F$ and a cartesian morphism ${G^* \mr{t} G}$ over $\varphi$ in $F^{(\cc{D})}$. 

\begin{align}\label{pullback directo de un cono}
\xymatrix{\delta (C^*) \ar[r]^{\delta s} \ar@{-->}[d]_{\exists !} & \delta C \ar[d]^c\\
		  G^* \ar[r]_t \ar@{}[ur]|\equiv & G \\
		  \alpha \ar[r]^\varphi & \beta}
\end{align}
\end{observation}

\begin{definition}
We will say that \textbf{limits of type $\cc{D}$ are stable}  if vertical universal cones are stable for pullbacks such as \ref{pullback directo de un cono}.
\end{definition}

\begin{observation}\label{definicion mala de estable}
Pullbacks are limits of functors whose domain is the following category $\cc{P}$.

\[
\xymatrix{ &  0  \ar [d] \\
		  2  \ar [r]  &  1}
\]

\end{observation}

\begin{definition}
 We will say that \textbf{pullbacks are stable}  if limits of type $\cc{P}$ are stable.
\end{definition}



\subsubsection{{On the stability of finite limits.}  
\rosa{A property equivalent to the stability of pullbacks}}

We will give a more comprehensive characterization of the stability of pullbacks in a fibration. Nevertheless we will develop it for the general type of finite category $\cc{D}$ with terminal object t.

\rojo{cambiar la notacion "t" para un objeto terminal}

\begin{lemma}
For every category $\cc{A}$ the functor $\cc{A} \mr{\Delta} \cc{A}^{\cc{D}}$ is fully faithful (this actually holds for any connected category $\cc{D}$).
\end{lemma}

\begin{proof}
 For $X,Y \in \cc{A}$ and ${\Delta X \mr{\eta} \Delta Y \in \cc{A}^{\cc{D}}}$ we have ${\eta=\Delta(\eta_t)}$. This is because for every $d\in\cc{D}$ the unique arrow $d \mr{} t \in \cc{D}$ yields the following diagram.

\[
\xymatrix{d \ar[d]& X \ar[d]_{1_X} \ar[r]^{\eta_d} & Y \ar[d]^{1_Y} \\
		  t & X \ar@{}[ur]|\equiv \ar[r]_{\eta_t} & Y }
\]


\end{proof}

\begin{corollary}
$\cc{E} \mr{\delta} \cc{E}^{(\cc{D})}$ and  $\cc{E}^{(\cc{D})} \mr{i} \cc{E}^{\cc{D}}$ are fully faithful functors. 
\end{corollary}

\begin{proof}
Since fully faithful functors are stable in $\cc{C}at$  \cite[page 128]{sga1} it follows from Remark \ref{F(D) es fibracion}  that  $i$ and consequently $\delta$ are fully faithful. 
\end{proof}

\begin{remark}\label{caracterization of universal cones}
In this context we can merge the sets $hom_{\cc{E}}(X,Y)$ and ${hom_{\cc{E}^{(\cc{D})}}(G,H)}$ with $hom_{\cc{E}^{(\cc{D})}}(\delta X,\delta Y)$ and ${hom_{\cc{E}^{\cc{D}}}(iG,iH)}$ respectively. Thus we will  adopt the abuse of notation of suppressing $\delta$, $\Delta$ and $i$ in our expressions. Looking at diagram \ref{cono universal pedorro} we have that a vertical cone $C \mr{c} G$ in ${\cc{E}_\alpha}^{\cc{D}}$ is universal if and only if it verifies the following universal property for every $X \in \cc{E}_\alpha$. 

\[
\xymatrix{X \ar[dr]^{\forall} \ar@{-->}[d]_{\exists ! f} & \\
		  C \ar[r]_c \ar@{}[ur]|(.3)\equiv & G \\
		  \alpha \ar[r]^{id_\alpha} & \alpha}
\]

\noindent That is for every $X \in \cc{E}_\alpha$ the function 

\begin{align*}
hom_\alpha(X,C) \mr{c_*} hom_\alpha(X,G)
\end{align*}
\noindent is a bijection.

\rojo{ }

\end{remark}

\begin{proposition}\label{preservacion de limite intrinseca}
Limits of type $\cc{D}$ are stable  if and only if for every ${\alpha \mr{\varphi} \beta \in \cc{G}}$, every $\cc{D} \mr{G} \cc{E}_\beta$ and every vertical universal cone ${C \mr{c} G \in {\cc{E}_\beta}^{\cc{D}}}$ we have that for every $X \in \cc{E}_\alpha$ the function 

\[
hom_\varphi(X,C) \mr{c_*} hom_\varphi(X,G)
\]

\noindent is a bijection.
\end{proposition}

\begin{proof}
Take cartesian morphisms $C^* \mr{s} C$ and $G^* \mr{t} G$ over $\varphi$. Diagram \ref{pullback directo de un cono} can be written as follows.

\begin{align*}
\xymatrix{C^* \ar[r]^s \ar[d]_{c^*} & C \ar[d]^c \\
		  G^* \ar[r]^t & G \ar@{}[ul]|\equiv \\
		  \alpha \ar[r]^\varphi & \beta}
\end{align*}
\noindent Thus we have the following commutative diagram.

\begin{align*}
\xymatrix{hom_\alpha(X,C^*) \ar[r]^{s_*} \ar[d]_{(c^*)_*} & hom_\varphi(X,C) \ar[d]^{c_*}\\
		  hom_\alpha(X,G^*) \ar[r]_{t_*} & hom_\varphi(X,G) \ar@{}[ul]|\equiv }
\end{align*}

\noindent The result follows from this diagram and Remark \ref{caracterization of universal cones}.
\end{proof}

\begin{remark}\label{conos en E estan sobre una flecha de G}
For a functor $\cc{D} \mr{G} \cc{E}$ a cone $C \mr{c} G \in \cc{E}^{\cc{D}}$ of $G$ in $\cc{E}$ is a limit cone if and only if for every $X \in \cc{E}$  the following function is a bijection.

\begin{align*}
hom_{\cc{E}^{\cc{D}}}(X,C) \mr{c_*} hom_{\cc{E}^{\cc{D}}}(X,G)
\end{align*}

\end{remark}

\begin{proposition}\label{equivalencia con pullbacks estables}
Limits of type $\cc{D}$ are stable if and only if the functors $\cc{E}_\alpha \mr{j_\alpha} \cc{E}$ preserves limits of type $\cc{D}$.
\end{proposition}

\begin{proof}
Take $\cc{D} \mr{G} \cc{E}_\alpha$ a functor and $C \mr{c} G$ a universal cone of $G$ in $\cc{E}_\alpha$. The result follows from Proposition \ref{preservacion de limite intrinseca}, Remark \ref{caracterization of universal cones}, Remark \ref{conos en E estan sobre una flecha de G} and the following sequence.

\[
{hom_{\cc{E}^{(\cc{D})}}(X,C)=\underset{\varphi}{\amalg}\ hom_\varphi(X,C) \mr{c_*} \underset{\varphi}{\amalg}\ hom_\varphi(X,G)=hom_{\cc{E}^{(\cc{D})}}( X,G)}
\]
\end{proof}

\begin{theorem}\label{prefi mas precofib preserva limites}
If $F$ is precofibred, then terminal objects and limits of type $\cc{D}$ are stable.
\end{theorem}

\begin{proof}
For $\alpha \mr{\varphi} \beta \in \cc{G}$, $1_\beta$ a terminal object of $\cc{E}_\beta$ and $(1_\beta)^* \mr{s} 1_\beta$  cartesian over $\varphi$ we will prove that $(1_\beta)^*$ is a terminal object in $\cc{E}_\alpha$. Take $X\in \cc{E}_\alpha$ and $X \mr{r} X_*$ cocartesian over $\varphi$. we have the following diagram.

\begin{align}\label{terminal sobre phi}
hom_\alpha(X,(1_\beta)^*) \mr{s_*}  hom_\varphi(X,1_\beta) \ml{r^*} hom_\beta(X_*,1_\beta)
\end{align}

\noindent Both arrows are bijections and the set on the right is a singleton. The result follows.

 Take $C \mr{c} G$ a universal vertical cone in ${\cc{E}_\beta}^{\cc{D}}$. We will prove that for every $X \in \cc{E}_\alpha$ the function $hom_\varphi(X,C) \mr{c_*} hom_\varphi(X,G)$ is a bijection. Take $X \mr{r} X_*$ cocartesian over $\varphi$. We have the following diagram.

\begin{align*}
\xymatrix{hom_\varphi(X,C) \ar[r]^{c_*} & hom_\varphi(X,G) \ar@{}[dl]|\equiv \\
		  hom_\beta(X_*,C) \ar[r]_{c_*} \ar[u]^{r^*} & hom_\beta(X_*,G) \ar[u]_{r^*}}
\end{align*}

\noindent The bottom arrow and the vertical arrows are bijections. The result follows.

 \end{proof}

\begin{definition}
We will say that \textbf{finite limits are stable} if terminal objects and pullbacks are stable.
\end{definition}
}
}
\end{comment}
%%%%%%%%%%%%%%%%%%%%%%%%%%% END %%%%%%%%%%%%%%%%%%%%%%%%%%%%%%%%%%%%%%%

\begin{definition}
A prefibration ${\cc{E} \mr{F} \cc{G}}$  is \textbf{finitely complete} if the categories $\cc{E}_\alpha$ are finitely complete and finite limits are stable.
\end{definition}

\noindent Accordingly we have the category of finitely complete prefibrations whose morphisms are cartesian $\cc{G}$-functors $f \in hom_{\cc{G}}(\cc{E},\cc{E}')$ such that for every $\alpha \in \cc{G}$ the restrictions 

\begin{align*}
\cc{E}_\alpha \mr{f_\alpha} \cc{E}'_\alpha
\end{align*}

\noindent preserve finite limits. 

\begin{definition}
A prefibration ${\cc{E} \mr{F} \cc{G}}$ is  \textbf{regular} if the categories $\cc{E}_\alpha$ are regular, finite limits are stable and strict epimorphisms are stable.
\end{definition}

The category of regular prefibrations  over $\cc{G}$ is the category that has regular prefibrations (fibrations) as objects and whose morphisms are cartesian $\cc{G}$-functors $f \in hom_{\cc{G}}(\cc{E},\cc{E}')$ such that for every $\alpha \in \cc{G}$ the restrictions 

\begin{align*}
\cc{E}_\alpha \mr{f_\alpha} \cc{E}'_\alpha
\end{align*}

\noindent are regular functors. This category is included in the category if finitely complete prefibrations. The restrictions to fibrations are natural and coherent. 
\rojo{que significa esta ultima frase}

{\erojo  Este rojo es para eliminar y/o modificar
%
\subsubsection{Reflection properties in a prefibration}
Properties of functors such as reflecting limits and other types of categorical objects can be defined in a prefibration. Here we define without using clivages the property corresponding to the fact that the pullback functors are conservative over the specific set of morphisms as defined in section \ref{families}.  

%\begin{definition} \label{estabilidad}
 We say that a subset $\cc{A} \subset \cc{E}$ of vertical arrows is 
 \emph{stable} if for any $X \mr{m} Y \in \cc{A}$ and pullback $X^* \mr{m^*} Y^*$, it follows that $m^* \in \cc{A}$
% \end{definition}

Consider $\cc{A}$ a \textit{stable} set of vertical arrows. We will use the notation  $\cc{A}_\alpha$ to represent the subset $\cc{A} \cap \cc{E}_\alpha$. 
%We will work freely identifying vertical arrows in $F$ with objects in $F^{\textbf{2}}$.

%{\erojo \begin{definition}
%\cc{A}$} if for every cartesian morphism $f^* \mr{} f$ such that 
%\mbox{$f \in \cc{A}$} and $f^*$ is an isomorphism, it follows that $f$ is %an isomorphism.
%\end{definition}}

\begin{definition}
A prefibration ${\cc{E} \mr{F} \cc{G}}$ is \textbf{conservative over 
$\cc{A}$} if for every \mbox{$X \mr{f} Y \, \in \cc{A}$} such that 
$X^* \mr{f^*} Y^*$ is an isomorphism, it follows that $f$ is an isomorphism.
\end{definition}

 We will say the prefibration ${\cc{E} \mr{F} \cc{G}}$ is \textit{conservative} if it is conservative over the complete set of arrows of $\cc{E}$. 

{\noindent \erojo Para que definir estable y considerar que $\cc{A}$ es estable en la definicion de conservative ?}
}

\subsection{Two facts about epimorphisms in a prefibration}

\begin{lemma}\label{casi epis}
The functors $j_\alpha$ preserve epimorphisms and strict epimorphisms. That is, epimorphisms and strict epimorphisms in $\cc{E}_\alpha$ remain so in 
$\cc{E}$.
\end{lemma}
\begin{proof}
Let $A \mr{f} B$ over $\alpha$,  consider morphisms $B \mr{} C$, $C$ over 
$\beta$, they sit over  $\alpha \mr{\varphi} \beta$. Take  
 $C^* \mr{} C$ cartesian over $\varphi$. We have the following diagram, where the vertical arrows are bijections:
$$
\xymatrix{hom_\varphi(B,C) \ar[r]^{f^*} \ar@{}[dr]|\equiv & hom_\varphi(A,C) \\
		  hom_\alpha(B,C^*) \ar[u] \ar[r]_{f^*} & hom_\alpha(A,C^*) \ar[u]}
$$
Assuming that the bottom is injective, it follows that so it is the top.

\vspace{1ex}

For the second statement, let ${A \mr{g} C}$ be compatible with $f$. Take 
${C^* \mr{} C}$ cartesian over ${F(g) = \alpha \mr{\varphi} \beta}$. It is straightforward that the unique factorization of $g$ through 
${C^* \mr{} C}$ over $\alpha$ is compatible with $f$. The result follows: In a diagram:

$$
\xymatrix 
    {
   & A \ar [dl]_{f}  
       \ar [ddr] ^{g}  
       \ar  @{-->} [dd]^{\exists ! a}  
   \\
     B \ar @{-->} [dr]_{\exists ! b}  
	&  \ar @{}[l] |(.4){\equiv} 
	& 
   \\
    & C^* \ar[r]_{s}  
          \ar@{}[ru] | (.3){\equiv} 
    & C  
   \\
	  \alpha \ar [r]^{id_\alpha}  
	& \alpha \ar [r]^{\varphi}  
	&  \beta 
	}
$$
\end{proof}

%%%%%%%%%%%%%%%%%%%%%%%%%%%%%%%%%%%%%%%%%%%%%%%%%%%%%%%%%%%%%%%%%%%%%%%
%%%%%%%%%%%%%%%%%%%%%%%%%%%%%%% COMMENT %%%%%%%%%%%%%%%%%%%%%%%%%%%%%%%
\begin{comment}
{\erojo
\begin{lemma}\label{casi epis}
If $A \mr{f} B$ is an epimorphism in $\cc{E}_\alpha$ and $g,h \in hom_\varphi(B,C)$ satisfy $gf=hf$, then $g=h$.
\end{lemma}

\begin{proof}
Take $C^* \mr{s} C$ a cartesian arrow over $\varphi$. We have the following diagram.

\[
\xymatrix{hom_\varphi(B,C) \ar[r]^{f^*} \ar@{}[dr]|\equiv & hom_\varphi(A,C) \\
		  hom_\alpha(B,C^*) \ar[u]^{s_*} \ar[r]_{f^*} & hom_\alpha(A,C^*) \ar[u]_{s_*}}
\]

\noindent Since the vertical arrows are bijections and the bottom is injective, the result follows.
\end{proof}


\begin{lemma}\label{casi strict epi}
If $A \mr{f} B$ is a strict epimorphism in $\cc{E}_\alpha$, then every compatible morphism with $f$ in $\cc{E}$ factors through $f$.
\end{lemma}

\begin{proof}
Let ${A \mr{g} C}$ be compatible with $f$. Take ${C^* \mr{s} C}$ cartesian over ${F(g)=\alpha \mr{\varphi} \beta}$. It is straightforward that the only factorization of $g$ through $s$ over $\alpha$ is compatible with $f$. In a diagram:

\begin{align*}
\xymatrix {& A  \ar [dl]  _{f}  \ar [ddr] ^{g}  \ar  @{-->} [dd] ^{\exists ! a}  \\
		   B  \ar @{-->} [dr] _{\exists ! b}  &   \ar @{} [l] |(.4){\equiv} & \\
		   & C^*  \ar [r] _{s}  \ar @{} [ru] |(.3){\equiv}  &  C  \\
		   \alpha  \ar [r] ^{id_\alpha}  & \alpha  \ar [r] ^{\varphi}  &  \beta }
\end{align*}

\end{proof}
}
\end{comment}
%%%%%%%%%%%%%%%%%%%%%%%%%%%%%%% END %%%%%%%%%%%%%%%%%%%%%%%%%%%%%%%%%%%%
%%%%%%%%%%%%%%%%%%%%%%%%%%%%%%%%%%%%%%%%%%%%%%%%%%%%%%%%%%%%%%%%%%%%%%%%

\section{COLIMIT OF A FIBRATION WITH COFILTERED BASE}
In this section we will study the structure of the colimit of a fibration  developed in \cite{sga4} for the particular case where the base category is cofiltered.

Let ${\cc{E} \mr{F} \cc{G}}$ be a fibration where $\cc{G}$ is cofiltered.  If $S$ denotes the set of cartesian morphisms in $\cc{E}$, the \textit{colimit} of the fibration is defined as the Gabriel Zisman category of fractions 
$\cc{E}[S^{-1}]$, \cite{gabzis}, characterized by a functor 
${\cc{E} \mr{Q} \cc{E}[S^{-1}]}$  that satisfies the following universal property in $\cc{C}at$. 

\begin{definition} \label{fractions}
For every functor ${\cc{E} \mr{G} \cc{X}}$ such that $G$ sends cartesian morphisms into isomorphisms, there exists a unique functor 
${\cc{E}[S^{-1}] \mr{H} \cc{X}}$ such that \mbox{$HQ=G$.}
$$
\xymatrix@R=3ex
    {
     \cc{E} \ar[dr]_{\forall G} 
            \ar[rr]^{Q} 
  && \cc{E}[S^{-1}] \ar[ld]^{\exists ! H}
 \\ 
   & \cc{X} \ar@{}[u]|(.6)\equiv  
    }		
$$
 \end{definition}

\begin{definitionst} \label{gzfractions}
Since $\cc{G}$ is cofiltered $S$ admits a calculus of right fractions  \cite{sga4}. For internal references we recall the Gabriel-Zisman construction of  $\cc{E}[S^{-1}]$ in the presence of a calculus of right fractions. 

\begin{enumerate}

\item 
\emph{Objects}: A \emph{object} is simply a object of $\cc{E}$. 
 
\item
 \emph{Premorphisms}: 
 A \emph{premorphism} $X \mr{} Y$ is a spam $X \ml{s} A \mr{f} Y$, with 
 $s \in S$. In case that there is no risk of confusion we will simply write $(f,s)$.
 
\item  \label{gzCompo}
\emph{Composition}: 
 For $X \xr{(f_1,s_1)} Y \xr{(f_2,s_2)} Z$, 
 having a calculus
of right fractions guarantees that there is a pair 
 $A_1 \ml{s_{21}} A_{21} \mr{f_{21}} A_2$, $s \in S$, such that
 $f_1 s_{21} = s_2 f_{21}$,  $s_1 s_{21} \in S$. This is visualized in the commutative diagram: 
$$
\xymatrix@R=2ex
    {
     && A_{21} \ar[dl]_{s_{21}} 
          \ar[dr]^{f_{21}} 
    \\
	  & A_1 \ar[dl]_{s_1}
	        \ar[dr]^{f_1} 
	 && A_2 \ar[dl]_{s_2} 
	        \ar[dr]^{f_2} 
	   & 
	  \\
		 X 
	  && Y 
	  && Z}
$$
We define a \emph{composition} by 
$(f_2, s_2) \circ (f_1, s_1) = (f_2 f_{21}, s_1 s_{21})$. 

\item \label{gzMorph}
\emph{Morphisms}: A morphism is an equivalence class of premorphisms under the following equivalence: 
$(f_1,s_1)$, $(f_2,s_2)$ are equivalent if there exists 
 $A \mr{a_1} A_1$, $A \mr{a_2} A_2$ in $\cc{E}$ such that 
 $s_1 a_1 = s_2 a_2 \,, say \, =  s  \in S$,  and $f_1 a_1 = f_2 a_2$. 
 Note that by basic facts \ref{basicdefinitions},  \ref{cartesiancomposing}. it follows that $a_1$ and $a_2$ are cartesian.
  
 We denote $(f_1,s_1) \sim (f_2,s_2)$. 
 This is visualized in the commutative diagram:
 $$
\xymatrix@R=2.5ex
      {
    && A_1 \ar@/^0.7pc/[dr]_{f_1}
           \ar[ld]_{s_1} | \car
    \\ 
       A \ar[r]^{s} | \car 
         \ar@/^0.8pc/[rru]_{a_1}   | \car
         \ar@/_0.8pc/[rrd]^{a_2}   | \car
     & X %\ar[ru]_{a_1} 
         %\ar[rd]^{a_2}
    && Y
    \\
    && A_2 \ar@/_0.7pc/[ur]^{f_2}
           \ar[ul]^{s_2} | \car           
       }
$$
\end{enumerate}
Equivalence is indeed an equivalence relation and the composition of premorphisms becomes unique and well defined on equivalent classes.

We will denote the class of a premorhism  $(f,s)$ by  ${f/s}$. For $A = X$ we adopt the abuse of notations $f/{id_X} = f$, 
%and $id_A/s = 1/s$,  we have that 
%${({1}/s)  s = id_A}$, $s  ({1}/s) = id_X$, and 
%${f/s = f ({1}/s)}$. 

\vspace{1ex}

The functor $Q$ is defined as $Q(X) = X$, and for $X \mr{f} Y$,  
$Q(f) = f/{id_X} = f$.  

\vspace{1ex}

For verifications and details see \cite{gabzis}[Ch I, 2.].

\end{definitionst}

\begin{remark} \label{Jalphaff_1}
Let  $X \mr{f/s} Y$ be a morphism in $\cc{E}[S^{-1}]$ with $s$ already invertible, then 
$f/s = fs^{-1}/id_X = Q(fs^{-1})$.
\end{remark}\begin{proof}
The reader can easily check the equivalence $(f,s) \sim (fs^{-1}, id_X)$.
\end{proof}
%
From \ref{basicdefinitions}, \ref{cartesianiffiso}, and Remark 
\ref{Jalphaff_1} it follows:
\begin{remark} \label{Jalphafandf}
The functors $J_\alpha$  defined as the composite
$
\xymatrix@R=0ex
     {
      \cc{E}_\alpha  \ar@/_0.5pc/[dr]^{j_\alpha} 
                     \ar[rr]^{J_\alpha} 
   && \cc{E}[S^{-1}]
  \\ 
    & \cc{E} \ar@/_0.5pc/[ur]^<(0.21)Q %\ar@{}[u]|(.6)\equiv  
     }		
$, \mbox{$J_\alpha = Qj_\alpha$}, are fully-faithful. \cqd
\end{remark}
The next remark follows directly by the definition of the composition and of the equivalence relation, Definition 
\ref{gzfractions}, items \ref{gzCompo} and \ref{gzMorph}. 
\begin{remark}[Lifting of triangles] \label{lift_triangles} ${ }$

For $X \xr{(f_1,s_1)} Y \xr{(f_2,s_2)} Z$, 
$X \xr{(f_3,s_3)} Z$, 
and the equation $f_3/s_3 = f_2/s_2 \circ f_1/s_1$ in $\cc{E}[S^{-1}]$,
there is a diagram in $\cc{E}$  witnessing this equation:
$$
\xymatrix@R=2.5ex
     {
  &&& A_{21} \ar[ld]_{s_{21}} | \car
        \ar[rd]^{f_{21}}
   \\ 
   && A_1 \ar[ld]_{s_1} | \car
        \ar[rd]^{f_1}
   && A_2 \ar[ld]_{s_2} | \car
        \ar[rd]^{f_2}
    \\
      A \ar@/^1.2pc/[rrruu]^{a_1}  % | \car    %\ar@/^0.7pc/[dr]^{f_1}
        \ar@/_1.2pc/[rrrd]^{a_2}   % | \car     %\ar@/_0.7pc/[ur]^{f_2}
        \ar[r]^{s} | \car
    & X
   && Y
   && Z
     \\
  &&& A_3 \ar[llu]_{s_3} | \car
        \ar[rru]^{f_3} 
     }
$$
Vice-versa, any such diagram in $\cc{E}$ corresponds to a commutative triangle in $\cc{E}[S^{-1}]$.
\cqd
\end{remark}



 

%%%%%%%%%%%%%%%%%%%%%%%%%%%%%%%%%%%%%%%%%%%%%%%%%%%%%%%%%%%%%%%%%%%%%
%%%%%%%%%%%%%%%%%%%%%%%% COMENTARIO %%%%%%%%%%%%%%%%%%%%%%%%%%%%%%%%%
\begin{comment}
{
\erojo
En este diagrama se "levanta" una composicion (triangulo commutativo) 
\mbox{$f_3/s_3 = f_2/s_2 \circ f_1/s_1$} en las fracciones en un diagrama en $\cc{E}$ (por definicion de composicion y de la relacion de equivalencia).

Esto con la esperanza de 
"levantar" cualquier diagrama finito en las fracciones en un diagrama en 
$\cc{E}$. Pero todavia no me esta saliendo. Si termina saliendo permitiria demostrar la proposicion \ref{pullbacks se traen a las fibras} para un diagrama finito cualquiera.
}

{\erojo 
$$
\xymatrix@R=2ex
    {
    && X
   \\
       A \ar[rru]^{s}
           \ar[rrd]_{f}
           \ar[r]
    &  A_1 \ar[ru]_{s_1}
         \ar[rd]^{f_1}
    \\
   &&  Y
    }
\hspace{8ex}
\xymatrix@R=2ex
    {
    && X
   \\
       A \ar[rru]^{s}
           \ar[rrd]_{f}
           \ar[r]
    &  A_2 \ar[ru]_{s_2}
         \ar[rd]^{f_2}
    \\
   &&  Y
    }
$$

$$ 
\xymatrix@+3ex 
    {
     & A \ar[dl]_s 
         \ar[dr]^f 
     & 
   \\
       X 
	  & A'' \ar[l]_{s''} 
            \ar[r]^{f''} 
	        \ar[u] \ar[d] 
	 & Y 
   \\  
	 & A' \ar[ul]^{s'} 
	      \ar[ur]_{f'} 
	 &
	}
$$
}
For ${X \mr{f/s} Y \mr{g/t} Z}$ having a calculus of right fractions guarantees that there is a diagram as follows, with $u,\, su \in S$:
% 
$$
\xymatrix@R=0ex
    {
     && C \ar[dl]_u 
          \ar[dr]^h 
      && 
    \\
	  & A \ar[dl]_s 
	      \ar[dr]_f 
	  && B \ar[dl]^t 
	       \ar[dr]^g 
	   & 
	  \\
		 X 
	  && Y 
	  && Z}
$$ 
%
The operation $(g/t)(f/s):=(gh)/(su)$ is well defined, which defines the composition. 

Adopting the abuse of notations $f/{id_A} = f$ and $id_A/s = 1/s$,  we have that 
${({1}/s)  s = id_A}$, $s  ({1}/s) = id_X$, and 
${f/s = f ({1}/s)}$.  For verifications and details see \cite{gabzis}
\end{comment}
%%%%%%%%%%%%%%%%%%%%%%%%%% FIN COMENTARIO %%%%%%%%%%%%%%%%%%%%%%%%%%%
%%%%%%%%%%%%%%%%%%%%%%%%%%%%%%%%%%%%%%%%%%%%%%%%%%%%%%%%%%%%%%%%%%%%%


%%%%%%%%%%%%%%%%%%%%%%%%%%%%%%%%%%%%%%%%%%%%%%%%%%%%%%%%%%%%%%%%%%%%%%
%%%%%%%%%%%%%%%%%%%%%%%%%% COMENTARIO %%%%%%%%%%%%%%%%%%%%%%%%%%%%%%%%
\begin{comment}
\begin{remark}
Because $\cc{G}$, is cofiltered $S$ admits a calculus of right fractions  \cite{sga4}, \cite{tesemi}. Thus we can describe the category of fractions $\cc{E}[S^{-1}]$  as done in \cite{gabzis} as follows. The objects of $\cc{E}[S^{-1}]$ are the objects of $\cc{E}$. A morphism ${X \mr{} Y}$ in $\cc{E}[S^{-1}]$ is an equivalence class of the quotient set of the set of pairs  $X \ml{s} A \mr{f} Y$ with $s \in S$ where the equivalence relation is given by the relation ${X \ml{s} A \mr{f} Y \sim X \ml{s'} A' \mr{f'} Y}$ if and only if there exists ${X \ml{s''} A'' \mr{f''} Y}$ with $s'' \in S$ and arrows $A'' \mr{} A$ and $A'' \mr{} A'$ in $\cc{E}$ such that the following diagram commutes. 

\[
\xymatrix @+3ex {& A \ar[dl]_s \ar[dr]^f & \\
		  X & A'' \ar[l]_{s''} \ar[r]^{f''} \ar[u] \ar[d] & Y \\
		  & A' \ar[ul]^{s'} \ar[ur]_{f'} &}
\]

\noindent We will denote the class of the pair ${X \ml{s} A \mr{f} Y}$ by  ${X \mr{f/s} Y} $. For ${X \mr{f/s} Y \mr{g/t} Z}$ having a calculus of right fractions guarantees that there is a pair $A \ml{u} C \mr{h} B$ with $u \in S$ such that $su \in S$ and the following diagram commutes.
 
\[
\xymatrix{&& C \ar[dl]_u \ar[dr]^h && \\
		  & A \ar[dl]_s \ar[dr]^f && B \ar[dl]_t \ar[dr]^g & \\
		  X && Y && Z}
\] 
 
\noindent The operation $(g/t)(f/s):=(gh)/(su)$ is well defined, which defines the composition. The functor $Q$ is defined as the identity on objects and for ${A \mr{f} Y \in \cc{E}}$ we have $Q(f)=f/{1_A}$. For ${A \mr{s} X \in S}$, and adopting the abuse of notations $f/{1_A}=f$ and $1_A/s=1/s$,  we have that ${({1}/s) \cdot s=1_A}$, $s \cdot ({1}/s)=1_X$ and ${f/s=f\cdot ({1}/s)}$.  For details see \cite{gabzis}.
\end{remark}
\end{comment}
%%%%%%%%%%%%%%%%%%%%%%%%%% END COMMENT %%%%%%%%%%%%%%%%%%%%%%%%%%%%%
%%%%%%%%%%%%%%%%%%%%%%%%%%%%%%%%%%%%%%%%%%%%%%%%%%%%%%%%%%%%%%%%%%%%

\subsection{Colimit of a finitely complete fibration}       
Our objective in this section is to prove the following theorem, which we do in \ref{prueba} below.

\begin{theorem}\label{colimit of finite complete fibrations is finite complete}
If $\cc{E} \mr{F} \cc{G}$ is finitely complete, then $\cc{E}[S^{-1}]$ is a finitely complete category and the functors $J_\alpha = Q \, j_\alpha$
%
$$
\xymatrix@R=0ex
     {
      \cc{E}_\alpha  \ar@/_0.5pc/[dr]^{j_\alpha} 
                     \ar[rr]^{J_\alpha} 
   && \cc{E}[S^{-1}]
  \\ 
    & \cc{E} \ar@/_0.5pc/[ur]^<(0.21)Q %\ar@{}[u]|(.6)\equiv  
     }		
$$
%
preserve finite limits. Moreover for $\ \cc{X} \in \cc{C}at_{fl}$  and 
${\cc{E}[S^{-1}] \mr{H} \cc{X}}$, $H$ preserves finite limits if and only if  for every ${\alpha \in \cc{G}}$ the functors 
$H J_\alpha$ preserves finite limits. \cqd
\end{theorem}
% 
%%%%%%%%%%%%%%%%%%%%%%%%%%%%%%%%%%%%%%%%%%%%%%%%%%%%%%%%%%%%%%%%%%%%
%%%%%%%%%%%%%%%%%%%%%%% COMENTARIO %%%%%%%%%%%%%%%%%%%%%%%%%%%%%%%%%
\begin{comment}
$$
\xymatrix@+2ex@1 
    {
  & \cc{E}[S^{-1}] \ar[r]^H 
                    \ar@{}[dr]|(.3)\equiv & \cc{I} 
 \\
   & \cc{E} \ar@{}[l]|(.35)\equiv 
            \ar[u]^Q \ar[ur]_G 
    & 
  \\
    & \cc{E}_\alpha 
             \ar@/^3pc/[uu]^{J_\alpha} 
             \ar[u]_{j_\alpha}
     } 
$$
\end{comment}
%%%%%%%%%%%%%%%%%%%%%%%%%%% FIN COMENTARIO %%%%%%%%%%%%%%%%%%%%%%%%%
%%%%%%%%%%%%%%%%%%%%%%%%%%%%%%%%%%%%%%%%%%%%%%%%%%%%%%%%%%%%%%%%%%%%
%
\begin{corollary} \label{buena restriccion a finite complete}
If $\ \cc{X} \in \cc{C}at_{fl}$ and ${\cc{E} \mr{G} \cc{X}}$ is such that $G$ sends cartesian morphisms into isomorphisms, and for every 
${\alpha \in \cc{G}}$ the functors $G \, j_\alpha \in \cc{C}at_{fl}$, then there exists a unique functor ${\cc{E}[S^{-1}] \mr{H} \cc{X}} \in \cc{C}at_{fl}$ such that $H Q = G$.
\end{corollary}
\begin{proof}
Consider the diagram
$$
\xymatrix@R=3ex
    {
     \cc{E}_\alpha \ar@/^2pc/[rrr]^{J_\alpha}
                   \ar[r]^{j_\alpha}
   & \cc{E} \ar[rr]^Q
            \ar[dr]_G
  && \cc{E}[S^{-1}] \ar[dl]^H
 \\
  && \cc{X}
     }
$$
By Definition \ref{fractions} it only remains to see that $H$ preserves finite limits, which follows by Theorem \ref{colimit of finite complete fibrations is finite complete}.
\end{proof}

 
\begin{corollary}\label{funtorialidad}
The assignment $\cc{E} \mr{F} \cc{G} \; \mapsto  \; \cc{E}[S^{-1}]$     determines a functor from the category of finitely complete fibrations into $\cc{C}at_{fl}$.
\end{corollary}
\begin{proof}
Let 
$\xymatrix@R=2ex{ \cc{E} \ar[rr]^f \ar[dr]_{F} && \cc{E'} \ar[dl]^{F'}
\\ & \cc{G}}$
be a morphism of finitely complete fibrations. Consider the  diagram
$$
\xymatrix
     {
      \cc{E}[S^{-1}]  \ar@{-->}[r]^{\exists !} 
    & \cc{E'}[{S'}^{-1}] 
  \\
	  \cc{E} \ar[u]^Q \ar[r]^f 
	& \cc{E'} \ar[u]^{Q'} 
	& 
  \\
	  \cc{E}_\alpha \ar[r]_{f_\alpha} 
	                \ar[u]^{j_\alpha} 
	& \cc{E'}_{\alpha} \ar[u]^{j'_{\alpha}} 
	                    \ar@/_2pc/[uu]_{J'_{\alpha}} 
	 }
$$
The functor $Q' f$ sends cartesian morphisms into isomorphisma, and for every
${\alpha \in \cc{G}}$ the functors 
$Q'f j_\alpha = J'_{\alpha} f_\alpha$ preserve finite limits.
The result follows.
\end{proof}
 %
 \begin{sinnadastandard}[{\bf Proof of theorem 
 \ref{colimit of finite complete fibrations is finite complete}}]
 \label{prueba}

\end{sinnadastandard}
%{\bf Proof of Theorem:}  
%\ref{colimit of finite complete fibrations is finite complete}
\begin{proposition}\label{fibras preservan terminal object}
The functors $\cc{E}_\alpha \mr{J_\alpha} \cc{E}[S^{-1}]$ preserve terminal objects, in particular terminal objects exist in 
$\cc{E}[S^{-1}]$.
\end{proposition}
\begin{proof}
 Let $\alpha \in \cc{G}$, $1_\alpha$ a terminal object in $\cc{E}_\alpha$, and $X \in \cc{E}$ over some $\beta \in \cc{G}$. With the following diagram we show the existence of an arrow $X \mr{} 1_\alpha$ in $\cc{E}[S^{-1}]$ (recall that terminal objects are stable $1_\alpha^* = 1_\gamma$).  
$$
\xymatrix
    {
   & X^* \ar@/_0.7pc/[ld]_s
         \ar@{-->}[d]
  \\
     X
   & 1_\alpha^* \ar[r]^t
   & 1_\alpha
  \\
   \beta 
   & \gamma \ar@{-->}[l]_\mu
            \ar@{-->}[r]^\nu
   & \alpha
    }
$$
Take $\mu, \; \nu$ as indicated, and $s$ cartesian over $\mu$, $t$ cartesian over $\nu$. 

\vspace{1ex}
  
Now suppose we have $X \mrpair{f_1/s_1}{f_2/s_2} 1_\alpha$ in $\cc{E}[S^{-1}]$. With the following diagrams we show that they are equivalent, 
$(f_1,s_1) \sim (f_2,s_2)$. We refer to \ref{gzfractions}.
 

%%%%%%%%%%%%%%%%%%%%%%%%%%%%%%%%%%%%%%%%%%%%%%%%%%%%
$$
 \xymatrix@R=2.5ex
      {
    && A_1 \ar@/^0.7pc/[dr]_{f_1}
           \ar[ld]_{s_1} | \car
    \\ 
  %     X^* \ar[r]^{s} | \car 
  %        \ar@/^0.8pc/[rru]_{a_1} %| \car
  %       \ar@/_0.8pc/[rrd]^{a_2} %| \car
     & X %\ar[ru]_{a_1} 
         %\ar[rd]^{a_2}
    && 1_\alpha
    \\
    && A_2 \ar@/_0.7pc/[ur]^{f_2}
           \ar[ul]^{s_2} | \car           
       }
\hspace{6ex}       
\xymatrix@R=2.5ex
      {
    && A_1 \ar@/^0.7pc/[dr]_{f_1}
           \ar[ld]_{s_1} | \car
    \\ 
       X^* \ar[r]^{s} | \car 
         \ar@/^0.8pc/[rru]_{a_1} %| \car
         \ar@/_0.8pc/[rrd]^{a_2} %| \car
     & X %\ar[ru]_{a_1} 
         %\ar[rd]^{a_2}
    && 1_\alpha
    \\
    && A_2 \ar@/_0.7pc/[ur]^{f_2}
           \ar[ul]^{s_2} | \car           
       }
$$
%%%%%%%%%%%%%%%%%%%%%%%%%%%%%%%%%%%%%%%%%%%%%%%%%%%%%
$$
 \xymatrix@R=2ex
      {
%    && A_1 \ar@/^0.7pc/[dr]^{f_1}
%           \ar@/_1pc/[dll]_{s_1}
%    \\
%       X
%   &&& 1_\alpha
%    \\
%    && A_2 \ar@/_0.7pc/[ur]^{f_2}
%           \ar@/^1pc/[ull]_{s_2}
%    \\ 
    && \gamma_1 \ar@/^0.7pc/[dr]_{\nu_1}
           \ar[ld]_{\mu_1} % | \car
    \\ 
  %     X^* \ar[r]^{s} | \car 
  %        \ar@/^0.8pc/[rru]_{a_1} %| \car
  %       \ar@/_0.8pc/[rrd]^{a_2} %| \car
     & \beta %\ar[ru]_{a_1} 
         %\ar[rd]^{a_2}
    && \alpha
    \\
    && \gamma_2 \ar@/_0.7pc/[ur]^{\nu_2}
           \ar[ul]^{\mu_2} % | \car           
       }
\hspace{5ex}
\xymatrix@R=2ex
      {
%    && A_1 \ar@/^0.7pc/[dr]^{f_1} 
%           \ar@/_1pc/[dll]_{s_1}
%    \\
%       X
%     & X^* \ar@{-->}[rd]^{a_2}
%           \ar@{-->}[ru]_{a_1}
%           \ar[l]^s
%    && 1_\alpha
%   \\
%    && A_2 \ar@/_0.7pc/[ur]^{f_2}
%           \ar@/^1pc/[ull]_{s_2}
%    \\
    && \gamma_1 \ar@/^0.7pc/[dr]_{\nu_1}
           \ar[ld]_{\mu_1} %| \car
    \\ 
       \gamma \ar@{-->}[r]^{\mu} %| \car 
         \ar@{-->}@/^0.8pc/[rru]_{\rho_1} %| \car
         \ar@{-->}@/_0.8pc/[rrd]^{\rho_2} %| \car
     & \beta %\ar[ru]_{a_1} 
         %\ar[rd]^{a_2}
    && \alpha
    \\
    && \gamma_2 \ar@/_0.7pc/[ur]^{\nu_2}
           \ar[ul]^{\mu_2} %| \car           
       }
 $$
%\end{proof}    

%%%%%%%%%%%%%%%%%%%%%%%%%%%%%%%%%%%%%%%%%%%%%%%%%%%%
We have the two diagrams on the left. Take a cone $\gamma$ as indicated in the bottom right diagram. Let $s$ be cartesian over $\mu$. Since 
$s_1,\; s_2$ are cartesian, there are  $a_1,\;a_2$
 uniques over $\rho_1,\; \rho_2$ respectively that factor $s$ through 
$s_1,\; s_2$ respectively. It only remains to check  $f_1 a_1 = f_2 a_2$, which we see as follows: Let $\nu := \nu_1 \rho_1 = \nu_2 \rho_2$, 
and consider $1_\alpha^* \mr{} 1_\alpha$ over $\nu$. Then 
$f_1 a_1,\; f_2 a_2$ would be equal if the corresponding unique factorizations through $1_\alpha^*$ are equal in $\cc{E}_\gamma$, which is clear since  
$1_\alpha^* = 1_\gamma$.
\end{proof}

%%%%%%%%%%%%%%%%%%%%%%%%%%%%%%%%%%%%%%%%%%%%%%%%%%%%%%%%%%%%%%%%%%%
%%%%%%%%%%%%%%%%%%%%% COMENTARIO %%%%%%%%%%%%%%%%%%%%%%%%%%%%%%%%%%
\begin{comment}
Take a cone of the following diagram in $\cc{G}$.
 
 \[
\xymatrix  @+3ex{  & F(X) & \\
		  {} \ar [ur] ^{F(s_0)} \ar [dr] _{F(f_0)} & \beta  \ar @{-->} [u] _{\varphi_0} \ar @{-->} [d] _{\varphi_1} \ar @{-->} [l] _{\psi_0} \ar @{-->} [r] _{\psi_1} & {} \ar[ul]_{F(s_1)} \ar[dl]^{F(f_1)}\\
		    & \alpha &} 
 \] 
 
\noindent Let $X^* \mr{s} X$ be a cartesian morphism over $\varphi_0$. Take ($i=0,1$) $a_i$ the unique morphism over $\psi_i$ that factors $s$ through $s_i$. It follows that $f_0a_0=f_1a_1$ in $\cc{E}$ (\ref{terminal sobre phi}). Thus precomposing with $1/s$ we obtain $f_0/s_0=f_1/s_1$.
\end{comment} 
%%%%%%%%%%%%%%%%%%%%%%%% FIN COMENTARIO %%%%%%%%%%%%%%%%%%%%%%%%%%%
%%%%%%%%%%%%%%%%%%%%%%%%%%%%%%%%%%%%%%%%%%%%%%%%%%%%%%%%%%%%%%%%%%% 
%%%%%%%%%%%%%%%%%%%%%%%% COMENTARIO %%%%%%%%%%%%%%%%%%%%%%%%%%%%%%%
\begin{comment}
\begin{nobservation}
Notice that the data of a functor $F$ from a category $\cc{A}$ to a category $\cc{B}$ can be seen as a function $\cc{A} \mr{F} \cc{B}$ respecting the domain and codomain operations, together with a disjoint union of commutative triangles $F(h) = F(g)F(f)$ in $\cc{B}$, one for each commutative triangle $h = gf$ in $\cc{A}$.
\end{nobservation}
\end{comment}
%%%%%%%%%%%%%%%%%%%%%%%%% FIN DEL COMENTARIO %%%%%%%%%%%%%%%%%%%%%%
\begin{proposition}\label{pullbacks se traen a las fibras}
Any diagram ${\cc{D} \mr{X} \cc{E}[S^{-1}]}$ in  $\cc{E}[S^{-1}]$ is naturally isomorphic to one that can be factored through a fibre. More precisely, there exists ${\alpha \in \cc{G}}$, $X^*$ and $\eta$
$$
\xymatrix
     {
      & \cc{D} \ar@{-->}[dl]_{X^*} 
               \ar@{}[d]|(.6){ \overset{\eta}{\implies}} 
               \ar[dr]^X 
      & 
    \\
        \cc{E}_\alpha \ar[rr]_{J_\alpha} 
     && \cc{E}[S^{-1}]
      }
$$
where the natural transformation is composed of cartesian arrows. In particular, since cartesian arrows are invertible, there is a natural isomorphism between $X$ and $X^*$ in $\cc{E}[S^{-1}]$ (we omit $J_\alpha$ and write $J_\alpha X^* = X^*$). 
\end{proposition}
\begin{proof}
For every $i \mr{f} j$ in $\cc{D}$, take a premorphism  
$L_f = (X_i \ml{s_f} A_f \mr{p_f} X_j)$ for each 
$X_i \mr{X_f} X_j$, $X_f = p_f/s_f$. 
Having done these choices, we obtain a function $\cc{D} \mr{L} \cc{E}$, \mbox{$L_i = X_i$,} and $L_f$ in place of $X_f$. 
Then for every 
composable pair $i \mr{f} j \mr{g} k$, $h = gf$, take a witnessing
diagram $L_{g,f}$ as in \ref{lift_triangles} for the equation 
\mbox{$X_{h} = X_g X_f$, $p_{h}/s_{h} = p_g/s_g \circ p_f/s_f$.}
$$
\xymatrix@R=2.5ex
     {
  &&& B_{h} \ar[ld]_{t_{h}} | \car
             \ar[rd]^{q_{h}}
   \\ 
   && A_f \ar[ld]_{s_f} | \car
          \ar[rd]^{p_f}
   && A_g \ar[ld]_{s_g} | \car
          \ar[rd]^{p_g}
    \\
      C_{h} \ar@/^1.3pc/[rrruu]^{b_{h}} | \car  
             \ar@/_1.3pc/[rrrd]^{a_{h}} | \car       
             \ar[r]^{c_h} | \car
    & X_i
   && X_j
   && X_k
     \\
  &&& A_{h} \ar[llu]_{s_{h}} | \car
        \ar[rru]^{p_{h}} 
     }
$$
All these  data can be seen as a diagram 
$\cc{L} = \{L_i,\, L_f,\, L_{g,f}\}$ in $\cc{E}$, which sits over the diagram $F\cc{L}$ in $\cc{G}$. Since $\cc{G}$ is cofiltered we can take a cone with vertex $\alpha$ for $F\cc{L}$, and pull back  $\cc{L}$ along this cone as in \ref{pbk2}, \ref{hdpullback}. In this way we obtain a diagram 
$\cc{L}^* = \{L_i^*,\, L_f^*,\, L_{g,f}^*\}$
 in the fiber $\cc{E}_\alpha$ and a natural transformation 
 $j_\alpha \cc{L}^* \Mr{\eta} \cc{L}$ whose components are cartesian arrows in $\cc{E}$. The data in $\cc{L}^*$ determines a diagram 
$\cc{D} \mr{X^*} \cc{E}[S^{-1}]$ with all the vertices $Q(X_i) = X_i$ in 
$\cc{E}_\alpha$, then Remark \ref{Jalphafandf} finishes the proof.
 \end{proof}
 %
\begin{proposition} \label{fibras preservan pull-back}
For any finite non empty category $\cc{D}$, the functors 
$\cc{E}_\alpha \mr{J_\alpha} \cc{E}[S^{-1}]$ preserve limits of type      
$\cc{D}$, and limits of type $\cc{D}$ (in particular pull-backs) exist in 
$\cc{E}[S^{-1}]$.    
\end{proposition}
\begin{proof}
The first assertion is an immediate consequence of Proposition 
\ref{stable=preserve} and the fact that $\cc{E} \mr{Q} \cc{E}[S^{-1}]$ preserves finite limits 
\cite{gabzis}. \rojo{ver GZ seccion 3.6 p 18.} 

Given $\cc{D}  \mr{X} \cc{E}[S^{-1}]$, we consider $\alpha$ and  
$\cc{D}  \mr{X^*} \cc{E}_\alpha$ as in Proposition \ref{pullbacks se traen a las fibras}, and take a limit of $X^*$ in $\cc{E}_\alpha$. This remains a limit in $\cc{E}[S^{-1}]$, and since $X^*$ is isomorphic to $X$, it will be a limit of $X$.
\end{proof}
%
\begin{sinnadastandard}[\bf{end of Proof of theorem \ref{colimit of finite complete fibrations is finite complete}}] 
\end{sinnadastandard}   
Clearly, since all finite limits are constructed with terminal objects and pull-backs, propositions \ref{fibras preservan terminal object} and 
\ref{fibras preservan pull-back} finish the proof. \cqd



\subsection{Colimit of a Regular Fibration}

Our objective in this section is to prove the following theorem.


\begin{theorem}\label{colimit of regular fibrations is regular}
If $F$ is a regular fibration, then $\cc{E}[S^{-1}]$ is a regular category and the functors

\[
\xymatrix{\cc{E}_\alpha  \ar[dr]_{j_\alpha} \ar[rr]^{J_\alpha} & & \cc{E}[S^{-1}]\\ 
		       & \cc{E} \ar[ur]_Q \ar@{}[u]|(.6)\equiv  }		
\]
\noindent are regular. More so if $\ \cc{I} \in \cc{R}eg$ and ${\cc{E}[S^{-1}] \mr{H} \cc{I}}$ is a functor such that for every ${\alpha \in \cc{G}}$ the functors $H \cdot J_\alpha \in \cc{R}eg$, it follows that ${H \in \cc{R}eg}$.
\end{theorem}

The proofs of the following two corollaries ar \text{identical} to the proofs of Corollaries \ref{buena restriccion a finite complete} and \ref{funtorialidad}.


\begin{corollary} 
If $\ \cc{I} \in \cc{R}eg$ and ${\cc{E} \mr{G} \cc{I}}$ is such that $G$ transforms cartesian morphisms into isomorphisms and for every ${\alpha \in \cc{G}}$ the functors ${G \cdot j_\alpha \in \cc{R}eg}$, then there exists a unique ${\cc{E}[S^{-1}] \mr{H} \cc{I} \in \cc{R}eg}$ such that $[HQ=G]$.
\end{corollary}


\begin{corollary}
The construction determines a functor from the category of regular fibrations into $\cc{R}eg$.
\end{corollary}

\begin{observation}
If $X \mrpair{f}{g} Y$  are in $\cc{E}$, we have that $f=g$ in $\cc{E}[S^{-1}]$ if and only if there exists $s \in S$ such that $fs=gs$ in $\cc{E}$. 
\end{observation}

\begin{proposition}\label{unicidad de la factorizacion}
The functors $\cc{E}_\alpha \mr{J_\alpha} \cc{E}[S^{-1}]$ send strict epimorphisms to epimorphisms.
\end{proposition}
%
\begin{proof}
Let $Z \mr{f} X$ be a strict epimorphism in 
$\cc{E}_\alpha$, and $X \mrpair{f_1/s_1}{f_2/s_2} Y$ be such that  
$(f_1/s_1) f = (f_2/s_2) f$. We have to show that 
$(f_1/s_1) = (f_2/s_2)$ in $\cc{E}[S^{-1}]$, that is 
$(f_1,s_1) \sim (f_2,s_2)$. 
We consider the diagram obtained in the proof of Proposition 
\ref{fibras preservan terminal object}, and, as in that proposition, we have to see that $f_1 a_1 = f_2 a_2$. It suffices to prove that $f_1 a_1 = f_2  a_2$ in 
$\cc{E}[S^{-1}]$. 

\rojo{no entiendo porque vale que es suficiente}

We expand the diagram as follows: 
$$
\xymatrix@R=2.5ex
      {
       Z^{**} \ar[r]^u | \car
           \ar[dd]^{f^{**}}
     & Z^* \ar[r]^t | \car
           \ar[dd]^{f^*}
     & Z \ar[dd]^(.3){f}
     \\
     &&& A_1 \ar@/^0.7pc/[dr]_{f_1}
           \ar[ld]_{s_1} | \car
     \\
       X^{**} \ar[r]^v | \car
     & X^* \ar[r]^{s} | \car 
         \ar@/^0.8pc/[rru]^(.35){a_1} %| \car
         \ar@/_0.8pc/[rrd]^{a_2} %| \car
     & X
    && Y
    \\
   &&& A_2 \ar@/_0.7pc/[ur]^{f_2}
           \ar[ul]^{s_2} | \car 
    \\
        &&& \gamma_1 \ar@/^0.7pc/[dr]_{\nu_1}
           \ar[ld]_{\mu_1} %| \car
    \\ 
       \gamma \ar[r]^{F(u)}
     & \beta \ar@{-->}[r]^{\varphi_1} %| \car 
         \ar@{-->}@/^0.8pc/[rru]_{\phi_1} %| \car
         \ar@{-->}@/_0.8pc/[rrd]^{\phi_2} %| \car
     &  \alpha %\ar[ru]_{a_1} 
         %\ar[rd]^{a_2}
    && \delta
    \\
    &&& \gamma_2 \ar@/_0.7pc/[ur]^{\nu_2}
           \ar[ul]^{\mu_2} %| \car           
    }
$$
 
%Let $B^* \mr{s} B$ be a cartesian morphism over $\varphi_0$. Take ($i=0,1$) $a_i$ the unique morphism over $\psi_i$ that factors $s$ through $s_i$. 

 Take $Z^* \mr{t} Z$ cartesian over 
$\varphi_1$ and $f^*$ the corresponding pullback of $f$ along $\varphi_1$. 
We have that $(f_1  a_1)f^*=(f_2  a_2)f^*$ in 
$\cc{E}[S^{-1}]$ \rojo{explicar en detalle esto}. Thus there is a cartesian morphism {$Z^{**} \mr{u} Z^*$} such that 
$(f_1  a_1f^*)u = (f_2  a_2f^*)u$ in $\cc{E}$. Take 
${X^{**} \mr{v} X^*}$ cartesian over $F(u)$ and call 
$f^{**}$ the corresponding pullback of $f^*$ along $F(u)$. Since  $(f_1  a_1v)f^{**} = (f_2 a_2v)f^{**}$ in $\cc{E}$ and $F(f_2 a_2v) = F(f_1 a_1v) = \varphi_2F(u)$ \rojo{para que esta ecuacion ?} from Lemma \ref{casi epis} we conclude  $f_1  a_1v = f_2 a_2v$ in 
$\cc{E}$ \rojo{porque strict epis son estables, luego $f^{**}$ es srtrict epi}. The result follows.
\end{proof}

{\erojo 
========================================
========================================

It suffices to prove that $f_1 a_1 = f_2  a_2$ in 
$\cc{E}[S^{-1}]$. 
Take a cone of the following diagram in $\cc{G}$.
I
 \[
\xymatrix @+3ex {  & \alpha & \\
		  {} \ar[ur]^{F(s_0)} \ar[dr]_{F(f_0)} & \beta  \ar@{-->}[u]_{\varphi_0} \ar@{-->}[d]_{\varphi_1} \ar@{-->}[l]_{\psi_0} \ar@{-->}[r]_{\psi_1} & {} \ar[ul]_{F(s_1)} \ar[dl]^{F(f_1)}\\
		    & F(C) &} 
 \] 
\noindent Let $B^* \mr{s} B$ be a cartesian morphism over $\varphi_0$. Take ($i=0,1$) $a_i$ the unique morphism over $\psi_i$ that factors $s$ through $s_i$. It suffices to prove that $f_0 a_0=f_1  a_1$ in $\cc{E}[S^{-1}]$. Take $A^* \mr{t} A$ cartesian over $\varphi_0$ and $f^*$ the corresponding pullback of $f$ along $\varphi_0$. 
 
 
\[
\xymatrix @+2ex {A^{**} \ar@{-->}[r]^u \ar[d]_{f^{**}} \ar@{}[dr]|\equiv& A^* \ar[rr]^t \ar[d]^{f^*} && A \ar[d]^f \ar@{}[dll]|\equiv \\
		  B^{**} \ar@{-->}[r]_v  & B^* \ar[rr]_s \ar[dr]_{a_i} && B \\
		   && {} \ar@{}[u]|\equiv \ar[ur]_{s_i} \ar[drr]^{f_i}& \\
		  &{} & {} && C \\
		  \gamma \ar@{-->}[r]^{F(u)}&\beta  \ar[rr]^{\varphi_0}&& \alpha }
\] 
 
 We have that $(f_0  a_0)f^*=(f_1  a_1)f^*$ in $\cc{E}[S^{-1}]$. Thus there is a cartesian morphism {$A^{**} \mr{u} A^*$} such that $(f_0  a_0f^*)u=(f_1  a_1f^*)u$ in $\cc{E}$. Take ${B^{**} \mr{v} B^*}$ cartesian over $F(u)$ and call $f^{**}$ the corresponding pullback of $f^*$ along $F(u)$. Since  $(f_0  a_0v)f^{**}=(f_1 a_1v)f^{**}$ in $\cc{E}$ and $F(f_1 a_1v)=F(f_0 a_0v)=\varphi_1F(u)$ from Lemma \ref{casi epis} we conclude  $f_0  a_0v=f_1 a_1v$ in $\cc{E}$. The result follows.

 =======================================  
 =======================================
}

\begin{proposition}\label{factorizacion a travez de epis}
If $A \mr{f} B$ is a strict epimorphism in $\cc{E}_\alpha$, then every compatible morphism with $f$ in $\cc{E}[S^{-1}]$ factors through $f$.
\end{proposition}

\begin{proof}
Let ${A \mr{gr^{-1}} C}$ be compatible with $f$. Take $K \mrpair{x_1}{x_2} A$ a kernel pair of $f$ in $\cc{E}_\alpha$  and ${K^* \mr{s} K}$ a cartesian morphisms over ${F(r)}$. 

\[
\xymatrix @+2.5ex {K^{**} \ar@{-->}[r]^t \ar@<-.5ex>[d]_{x_2^{**}} \ar@<.5ex>[d]^{x_1^{**}} & K^* \ar[r]^s \ar@<-.5ex>[d]_{x_2^*} \ar@<.5ex>[d]^{x_1^*} & K \ar@<-.5ex>[d]_{x_2} \ar@<.5ex>[d]^{x_1}  \\
		  A^{**} \ar[d]^{f^{**}} \ar@{-->}[r]^u & A^*  \ar[r]^r & A \ar[d]^f \\
		  B^{**}  \ar@{-->}[rr]^v & & B \\
		  {F(K^{**})} \ar[r]^{F(t)} & {F(A^*)} \ar[r]^{F(r)} & \alpha}
\]

\noindent Since $(g/r)  x_1=(g/r) x_2$, we have that ${gx_1^*=gx_2^*}$ in $\cc{E}[S^{-1}]$. Thus there is a cartesian morphism ${K^{**} \mr{t} K^*}$ such that ${(gx_1^*)t=(gx_2^*)t}$ in $\cc{E}$. Take ${A^{**} \mr{u} A^*}$ cartesian over ${F(t)}$ and ${B^{**} \mr{v} B}$ cartesian over ${F(r)F(t)=F(st)}$. The morphisms ${K^{**} \mrpair{x_1^{**}}{x_2^{**}} A^{**}}$ are a kernel pair of the strict epimorphism $f^{**}$ in the fibre over $F(K^{**})$, so $gu$ is compatible with $f^{**}$ in $\cc{E}$. By Lemma \ref{casi epis} there is a morphism $h \in hom_\cc{E}(B^{**},C)$ such that $gu=fh$ in $\cc{E}$. The morphism ${B \mr{h/v} C}$ yields the desired factorization.
\end{proof}

\begin{theorem} \label{los J alpha preservan epis estrictos}
The functors $\cc{E}_\alpha \mr{J_\alpha} \cc{E}[S^{-1}]$ preserve strict epimorphisms.
\end{theorem}

\begin{proof}
It follows from Propositions \ref{unicidad de la factorizacion} and \ref{factorizacion a travez de epis}.
\rojo{porque 3.17}   



\end{proof}



\begin{proposition}
Any morphisms  ${X \mr{f/s} Y \in \cc{E}[S^{-1}]}$ admits a strict epic - monic factorization.
\end{proposition}

\begin{proof}
For any morphism ${X \mr{f/s} Y \in \cc{E}[S^{-1}]}$ take a cartesian morphisms ${Y^* \mr{t} Y}$ over $F(f)$. We have the following situation.

\begin{align*}
\newdir{(>}{{}*!/-6pt/\dir{>}}
\xymatrix{ & X^* \ar[rr]^s \ar@{-->>}[dl]_e \ar@{-->}[dd]^{\exists ! f'} \ar[ddr]^f && X   \\
		  I \ar@{}[r]|(.6)\equiv \ar@{(>-->}[dr]_m  && \\
		   & Y^* \ar[r]_t \ar@{}[uur]|(.30)\equiv & Y \\
		   & F(X^*) \ar[r]^{F(f)} & F(Y)}
\end{align*}

\noindent The morphisms $m$ and $e$ form a strict epic - monic factorization of $f'$ in the fibre over $F(X^*)$. From Theorem \ref{los J alpha preservan epis estrictos} and the fact that the $J_\alpha$ preserve monics we have that the morphisms $e/s$ and $tm$ yield a strict epic - monic factorization of $f/s$.

\end{proof}


\begin{proposition}
Strict epimorphisms are stable in $\cc{E}[S^{-1}]$.
\end{proposition}

\begin{proof}
We will use as reference diagram \ref{pullbacks en el colimite}. Suppose $f_0/{s_0}$ is a strict epimorphism. Then $G^*(a_0)$ is a strict epimorphism in $\cc{E}[S^{-1}]$. Take a strict epic - monic factorization of $G^*(a_0)$ in $\cc{E}_\alpha$. We will take a composite pullback of $G^*(a_0)$ along $G^*(a_1)$ in $\cc{E}_\alpha$.


\begin{align*}
\newdir{(>}{{}*!/-6pt/\dir{>}}
\xymatrix @+2ex {P  \ar[rr]^p \ar@{->>}[d]_{e'} &{} \ar@{}[d]|{p.b.}  & {G^*}_0 \ar@{->>}[d]_e \ar@/^2pc/[dd]^{G^*(a_0)} \\
		  I'\ar[rr] \ar@{(>->}[d]_{m'} & {} \ar@{}[d]|{p.b.} & I \ar@{}[r]|(.3)\equiv \ar@{(>->}[d]_{m} & \\
		  {G^*}_2 \ar[rr]_{G^*(a_1)} && {G^*}_0}
\end{align*}

\noindent This diagram in fact is also true in $\cc{E}[S^{-1}]$. In fact  in $\cc{E}[S^{-1}]$  we have that $G^*(a_0)$ is a strict epic and consequently $m$ is an isomorphism in $\cc{E}[S^{-1}]$. Therefore $m'$ is an isomorphism in in $\cc{E}[S^{-1}]$ and so $m'e'$ is a strict epimorphisms in $\cc{E}[S^{-1}]$. The result follows.
\end{proof}

\subsection{Colimit of a Conservative Fibration Over $\cc{A}$}
Take  $\cc{A}$   a stable set of vertical arrows.


\begin{theorem} \label{colimit of conservative fibration}
If  $F$ is conservative over $\cc{A}$, then for every $\alpha \in \cc{G}$ the functors 
\mbox{$\cc{E}_\alpha \mr{J_\alpha} \cc{E}[S^{-1}]$} reflect isomorphisms that are already in $\cc{A}_\alpha$.
\end{theorem}

\noindent That is to say that if $f \in \cc{A}_\alpha$ and $J_\alpha(f)$ is an isomorphism, then $f$ is an isomorphisms.

\begin{proof}
Suppose ${X \mr{f} Y \in \cc{A}_\alpha}$ is such that $J_\alpha(f)$ is an isomorphism. Let ${Y \ml{t} Y^* \mr{g} X}$ represent its inverse. Take ${\alpha \mr{\varphi} F(Y^*)}$ such that ${F(t) \cdot \varphi=F(g) \cdot \varphi}=\psi$ and construct the following commutative diagram as indicated below.

\begin{align*}
\xymatrix @+2ex {Y^{****} \ar[r]^y \ar[d]_{g^{***}} & Y^{***} \ar[r]^v \ar[d]_{g^{**}}& Y^{**} \ar[r]^s \ar[d]_{g^{*}}& Y^* \ar[r]^t \ar[dr]_g & Y \\
		  X^{***} \ar[r]^x \ar[d]_{f^{***}}& X^{**} \ar[r]^w \ar[d]_{f^{**}}& X^* \ar[rr]^u \ar[d]_{f^{*}} && X \ar[d]^f \\
		  Y^{****} \ar[r]^y \ar[d]_{g^{***}} & Y^{***} \ar[r]^v \ar[d]_{g^{**}}& Y^{**} \ar[r]^s \ar[d]_{g^{*}}& Y^* \ar[r]^t \ar[dr]_g & Y \\
      	  X^{***} \ar[r]^x & X^{**} \ar[r]^w & X^* \ar[rr]^u && X \\		 
      	  F(X^{***}) \ar[r]^{F(x)} & F(Y^{***}) \ar[r]^{F(v)} & \alpha \ar[r]^\varphi & F(Y^*) \ar@/^/[r]^{F(t)} \ar@/_/[r]_{F(g)} & F(X)    }
\end{align*}

\noindent Take $s$ cartesian over $\varphi$, $u$ cartesian over $\psi$ and the corresponding vertical arrows $g^*$ and $f^*$. So it happens that $f^*g^*=1_{Y^{**}}$ and $g^*f^*=1_{X^*}$ in $\cc{E}[S^{-1}]$. Take $v$ a cartesian morphism such that $(f^*g^*)v=1_{Y^{**}}v$ in $\cc{E}$ followed by $w$ cartesian over $F(v)$. For the corresponding vertical arrows we have that $f^{**}g^{**}=1_{Y^{***}}$ in $\cc{E}$ and $g^{**}f^{**}=1_{X^{**}}$ in $\cc{E}[S^{-1}]$. Take $x$ a cartesian morphism such that $(f^{**}g^{**})x=1_{Y^{***}}x$ in $\cc{E}$ and $y$ cartesian over $F(x)$. It follows that $f^{***}$ and $g^{***}$ are inverse of eachother in the fibre over $F(X^{***})$. The result follows.



\end{proof}




\section{CONSTRUCTION OF A REGULAR SET VALUED FUNCTOR THAT IS CONSERVATIVE OVER MONICS WITH GLOBALLY SUPPORTED CODOMAIN FOR ANY REGULAR CATEGORY $\cc{A}$ THAT POSSESSES A DISTINGUISHED TERMINAL OBJECT} \label{construction of a model from A to Ens}

\subsection{Construction of the Functor from $\cc{A}$ to $\cc{A'}$ That Sends Globally Supported Objects into Objects That Have a Generic Global Section} \label{thee fibration}

In this section $\cc{A}$ will denote a regular category that possesses a distinguished terminal object $1$.

\subsubsection{A fibration that has $\cc{A}$ as  its fibres}
For  the following  fibration we will have that $\cc{A}$ can be identified as de fibre over $\{1\}$.

\subsubsection*{The Cofilitered Base for the Fibration}
 Strict epimorphisms are closed under composition in $\cc{A}$ (\ref{strict epics are closed under composition}). Take $\cc{G}l_s(\cc{A})$ the category whose objects are the globally supported objects of $\cc{A}$ and whose morphisms are the strict epimorphisms in $\cc{A}$. We define $\subC{G}{}$ to be  the category whose objects are finite sequences of objects $\Bi \subset \cc{G}l_s(\cc{A})$  whose first term is $B_0=1$. A morphism  ${\Bi \mr{\varphi} \Cj \in \subC{G}{}}$ is a function ${\intnat{m} \mr{\varphi} \intnat{n}}$ that verifies ${\varphi(0)=0}$ and that for every $j \in [m]$ it verifies ${B_{\varphi(j)}=C_j}$. 

\begin{remark}
$\subC{G}{}$ is a cofilitered category. More so it is finitely complete and has a \textit{unique} terminal object. This can be verified interpreting $\subC{G}{}^{op}$ embedded in ${\cc{E}ns^*}/{\cc{G}l_s(\cc{A})}$ where  $\cc{E}ns^*$ denotes the category of pointed sets and where we distinguish $1 \in \cc{G}l_s(\cc{A})$. 
\end{remark}



\subsubsection*{A Finitely Complete Fibration} \label{grothendieck construction}

We will give an explicit description of Grothendiecks construction of a split cofibration associated to the covariant functor  $D_{\cc{A}}:\subC{G}{} \mr{} \cc{C}at$ that assigns to each object $\Bi$ the multislice category $\cc{A}_{/\Bi}$ and to each arrow ${\Bi \mr{\varphi} \Cj}$ the functor $\varphi_*:{\cc{A}_{/\Bi} \mr{} \cc{A}_{/\Cj}}$ that is defined as ${\varphi_*(\{X \mr{x_i} B_i\}_{i \in [n]})=\{X \mr{x_{\varphi(j)}} C_j\}_{j \in [m]}}$ on objects and is the identity on arrows.


Take $\subC{E}{}$ the category whose objects are ordered pairs $(X,\alpha)$  where ${\alpha \in \subC{G}{}}$ and $X \in D_{\cc{A}}(\alpha)$. Its arrows are ordered pairs ${(X,\alpha) \mr{(f,\varphi)} (Y,\beta)}$ where $\alpha \mr{\varphi} \beta \in \subC{G}{}$ and $f:\varphi_*X \mr{} Y$. Composition is defined for ${(Y,\beta) \mr{(g,\psi)} (Z,\gamma)}$ as $(g,\psi)(f,\varphi)=(g \cdot \psi_* (f),\psi\varphi)$. Take $F_{\cc{A}}$ be the projection in the second coordinate. The arrow ${(X,\alpha)\mr{(1_{\varphi_*X},\varphi)}(\varphi_*X,\beta)}$ is cocartesian over $\varphi$ with source $X$ and these arrows are closed under composition. The projection in the first coordinate restricted to a fiber ${(\subC{E}{})_\alpha \mr{\pi_1} D_{\cc{A}}(\alpha)}$ is an isomorphism. If $\varphi_*$ denoted the (co) pullback functor along $\varphi$ we have in fact this isomorphism  that is natural in the following sense:


\begin{align} \label{transfers son los funtores}
\xymatrix @+2.5ex {{(\subC{E}{})_\alpha} \ar [d] _{\pi_1} \ar [r] ^{\varphi_*} &  {(\subC{E}{})_\beta} \ar [d] ^{\pi_1} \\
		     {D_{\cc{A}}(\alpha)} \ar [r] _{D_{\cc{A}}(\varphi)}      & {D_{\cc{A}}(\beta)} \ar @{} [ul] |{\equiv}}
\end{align}





Thus we can make the abuse of language of identifying the fiber of the split cofibration ${\subC{E}{} \mr{F_{\cc{A}}} \subC{G}{}}$ over $\Bi$ with $\cc{A}_{/\Bi}$ and similarly identify the cotransport functor along $\varphi$ with $D_{\cc{A}}(\varphi)$. 


\begin{proposition}\label{obtuvimos una fibracion}
$F_{\cc{A}}$ is a fibration.
\end{proposition}

\begin{proof}

\rojo{FIX-001.02 START Dar referencia directa a SGA1 en lugar de proposition 2.2}

It suffices to prove that $F_{\cc{A}}$ is prefibered (see \cite[page 143]{sga1}).
%It suffices to prove that $F_{\cc{A}}$ is prefibered (see \ref{cartes son cartes fuertes en una }). 

\rojo{FIX-001.02 END}



Take ${\Bi \mr{\varphi} \Cj \in \subC{G}{}}$ and  $\{Y \mr{y_j} C_j\}_{j \in [m]}$ over $\Cj$. Let $\cc{D}_{\varphi}$ be the finite graph whose objects are $[n+1]$ and whose arrows are identified with $[m]$. The arrow $j \in [m]$ has source $n$ and target $\varphi(j)$. The object $\{Y \mr{y_j} C_j\}_{j \in [m]}$ induces a functor ${\cc{D}_{\varphi} \mr{\tilde{Y}} \cc{A}}$ defined as $\tilde{Y}n=Y$, as $\tilde{Y}i=B_i$ for any other $i \in [n]$  and $\tilde{Y}j=y_j$ on arrows.

 A cone  for this functor is a family of arrows ${\{X \mr{x_i} \tilde{Y}i\}_{i \in [n+1]}}$ such that for every ${j\in[m]}$ the following diagram is commutative.
 
\[
\xymatrix{& X \ar[dl]_{x_n} \ar[dr]^{x_{\varphi(j)}} \ar@{}[d]|\equiv \\
		  Y \ar[rr]_{y_j} && B_{\varphi(j)}}
\] 
 
\noindent Thus it is naturally identified with a morphism ${\{X \mr{x_i} \tilde{Y}i\}_{i \in [n]} \mr{(x_n,\varphi)} \{Y \mr{y_j} C_j\}_{j \in [m]}}$ over $\varphi$ with target $\{Y \mr{y_j} C_j\}_{j \in [m]}$. A limit cone corresponds to a cartesian morphism. Since $\cc{A}$ is finitely complete the result follows.
\end{proof}

\begin{proposition}
 $F_{\cc{A}}$ is finitely complete.
\end{proposition}

\begin{proof}
It follows from Theorem \ref{prefi mas precofib preserva limites} and that the fibers are multislice categories of a regular category, in particular finitely complete.
\end{proof}

\subsubsection*{A Regular Fibration}
We will in fact prove that  ${\subC{E}{} \mr{F_{\cc{A}}} \subC{G}{}}$ is a regular fibration.


\begin{lemma}\label{pullbacks along phi are pullbacks in C}
In $\subC{E}{}$ if

\[
\xymatrix{ \{W \mr{w_i} B_i\}_{i \in [n]} \ar[r]^{(f,\varphi)} \ar[d]_{(a,1)} \ar@{}[dr]|{\equiv} & \{X \mr{x_j} C_j\}_{j \in [m]} \ar[d]^{(b,1)} \\   
           \{Z \mr{z_i} B_i\}_{i \in [n]} \ar[r]^{(g,\varphi)} & \{Y \mr{y_j} C_j\}_{j \in [m]} \\
           \Bi  \ar[r]^\varphi & \Cj
           }
\]



is such that $(f,\varphi)$ and $(g,\varphi)$ are cartesian, then

\[
\xymatrix{ W \ar[r]^f \ar[d]_a \ar@{}[dr]|{\equiv} & X  \ar[d]^b \\   
           Z \ar[r]_g & Y}
\]
is a pullback in $\cc{A}$.

\end{lemma}


\begin{proof}
For any cone $\{V \mr{h} X, V \mr{c} Z \}$ in $\cc{A}$ we have the object $\{V \mr{z_ic} B_i\}_{i \in [n]}$ together with the cone $\{(c,1),(h,\varphi)\}$. A factorization of the former cone in $\cc{A}$ is identified with a factorization of the latter in $\subC{E}{}$ over $\Bi$. 
\end{proof}

\begin{proposition}\label{trans pres epics}
Strict epimorphisms are stable in the fibration $F_{\cc{A}}$.
\end{proposition}

\begin{proof}
It follows from Lemma \ref{pullbacks along phi are pullbacks in C} and the fact that the domain functors $\Sigma$ in multislice categories preserve and reflect strict epimorphisms.
\end{proof}

\begin{corollary}
$F_{\cc{A}}$ is a regular fibration.
\end{corollary}


\subsubsection{Construction of the colimit $\cc{A'}$ of the fibration and proof that including the first fibre is conservative over monics with globally supported codomain}

Theorem \ref{colimit of regular fibrations is regular} guarantees that the colimit of this fibration is a regular category and in particular the functors $J_{\{1\}}$ in the diagram below is regular.  
\[
\xymatrix{ ({\subC{E}{}})_{{\{1\}}}  \ar[dr]_{j_{\{1\}}}  \ar[rr]^{J_{\{1\}}}&& \subC{E}{}[S^{-1}] \\
& \subC{E}{} \ar[ur]_Q \ar@{}[u]|\equiv}
\]

\noindent Identifying $({\subC{E}{}})_{{\{1\}}}$ with $\cc{A}$ we will label to top arrow in the diagram with $\cc{A} \mr{j} \cc{A'}$. For  a morphism $X \mr{f} Y \in \cc{A}$  will use the abuse of language of saying  $X \mr{f} Y$ in $\cc{A'}$ referring to the morphism $j(X) \mr{j(f)} j(Y) \in \cc{A'}$. Taking into consideration that $j$ transforms $1$ into a terminal object, preserves monics and preserves strict epimorphisms makes the abuse coherent with these objects. 

\subsubsection*{A generic section for every $B \twoheadrightarrow 1 \in \cc{A}$ }

Take a globally supported object ${\xymatrix{B \ar@{->>}[r]^\pi & 1}\in\cc{A}}$. The fiber over $\{1,B\}$ is naturally identified with $\cc{A}_{/B}$. Choose a product $\{B \times B \mrpair{\pi_1}{\pi_2} B\}$  of $B$ with itself in $\cc{A}$ and take
${B \mr{\Delta} B \times B}$ the diagonal morphism. We obtain the following diagram in $\subC{E}{}$.

\[
\xymatrix{\{P \mr{\pi_2} B\}\ \ar[r]^(.6){(\pi_1,\varphi)} & B  \\
			\{B \mr{id_B} B \}\ \ar[r]_(.6){(\pi,\varphi)} \ar[u]^{(\Delta,1)} & 1 \\
		  \{1,B\} \ar[r]^\varphi  & \{1\} 	}
\]


\begin{lemma}
$\{B \mr{id_B} B \} \mr{(\pi,\varphi)} 1$ and $\{P \mr{\pi_2} B\} \mr{(\pi_1,\varphi)} B$ are cartesian morphisms.
\end{lemma}

\begin{proof}
This follows immediately using the characterization of cartesian morphisms given in Proposition \ref{obtuvimos una fibracion}. 
\end{proof}

\begin{remark}
We have a section ${\frac{(\pi_1  \Delta,\varphi)}{(\pi,\varphi)}=\frac{(1_B,\varphi)}{(\pi,\varphi)}}$ of ${\xymatrix{B \ar@{->>}[r]^\pi & 1}}$ in $\cc{A'}$. This section is in fact \textit{canonical} in the sense that any choice of product ${\{B \times B \mrpair{\pi_1}{\pi_2} B\}}$  of $B$ with itself in $\cc{A}$ will induce the \textit{same} arrow in $\cc{A'}$ built this way. This follows from the fact that $(\pi_1,\varphi)$ is cartesian. We will label this uniquely determined arrow ${1 \mr{\Delta_B} B}$.

\end{remark}

\subsubsection*{Separating $B$ from its subobjects in $\cc{A}$}

We will prove is that $1 \mr{\Delta_B} B$  \textit{separates} $B$ from its subobjects in $\cc{A}$, in the sense of Theorem \ref{delta separa a B de sus subobjetos}.

\begin{lemma}\label{esos epis especiales}
For any ${\Bi \mr{\psi} \{1,B\}}$ in $\cc{G}$, if ${\{X \mr{x_i} B_i\}_{i \in [n]} \mr{(f,\psi)} \{B \mr{id_B} B \}}$ is cartesian, then $f$ is a strict epimorphism in $\cc{A}$.
\end{lemma}

\begin{proof}
Note that  $\{X \mr{x_i} B_i\}_{i \in [n]}$ is a product of the family $\Bi$ and that $f$ is in fact one of the projections. The result follows (recall every $B_i$ is globally supported and \ref{projectar es epi estricto}).
\end{proof}

\begin{theorem}\label{delta separa a B de sus subobjetos}
If $A\ \mnr{m} B \in \cc{A}$ is such that $\Delta_B$ lifts along $m$,
\[
\newdir{(>}{{}*!/-6pt/\dir{>}}
\xymatrix{A \ar@{(>->}[rr]^{m} & & B \\ & & 1 \ar[u]_{\Delta_B} \ar@{.>}[ull]^{\exists} \ar@{}[ul]|(.6){\equiv}}\ \ 
\]
it follows that $m$ is an isomorphism.
\end{theorem}

\begin{proof}
In our context the existence of such a lifting of $\Delta_B$ reduces to there being a morphism ${\Bi \mr{\psi} \{1,B\}}$ in $\subC{G}{}$, a cartesian morphism ${\{X \mr{x_i} B_i\}_{i \in [n]} \mr{(f,\psi)} \{B \mr{id_B} B \}}$ over $\psi$ and a morphism ${\{X \mr{x_i} B_i\}_{i \in [n]} \mr{(g,\varphi \circ	\psi)} A}$ such that the following diagram is commutative.

\[
\xymatrix@+2.5ex {                   &      & {}^{\Huge A} \ar@{>->}[d]^m  \\                                                                                   
			                 &       \{P \mr{\pi_2} B\}\ \ar[r]^(.6){(\pi_1,\varphi)}           & B  \\
\{X \mr{x_i} B_i\}_{i \in [n]} \ar[r]^{(f,\psi)} \ar@/^3pc/[uurr]^{(g,\varphi \circ	\psi)} & \{B \mr{id_B} B \}\ \ar[r]_(.6){(\pi,\varphi)} \ar[u]_{(\Delta,1)} & 1\\
\Bi \ar[r]^\psi  & \{1,B\} \ar[r]^\varphi & \{1\}  }
\]

It follows that  $m  g=f$ and together with Lemma \ref{esos epis especiales} $m$ must be an isomorphism. 


\end{proof}

\begin{corollary}
The functor $j_{\{1\}}$ is conservative over monics with globally supported codomain.
\end{corollary}

\input{unoprojdosbis_v2.tex}

\input{reduction_v2.tex}
%\pagebreak
%%%%% end Body of the thesis

\bibliographystyle{unsrt}

\begin{thebibliography}{99}

\bibitem{bordos}
Borceux, Francis, \emph{Handbook of categorical algebra}, 2, volume 51 of Encyclopedia of Mathematics and its Applications, Cambridge University Press, Cambridge (1994).

\bibitem{sga1}
Grothendieck, Alexander, Raynaud, Michele
\emph{Rev\^etements \'etales et groupe fondamental (SGA 1)}, 
arXiv preprint math/0206203, (2002).

\bibitem{sga4}
Artin, M and Grothendieck, A and Verdier, J
\emph{SGA 4,(1963-64)}, 
Springer Lecture Notes in Mathematics 270, (1972).

\bibitem{gabzis} Gabriel, Peter and Zisman, Michel, 
\emph{Calculus of fractions and homotopy theory},
Springer xxxx  6  (1967).

\bibitem{tesemi} Descotte, Mar\'ia Emilia, 
\emph{Una Generaliciaci\'on de la Teor\'ia de Ind-objetos de Grothendieck a 2-Categor\'ias}, Tesis de Licenciatura,
Universidad de Buenos Aires, (2010).  

\bibitem{whitney} Scholz, Erhard, 
\emph{History of topology}, pp 25-64, 
Dordrecht: Kluwer, (1999).

\end{thebibliography}


\end{document}

%%%%%%%%%%%%%%%%%%%%%%%%%%% COMENT %%%%%%%%%%%%%%%%%%%%%%%%%%%%%%%%%%%%
\begin{comment}
\begin{proof}
For each $i$ consider the following commutative diagram in $\cc{C}$

\[
\xymatrix{P \ar[rr]^{\pi_1} \ar[dr]_{\pi_2} \ar@{}[drrr]|{p.b.} &   & X 
\ar[dd]_(.7){x_i}|!{[dl];[dr]}\hole \ar[dr]^f & \\ 
		   &  Y \ar[rr]^(.6)g \ar@/_/[dr]_{y_i}  & & Z \ar@/^/[dl]^{z_i}
\\
		   & & B_i &			    }
\]

This diagram determines a cone for $f$ and $g$ in $\cc{C}_{/\Bi}$. It follows from Remark \ref{props de F} that this cone is a pullback.
\end{proof}

\begin{observation}
If  distinguished pullbacks exist in $\cc{C}$, then  distinguished pullbacks exist in $\cc{C}_{/\Bi}$ and $\Sigma$ preserves them.
\end{observation}


\begin{remark}\label{slice have terminal object}
If a product of the family $\Bi$ exists, then the family of its projections  $\{P \mr{\pi_i} B_i\}_{i \in [n]}$ form a terminal object of the multislice category.
\end{remark}

\end{comment}
%%%%%%%%%%%%%%%%%%%%%%%%%% END %%%%%%%%%%%%%%%%%%%%%%%%%%%%%%%%%%%%%%%%%
