\subsection{Multislice Categories of Regular Categories} \label{slices}
In this section we will define what a multislice category is and prove that  multislice categories of finitely complete categories and  of regular categories are finitely complete and regular, respectively.

\begin{definition}
For a category $\cc{C}$ and a family $\Bi$ of objects of $\cc{C}$ we define the \emph{multislice category} $\cc{C}_{/\Bi}$ whose objects are families 
${\{X \mr{x_i} B_i\}_{ i \in [n]}}$ of arrows of $\cc{C}$ and morphisms ${\{X \mr{x_i} B_i\}_{ i \in [n]} \mr{f} \{Y \mr{y_i} B_i\}_{ i \in [n]}}$ are arrows $X \mr{f} Y$ in $\cc{C}$ such that for every ${ i \in [n]}$

\[
\xymatrix{X \ar[rr]^f \ar[dr]_{x_i} && Y \ar[dl]^{y_i}
\\
		    & B_i \ar@{}[u]|(.6)\equiv & }
\]
\end{definition}

\begin{remark} \label{Sigma}
Clearly there is a forgetful functor,   
$\cc{C}_{/{\{B_i\}_{i \in I}}} \mr{\Sigma} \cc{C}$\,, 
\mbox{$\Sigma({\{X \mr{x_i} B_i\}_{ i \in [n]}} \,=\, X$, $\Sigma(f) = f$,}
which is faithful and reflects isomorphisms. We will refer to this functor  as \emph{the functor $\Sigma$}. \cqd
\end{remark}

%\end{document}
The following holds by definition of products:
 \begin{remark} \label{propSigma}
Let ${\{B \mr{\pi_i} B_i\}_{ i \in [n]}}$ be a product diagram in $\cc{C}$, then the functor:
$$
\cc{C}_{/B}  \mr{\Phi}  \cc{C}_{/\Bi}: \hspace{6ex} 
\Phi(X \mr{x} B)  \;=\;  {\{X \xr{\pi_i  x} B_i\}_{ i \in [n]}}
$$
and $\Phi(f) = f$ on arrows, establishes an isomorphism of categories. Furthermore the family of its projections is a terminal object in the multislice category.
\end{remark}
The reason we consider multislice categories is not that we want to work with categories lacking products, but lacking a distinguished choice of products.

\emph{For categories with finite products we can use this Remark to transport to multislice categories the properties of slice categories.}

Though properties needed here may be proven directly for multislice categories, to simplify the notation we will do the proofs for slice categories and then use this remark.



%\end{document}
%\begin{remark}\label{props de F}
%The domain functor $\cc{C}_{/{\{B_i\}_{i \in I}}} \mr{\Sigma} \cc{C}$ is faithful, conservative and preserves pullbacks. It follows that $\Sigma$ reflects monomorphisms and pullbacks (\ref{con pullback reflejo monos}).
%\end{remark}

\begin{remark} \label{adjuntoderecha}
Let $X \in \cc{C}$, let  
$B \ml{\pi_2} E \mr{\pi_1}  X$ be a product diagram in $\cc{C}$, and let 
$Y \mr{y} B \, \in \, \cc{C}_{/B}$.  Define $\Pi(X) = (E \xr{\pi_2} B)$, then we have:

$$
\xymatrix@C=4ex@R=2ex
     {
      Y \ar@/^0.8pc/[drr]^{g}
        \ar@/_0.8pc/[ddr]^{y} 
        \ar[dr]^f
    \\
    & E \ar[d]^{\pi_2}
        \ar[r]^{\pi_1}
    & X
   \\
    & B
     }
\hspace{8ex} 
\xymatrix@C=5ex@R=0ex
     {
      & (Y \xr{y} B) \ar[r]^f  &  \Pi(X) %(E \xr{\pi_2} B)
     \\
     {} \ar@{-}[rrr] &&& {}
     \\
      & \Sigma(Y \xr{y} B)  \ar[r]^g & X
     }
$$    
By definition of product the diagram on the left establishes a bijective correspondence between the arrows $f$ and $g$, which is the same that the  correspondence  indicated in the right diagram. This shows that the object $\Pi(X) \in \cc{C}_{/B}$ is the value at $X$ of a right adjoint to the functor $\Sigma$, the defining universal property shows it is defined in arrows in a way that preserves composition, compare with Definition \ref{pbk1}.
 
It follows that \emph{when the product $E$ exists the functor $\Sigma$ will preserve any colimit that may exists in $\cc{C}_{/B}$}.


 \cqd
\end{remark}

The following is immediate and very easy to prove:
{
%\eazul
\begin{proposition}\label{p-pulback}
The category  $\cc{C}_{/\Bi}$ inherits any pull-back that exists in $\cc{C}$, moreover the functor $\Sigma$ preserves and reflects pull-backs. It follows it always preserves and reflects monomorphisms \cqd
\end{proposition} 
}

Since as observed in \ref{propSigma} for a category with finite products any multislice category has terminal objects, it follows:
\begin{proposition} \label{finitelycomplete}
If $\cc{C}$ is finitely complete, then $\cc{C}_{/\Bi}$ is finitely complete.
\cqd
\end{proposition}

%%%%%%%%%%%%%%%%%%%%%%%%%%%%%%%%%%%%%%%%%%%%%%%%%%%%%%%%%%%%%%%%%%%%
%%%%%%%%%%%%%%%%%%%%%%%% COMENTARIO %%%%%%%%%%%%%%%%%%%%%%%%%%%%%%%%
\begin{comment}
\begin{proposition} \label{reflexepi}
The functor $\Sigma$ reflex and preserves strict epimorphism.
\end{proposition}
\begin{proof}
Let $(X \mr{x} B) \mr{f} (Y \mr{y} B)$ be an arrow in 
$\cc{C}_{/B}$. We will argue below referring to the following diagram:
$$
\xymatrix
    {
     C \ar@<.5ex>[r]^r 
       \ar@<-.5ex>[r]_s 
       \ar@/_1.3pc/[ddrr]^{c} 
   & X \ar[rr]^f 
       \ar[dr]^g 
       \ar@/_/[ddr]^{x} 
  && Y \ar@/^/[ddl]^{y} 
       \ar[dl]_{h}
\\ 
  && Z \ar[d]^{z} 
  & 
\\
  && B 	
  &		  
     }
$$
\emph{reflex}: Assume $f$ is a strict epimorphism in $\cc{C}$. Let 
$C \mrpair{r}{s} X$ any two arrows in $\cc{C}$, and
$(X \mr{x} B) \mr{g} (Z \mr{z} B)$ a $f$-compatible arrow in $\cc{C}_{/B}$. Composing with $x$ we have $C \mrpair{xr}{xs} B$, so that $r,\;s$ determine arrows in  
$\cc{C}_{/B}$. This shows that $g$ is also \mbox{$f$-compatible} in 
$\cc{C}$, so we have a unique arrow $h$ such that $hf = g$ in $\cc{C}$. Since $f$ is also an epimorphism, it follows that $zh = y$, thus $h$ is an arrow in $\cc{C}_{/B}$.

\vspace{1ex}

\noindent \emph{preserves}: Assume $f$ is a strict epimorphism in 
$\cc{C}_{/B}$.
% Let $(C \mr{c} B) \mrpair{r}{s}  (X \mr{x} B)$ be any two arrows in $\cc{C}_{/B}$, and 
Let $C \mrpair{r}{s} X$ be any two arrows in $\cc{C}$, as before they determine two arrows in $\cc{C}_{/B}$, and let 
$(X \mr{x} B) \mr{g} (Z \mr{z} B)$ be a $f$-compatible arrow  in 
$\cc{C}_{/B}$. It follows there is a unique $h$, $hf = g$ in $\cc{C}_{/B}$, and since the functor  
$\Sigma$ is faithful, $h$ is also uniqe in $\cc{C}$. 
\end{proof}
\end{comment}
%%%%%%%%%%%%%%%%%%%%%%%%%%%%% FIN COMENTARIO %%%%%%%%%%%%%%%%%%%%%%%%%%
%%%%%%%%%%%%%%%%%%%%%%%%%%%%%%%%%%%%%%%%%%%%%%%%%%%%%%%%%%%%%%%%%%%%%%%

 From Remaks \ref{kernelpair}, \ref{adjuntoderecha}, and Proposition 
 \ref{p-pulback} if follows:
 \begin{proposition}\label{preserves}
 If $\cc{C}$ has pullbacks and binary products, then the functor $\Sigma$ preserves strict epimorphisms. 
 \end{proposition}
 
 \begin{proposition} \label{reflects}
The functor $\Sigma$ always reflects strict epimorphisms.
\end{proposition}
\begin{proof}
Let $X \mr{f} Y$ be a strict epimorphism in $\cc{C}$. We want to prove that any 
$$
(X \mr{x} B) \mr{f} (Y \mr{y} B)
\hspace{10ex} 
\xymatrix 
     {
      X \ar[rr]^{f} 
        \ar@/_5pt/[rd]^{x}
   && Y \ar@/^5pt/[ld]_{y}
  \\ 
    & B
     }
$$    
is a strict epimorphism in $\cc{C}_B$. Let $g$ be a $f$-compatible arrow in 
$\cc{C}_B$
$$
(X \mr{x} B) \mr{g} (Z \mr{y} B)
\hspace{10ex}
\newdir{(>}{{}*!/-6pt/\dir{>}}
\xymatrix
    {
     X \ar[rr]^f 
       \ar@{->>}[dr]_g 
       \ar@/_/[ddr]_{x} 
  && Y \ar@/^/[ddl]^{y} 
   \\  
   & Z %\ar@{}[u]|(.6){\equiv} 
       \ar[d]^(.43){z} 
   &
    \\
   & B    
   &	          }
$$ 
 We want to see that $g$ is $f$-compatible in $\cc{C}$. Let $C \mrpair{r}{s} X$ in $\cc{C}$ be such that $fr = fs$. Since $xs = yfs = yfr = xr, \; say \; = c$,  composing with 
$X \mr{x} B$ yields 
$$\xymatrix 
     {
      C \ar@<.5ex>[rr]^{r}
        \ar@<-.5ex>[rr]_{s}
        \ar@/_5pt/[rd]^{c}
   && X \ar@/^5pt/[ld]_{x}
  \\ 
    & B
     }
$$ 
Thus $r$ and $s$ determine arrows in $\cc{C}_{/B}$ such that $fr = fs$, and since $g$ is $f$-compatible in $\cc{C}_{/B}$ it holds $gr = gs$. This shows that $g$ is $f$-compatible in $\cc{C}$. It follows then that there exists a unique $Y \mr{h} Z$ such that $hf = g$ in $\cc{C}$. We have
$$
\newdir{(>}{{}*!/-6pt/\dir{>}}
\xymatrix
    {
     X \ar[rr]^f 
       \ar@{->>}[dr]^g 
       \ar@/_/[ddr]_{x} 
  && Y \ar@/^/[ddl]^{y} 
       \ar[dl]_{h}
   \\  
   & Z %\ar@{}[u]|(.6){\equiv} 
       \ar[d]^(.43){z} 
   &
    \\
   & B    
   &
   }
$$
It remains to see that $h$ is an arrow in $\cc{C}_{/B}$, that is, 
$zh = y$. But $zhf = zg =x =yf$, and since $f$ is in particular an epimorphism, this shows what we want.
\end{proof}



\begin{proposition}\label{slice have factorization}
For any arrow  
\mbox{$(X \mr{x} B) \mr{f} (Y \mr{y} B)$} in $\cc{C}_{/B}$, a strict factorization of $f$ in $\cc{C}$ determines a strict factorization of $f$ in 
$\cc{C}_{/B}$. Thus if every arrow in $\cc{C}$ admits a strict factorization, so does every arrow in $\cc{C}_{/\Bi}$.
\end{proposition}
\begin{proof} 
Let $(X \mr{x} B) \mr{f} (Y \mr{y} B)$ be an arrow in 
$\cc{C}_{/B}$, and 
take a strict factorization of $f$ in $\cc{C}$:
%
$$
\newdir{(>}{{}*!/-6pt/\dir{>}}
\xymatrix
    {
     X \ar[rr]^f \ar@{->>}[dr]_e 
       \ar@/_/[ddr]_{x} && Y \ar@/^/[ddl]^{y} 
   \\  
   & Z \ar@{(>->}[ur]_m 
       \ar@{}[u]|(.6){\equiv} 
       \ar@{-->}[d]^(.43){z} 
   &
    \\
   & B    
   &	          }
$$ 
% 
Since $x$ is $e$-compatible, there is a unique arrow 
$Z \mr{z} B$ such that $ze = x$. Since $e$ is epic it follows that $ym = z$. By \ref{reflects} $e$ is an strict epimorphism in  $\cc{C}_{/B}$, and by \ref{p-pulback} $m$ is a monomorphism in  $\cc{C}_{/B}$. 
\end{proof}

%\begin{observation}
%If every morphism in $\cc{C}$ has a distinguished strict factorization, then so does every morphism in $\cc{C}_{/\Bi}$ and $\Sigma$ preserves them.
%\end{observation}

%{\eazul
\begin{proposition} \label{universal}
Strict epimorphisms are universal in $\cc{C}_{/\Bi}$.
\end{proposition}
\begin{proof}
It follows from the fact that the functor $\Sigma$ preserves and reflects pullbacks and strict epimorphism, \ref{p-pulback},   
\ref{preserves}, \ref{reflects}.       
\end{proof}

\vspace{1ex}

Propositions \ref{finitelycomplete},  \ref{slice have factorization} and
 \ref{universal} put together show:

\begin{proposition} \label{sliceregular}
If $\cc{C}$ is regular, then $\cc{C}_{/\Bi}$ is regular. \cqd
\end{proposition}
%}
%%%%%%%%%%%%%%%%%%%%%%%%%%%%%%%%%%%%%%%%%%%%%%%%%%%%%%%%%%%%%%%%%%%%
%%%%%%%%%%%%%%%%%%%%%%%% COMENTARIO %%%%%%%%%%%%%%%%%%%%%%%%%%%%%%%%
\begin{comment}
In this section we will establish some generalities on the collective behaviour that a family of functors may have. We will prove that for a regular category $\cc{C}$,  a \textit{monic-conservative} family of regular functors is faithful and conservative.

 Let $\mathcal{F}$ be a family of functors with common domain $\cc{C}$.

\begin{definition}
We will say that pullbacks (pushouts,...) are \textbf{preserved} by $\cc{F}$ if for every $F \in \cc{F}$ the functor $F$ preserves pullbacks (pushouts,...).
\end{definition}

\begin{definition}
We will say that monomorphisms (epimorphisms,...) are \textbf{reflected} by $\cc{F}$ if for every arrow ${X\mr{u}}Y \in \mathcal{C}$ such that for every $F \in \cc{F}$  its image $Fu$ is a monomorphism (epimorphism,...), it follows that $u$ is a monomorphism (epimorphism,...).
\end{definition}


\begin{definition}
$\cc{F}$ is \textbf{conservative} if $\cc{F}$ reflects isomorphisms.
\end{definition}

\begin{definition}
$\cc{F}$ is \textbf{monic} (\textbf{epic})\textbf{-conservative} if for every monic (epic) ${{X \mr{u} Y}  \in \mathcal{C}}$ such that for every $F \in \cc{F}$ its image $Fu$ is an isomorphism, it follows that $u$ is an isomorphism.  
 \end{definition}


\begin{observation}
A conservative family is monic (epic)-conservative. 
\end{observation}

\begin{definition}
$\cc{F}$ is \textbf{faithful} if for every pair ${X\mrpair{u}{v}Y} \in \cc{C}$ such that for every $ F\in \cc{F}$ their images are  equal ($Fu=Fv$), it follows that $u=v$.
 \end{definition}


\begin{lemma}\label{si fiel, refleja monos}
If $\cc{F}$ is faithful, then $\cc{F}$ reflects monics(epics). 
\end{lemma}

\begin{proof}
Let $X \mr{u} Y$ in $\cc{C}$ be such that for every $F\in\cc{F}$, $Fu$ is monic. Suppose we have $A \mrpair{x}{y} X$ such that $ux=uy $. Then for every $F\in\cc{F}$, $Fu \cdot Fx=Fu \cdot Fy$ and since $Fu$ is monic, it follows that for every $F\in\cc{F}$, $Fx=Fy$. Thus $x=y$. The dual proposition follows.
\end{proof}

\begin{lemma}\label{con pullback reflejo monos}
If $\cc{C}$ has pullbacks (pushouts), $\cc{F}$ preserves them and $\cc{F}$ is monic (epic)-conservative, then $\cc{F}$ reflects monics (epics). 
\end{lemma}

\begin{proof}
Let $X \mr{u} Y$ be such that for every $F \in \cc{F}$, $Fu$ is monic. We will prove that the following diagram is a pullback
\[
\xymatrix{X \ar[r]^{id_X} \ar[d]_{id_X} & X \ar[d]^u \\ X \ar[r]_u & Y}
\]

Take a pullback in $\cc{C}$ and $\delta$ as follows:

%
%$$  
%\xymatrix @+3.5ex {X \ar@/_1.5pc/[ddr]_{id_X} \ar@/^/[drr]^{id_X} \ar@{-->}[dr]_{\exists! \delta} & \\ & P \ar@{}[dr]|{p.b.} \ar@<.1ex>@{}[u]|(.36){\equiv} \ar@<.1ex>@{}[l]|(.5){\equiv} \ar[r]^{p_1} \ar[d]_{p_2} & X \ar[d]^u \\ & X \ar[r]_u & Y} 
%$$
% 
$$ 
\xymatrix @+3.5ex
     { 
      X \ar@/_1.5pc/[ddr]_{id_X} 
        \ar@/^/[drr]^{id_X} 
        \ar@{-->}[dr]_{\exists! \delta} 
     & {}
   \\ 
     & P \ar@{}[dr]|{p.b.} 
         \ar@<.1ex>@{}[u]|(.36){\equiv} 
         \ar@<.1ex>@{}[l]|(.5){\equiv} 
         \ar[r]^{p_1} 
         \ar[d]_{p_2} 
      & X \ar[d]^u 
     \\ 
     & X \ar[r]_u 
     & Y
     } 
$$
 
From this diagram we see $\delta$ is monic and for every $F \in \cc{F}$, $F\delta$ is the isomorphism between the corresponding pullback diagrams. Therefore $\delta$ is an isomorphism. The dual proposition follows.
\end{proof}

\begin{remark}
The proof of Lemma \ref{con pullback reflejo monos} gives us a technique to prove that under the additional hypothesis of preserving limits (colimits) of a certain type, conservative families reflect limits (colimits) of that type.
\end{remark}

\begin{proposition}\label{pb + mc implica c}
If $\cc{C}$ has pullbacks (pushouts), $\cc{F}$ preserves them and $\cc{F}$ is monic (epic)-conservative, then $\cc{F}$ is conservative. 
\end{proposition}

\begin{proof}
Observe that monic (epic)-conservative families that reflect monics (epics) are conservative.
\end{proof}

\begin{proposition}\label{cmc implica cf}
If $\cc{C}$ has equalizers (coequalizers), $\cc{F}$ preserves them and $\cc{F}$ is monic/strict-monic (epic/strict-epic)-conservative, then $\cc{F}$ is faithful.
\end{proposition}

\begin{proof}
Take $X\mrpair{u}{v}Y$ in $\cc{C}$ such that for every $F \in \cc{F}$ their images are equal ($Fu=Fv$). An equalizer $E\mr{e}X$  of $u$ and $v$  is monic and its image $Fe$ is an equalizer of $Fu$ and $Fv$ which are equal, so $Fe$ is an isomorphism. Thus $e$ is an isomorphism and consequently $u=v$. For the strict case we need only note that equalizers are strict monics. The dual propositions follow.
\end{proof}

\begin{proposition}
If $\cc{C}$ has equalizers (coequalizers), $\cc{F}$ preserves them and $\cc{F}$ is monic (epic)-conservative, then $\cc{F}$ is conservative.
\end{proposition}

\begin{proof}
It follows from Lemma \ref{si fiel, refleja monos} and the observation made in Proposition \ref{pb + mc implica c}.
\end{proof}


\begin{corollary}
If $\cc{C}$ has equalizers, $\cc{F}$ preserves them and $\cc{F}$ is monic (epic)-conservative, then $\cc{F}$ is conservative and faithfull.
\end{corollary}

\begin{remark}\label{monic conservative implica conservative}
If $\cc{C}$ is regular and $\cc{F}$  the set of all regular functors $\cc{C} \mr{} \cc{E}ns$ is monic conservative, it follows that it is conservative and faithful.
\end{remark}


\begin{remark}
Even under the strictest limit-preserving conditions we will not be able to guarantee that a faithful family is conservative in any sense. Take the following counterexample: Let $\cc{C}=\{0 \mr{u} 1\}$ and take the family whose only member is the functor $\cc{C} \mr{F} \{*\}$. $\cc{C}$ is a regular category that in fact  has all limits and colimits, $F$ is regular and preserves all limits and colimits , $F$ is faithful but nevertheless does not reflect the isomorphism $Fu$.
\end{remark}
 
\begin{proposition}
If in $\cc{C}$ every bimorphism (a morphism that is both epic and monic) is an isomorphism and $\cc{F}$ is faithful, then $\cc{F}$ is conservative.
\end{proposition}
\end{comment}
%%%%%%%%%%%%%%%%%%%%%%%%%%%%% FIN COMENTARIO %%%%%%%%%%%%%%%%%%%%%%%%%%
%%%%%%%%%%%%%%%%%%%%%%%%%%%%%%%%%%%%%%%%%%%%%%%%%%%%%%%%%%%%%%%%%%%%%%%

%\end{document}

\subsection{Families of Functors With Common Domain} \label{families}


In this section we will establish a result, Proposition 
\ref{pb + mc implica c} below, which is essential (although elementary) for the completeness theorem in this paper.

\vspace{1ex}

%Consider a family $\cc{F}$ of functors with comun domain a category 
%$\cc{X}$.
%\begin{proposition} \label{monic-cons ==> cons}
%For a category $\cc{X}$ with pull-backs, any family $\cc{F}$ of pull-back preserving functors with comun domain $\cc{X}$  which is 
%\emph{monic-conservative} is \emph{conservative}. 
%\end{proposition}

Everything in this section is based and contained in 
\cite{sga4}[Ex I, \S 6], for the convenience of the reader we recall the terminology and extract only the parts that we need.

%We will prove that for a regular category $\cc{C}$,  a \textit{monic-conservative} family of regular functors is faithful and conservative.

 Let $\mathcal{F}$ be a family of functors with common domain $\cc{C}$.

\begin{definitionst} ${}$
\begin{enumerate} 
\item $\cc{F}$ \emph{preserves} pullbacks, equalizers, if for every $F \in \cc{F}$ the functor $F$ preserves pullbacks, equalizers, respectively.
\item $\cc{F}$ \emph{reflects} monomorphisms if for every arrow 
${X\mr{u}}Y$ such that for every $F \in \cc{F}$  its image $Fu$ is a monomorphism, it follows that $u$ is a monomorphism.
\item $\cc{F}$ is \emph{conservative} if $\cc{F}$ reflects isomorphisms.
\item $\cc{F}$ is \emph{monic-conservative} if for every monomorphism ${X \mr{u} Y}$ such that for every $F \in \cc{F}$ its image $Fu$ is an isomorphism, it follows that $u$ is an isomorphism.
\item 
$\cc{F}$ is \emph{faithful} if for every pair ${X\mrpair{u}{v}Y}$ such that for every $ F\in \cc{F}$ their images are  equal, $Fu=Fv$, it follows that $u=v$.
\end{enumerate}
\end{definitionst}

%\begin{observation}
%A conservative family is monic-conservative. 
%\end{observation}

\begin{proposition} \label{reflects + m-cons ==> cons}
If $\cc{F}$ reflects monomorphisms and is monic-conservative, then it is conservative.
\end{proposition}
\begin{proof}
Let $u$ be such that for every $F\in\cc{F}$, $Fu$ is an isomorphism. Then in particular $u$ is a monomorphism, and so by the first assumption it is a monomorphism, and in turn by the second assumption it is a isomorphism.  
\end{proof}

We now put a condition that assures us that F reflects monomorphisms.

\begin{proposition}\label{con pullback reflejo monos}
If $\cc{C}$ has pullbacks, $\cc{F}$ preserves them and $\cc{F}$ is monic-conservative, then $\cc{F}$ reflects monomorphism. 
\end{proposition}

\begin{proof}
Let $X \mr{u} Y$ be such that for every $F \in \cc{F}$, $Fu$ is monic. We will prove that the following diagram is a pullback
\[
\xymatrix{X \ar[r]^{id_X} \ar[d]_{id_X} & X \ar[d]^u \\ X \ar[r]_u & Y}
\]

Take a pullback in $\cc{C}$ and $\delta$ as follows:

%
%$$  
%\xymatrix @+3.5ex {X \ar@/_1.5pc/[ddr]_{id_X} \ar@/^/[drr]^{id_X} \ar@{-->}[dr]_{\exists! \delta} & \\ & P \ar@{}[dr]|{p.b.} \ar@<.1ex>@{}[u]|(.36){\equiv} \ar@<.1ex>@{}[l]|(.5){\equiv} \ar[r]^{p_1} \ar[d]_{p_2} & X \ar[d]^u \\ & X \ar[r]_u & Y} 
%$$
% 
$$ 
\xymatrix @+3.5ex
     { 
      X \ar@/_1.5pc/[ddr]_{id_X} 
        \ar@/^/[drr]^{id_X} 
        \ar@{-->}[dr]_{\exists! \delta} 
     & {}
   \\ 
     & P \ar@{}[dr]|{p.b.} 
         \ar@<.1ex>@{}[u]|(.36){\equiv} 
         \ar@<.1ex>@{}[l]|(.5){\equiv} 
         \ar[r]^{p_1} 
         \ar[d]_{p_2} 
      & X \ar[d]^u 
     \\ 
     & X \ar[r]_u 
     & Y
     } 
$$
 
From this diagram we see $\delta$ is a monomorphism (since either projection is a section), and for every $F \in \cc{F}$, $F\delta$ is the isomorphism between the corresponding pullback diagrams. Therefore $\delta$ is an isomorphism. \end{proof}

From \ref{reflects + m-cons ==> cons} and \ref{con pullback reflejo monos} it follows:
\begin{corollary}\label{pb + mc implica c}
If $\cc{C}$ has pullbacks, $\cc{F}$ preserves them and $\cc{F}$ is 
monic-conservative, then $\cc{F}$ is conservative. 
\end{corollary}



%\begin{remark}\label{monic conservative implica conservative}
%If $\cc{C}$ is regular and $\cc{F}$  the set of all regular functors $%\cc{C} \mr{} \cc{E}ns$ is monic conservative, it follows that it is conservative and faithful.
%\end{remark}


%%%%%%%%%%%%%%%%%%%%%%%%%%%%%%%%%%%%%%%%%%%%%%%%%%%%%%%%%%%%%%%%%%%%%%
%%%%%%%%%%%%%%%%%%%%%%% COMENTARIO %%%%%%%%%%%%%%%%%%%%%%%%%%%%%%%%%%%

\begin{comment}
\begin{lemma}\label{si fiel, refleja monos}
If $\cc{F}$ is faithful, then $\cc{F}$ reflects monics(epics). 
\end{lemma}

\begin{proof}
Let $X \mr{u} Y$ in $\cc{C}$ be such that for every $F\in\cc{F}$, $Fu$ is monic. Suppose we have $A \mrpair{x}{y} X$ such that $ux=uy $. Then for every $F\in\cc{F}$, $Fu \cdot Fx=Fu \cdot Fy$ and since $Fu$ is monic, it follows that for every $F\in\cc{F}$, $Fx=Fy$. Thus $x=y$. The dual proposition follows.
\end{proof}


\begin{remark}
The proof of Lemma \ref{con pullback reflejo monos} gives us a technique to prove that under the additional hypothesis of preserving limits (colimits) of a certain type, conservative families reflect limits (colimits) of that type.
\end{remark}

\begin{proposition}\label{cmc implica cf}
If $\cc{C}$ has equalizers (coequalizers), $\cc{F}$ preserves them and $\cc{F}$ is monic/strict-monic (epic/strict-epic)-conservative, then $\cc{F}$ is faithful.
\end{proposition}

\begin{proof}
Take $X\mrpair{u}{v}Y$ in $\cc{C}$ such that for every $F \in \cc{F}$ their images are equal ($Fu=Fv$). An equalizer $E\mr{e}X$  of $u$ and $v$  is monic and its image $Fe$ is an equalizer of $Fu$ and $Fv$ which are equal, so $Fe$ is an isomorphism. Thus $e$ is an isomorphism and consequently $u=v$. For the strict case we need only note that equalizers are strict monics. The dual propositions follow.
\end{proof}

\begin{proposition}
If $\cc{C}$ has equalizers (coequalizers), $\cc{F}$ preserves them and $\cc{F}$ is monic (epic)-conservative, then $\cc{F}$ is conservative.
\end{proposition}

\begin{proof}
It follows from Lemma \ref{si fiel, refleja monos} and the observation made in Proposition \ref{pb + mc implica c}.
\end{proof}


\begin{corollary}
If $\cc{C}$ has equalizers, $\cc{F}$ preserves them and $\cc{F}$ is monic (epic)-conservative, then $\cc{F}$ is conservative and faithfull.
\end{corollary}



\begin{remark}
Even under the strictest limit-preserving conditions we will not be able to guarantee that a faithful family is conservative in any sense. Take the following counterexample: Let $\cc{C}=\{0 \mr{u} 1\}$ and take the family whose only member is the functor $\cc{C} \mr{F} \{*\}$. $\cc{C}$ is a regular category that in fact  has all limits and colimits, $F$ is regular and preserves all limits and colimits , $F$ is faithful but nevertheless does not reflect the isomorphism $Fu$.
\end{remark}
 
\begin{proposition}
If in $\cc{C}$ every bimorphism (a morphism that is both epic and monic) is an isomorphism and $\cc{F}$ is faithful, then $\cc{F}$ is conservative.
\end{proposition}

\end{comment}

%%%%%%%%%%%%%%%%%%%%%%%% FIN COMENTARIO %%%%%%%%%%%%%%%%%%%%%%%%%%%%%%%%%%
%%%%%%%%%%%%%%%%%%%%%%%%%%%%%%%%%%%%%%%%%%%%%%%%%%%%%%%%%%%%%%%%%%%%%%%%%%
